\chapter{ハッシュテーブル}
\chaplabel{hashtables}
\chaplabel{hashing}

ハッシュテーブルは大きな集合$U=\{0,\ldots,2^{#w#}-1\}$の要素#n#個(nは小さい整数)を格納するための効率的な方法だ。
\emph{ハッシュテーブル}という言葉が指すデータ構造はたくさんある。
\index{hash table}%
この章の前半ではハッシュテーブルの一般的な実装ふたつを紹介する。
これはチェイン、または線形探索を使うものだ。

ハッシュテーブルは整数でないデータを格納することもよくある。
この場合\emph{ハッシュ値}というデータに対応する値を使う。
\index{hash code}%
この章の後半ではハッシュ値の生成方法について説明する。

この章で扱う手法にはある範囲からランダムに生成した整数を利用する。
サンプルコードではこのランダム整数はハードコードされた定数になっている。
この定数は空気中のノイズを利用したランダムなビット列から得られる。
