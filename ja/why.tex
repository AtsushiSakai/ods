\chapter*{なぜこの本を読むのか}
\addcontentsline{toc}{chapter}{なぜこの本を読むのか}

データ構造の入門書は多くある。非常に良いものもある。大半は無料ではなく、コンピュータサイエンスの学部生はデータ構造の本にきっとお金を払うだろう。

オンラインで無料で入手できるデータ構造の本もある。非常に良いものもあるが、大部分は古くなっている。著者や出版社が更新をやめるときに無料になったものが大部分である。これらは通常、次の2つの理由から更新できない。(1)著作権は著者または出版社に属し、いずれかの許可が得られないため。(2)書籍の\emph{ソースコード}が利用できないため。つまり、本のWord、WordPerfect、FrameMaker、または\LaTeX{}ソースコードが利用できなかったり、そのソースを扱うソフトウェアのバージョンが利用できなかったりするため。

このプロジェクトはコンピュータサイエンスの学部生がデータ構造の入門書代を支払わなくてよくすることを目指す。この目標を達成するため、この本をオープンソース\ejindex{Open Source}{おーぷんそーす@オープンソース}ソフトウェアプロジェクトのように扱うことにした。この本の\LaTeX{}ソース、\lang{}ソース、およびビルドスクリプトを、著者のWebサイト\footnote {\url{http://opendatastructures.org}}あるいは信頼できるソースコード管理サイト\footnote {\url{https://github.com/patmorin/ods}}からダウンロードできる。

ソースコードはCreative Commons Attributionライセンスで公開されている。つまり誰でも自由にコピー、配布、送信してよい。取り込んで何かを作ってもよい。そしてそれを商業的に利用してもよい。このとき唯一の条件は\emph{attribution}です。つまり派生した作品が\url{opendatastructures.org}のコードやテキストが含むことを認める必要がある。

誰でもソースコード管理システム\texttt{git} \index{git@\texttt{git}}を使って手を加えられる。誰でも本のソースをフォークして別バージョンを作れる(例えば別のプログラミング言語版)。私の望みは、私のやる気や興味が衰えた後も、この本が有用であり続けることだ。
