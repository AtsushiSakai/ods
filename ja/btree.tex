\chapter{外部メモリの探索}
\chaplabel{btree}
この本を通じて計算のモデルとしては\secref{model}で定義した#w#ビットのワードRAMモデルを使ってきた。
これは、コンピュータのランダムアクセスメモリはデータ構造内のすべてのデータを格納できるくらい大きいと暗に仮定していたということである。
しかし時にはこの仮定が成り立たないこともある。
大きすぎてどんなコンピュータのメモリにも収まりきらないデータの集まりが存在するのである。
このような場合、ハードディスク・SSD・ネットワーク越しのサーバーなどの外部ストレージにデータを蓄えざるを得ない。

\index{external storage}%
\index{external memory}%
\index{hard disk}%
\index{solid-state drive}%
外部ストレージへのデータアクセスは非常に遅い。
この本を書くのに使っているコンピュータでは、ハードディスクへの平均アクセス時間は19ms、SSDの場合は0.3msである。
これに対してにランダムアクセスメモリの場合は0.000113ms未満である。
RAMへのアクセスはSSDと比べて2500倍、HDDと比べて160000倍以上高速である。

%  HDD: Fantom ST3000DM001-9YN166 USB 3 external drive (3TB)
%  SSD: ATA OCZ-AGILITY 3 (60GB)
%  Mem: Mushkin Enhanced Essentials 4GB (2 x 2GB) 204-Pin DDR3 
%       SO-DIMM DDR3 1066 (PC3 8500) Dual Channel Kit Laptop Memory
%       Model 996643
%  RAM speed was estimated using this program:
% #include<stdlib.h>
% #include<stdio.h>
% #include<time.h>
% 
% int main(void) {
%    unsigned *a, x, i, n = 50000000;
%    clock_t start, stop;
% 
%    start = clock();
%    a = malloc(sizeof(unsigned)*n);
%    for(i = 0; i < n; i++) {
%      x |= a[rand()%n];
%    }
%    stop = clock();
%    printf("x=%x, %g\n", x, (((double)(stop-start))/(double)CLOCKS_PER_SEC)/(double)n);
%    free(a);
%    return 0;
% }

これらの速度は典型的な値である。
RAMへのランダムアクセスはHDDやSSDへのランダムアクセスと比べて数千倍高速である。
しかしアクセスにかかる時間だけを考えれば十分というわけではない。
HHDやSSDのバイトにアクセスするときには、実際は
\emph{ブロック}単位での読み出しが行われる。
\index{block}%
XXX: 著者の環境の話?
コンピュータに接続されるドライブは大きさ4096のブロックを持つ。
1バイトの読み出しをする度にドライブは4096バイトを返してくる。
このことを踏まえてデータ構造を設計すれば、この4096バイトを操作にうまく利用するのがよいだろう。

% morin@peewee:~/git/ods/latex$ sudo blockdev --report
% RO    RA   SSZ   BSZ   StartSec            Size   Device
% rw   256   512  4096          0     60022480896   /dev/sda   SSD
% rw   256  4096  4096        504   3000581885952   /dev/sdb1  HDD

これが\emph{外部メモリモデル}の背後にあるアイデアである。
\index{external memory model}%
\figref{em}にはこれを図示した。
このモデルではコンピュータはすべてのデータが保存されている大きな外部メモリにアクセスできる。
このメモリは\emph{ブロック}に分割されており、
\index{block}%
それぞれのブロックは$B$ワードからなる。
このコンピュータは有限の内部メモリも利用でき、ここで計算を実行できる。
内部メモリと外部メモリの間でブロックを転送するのには一定の時間がかかる。
内部メモリでの計算は\emph{タダ}である。つまり一切の時間がかからない。
この仮定は奇妙に感じるかもしれないが、外部メモリへのアクセスが非常に遅いことを強調した結果である。

\begin{figure}
  \centering{\includegraphics[width=\ScaleIfNeeded]{figs/em}}
  \caption[The external memory model]{In the external memory model,
  accessing an individual item, #x#, in the external memory requires
  reading the entire block containing #x# into RAM.}
  \figlabel{em}
\end{figure}

本格的な外部メモリモデルでは内部メモリの大きさもパラメータである。
しかし、この章で扱うデータ構造では内部メモリのサイズは$O(B+\log_B #n#)$で十分である。
これは定数個だけのブロックと、高さ$O(\log_B #n#)$だけのスタックに必要なメモリである。
多くの場合$O(B)$が支配的な項である。
例えば比較的小さい値$B=32$であっても、すべての$#n#\le 2^{160}$について$B\ge \log_B #n#$が成り立つ。
十進法では、$B\ge \log_B #n#$が以下の範囲で成り立つ。
\[
#n# \le 1\,461\,501\,637\,330\,902\,918\,203\,684\,832\,716\,283\,019\,655\,932\,542\,976 \enspace 
. \]

\section{Block Store}
\index{block store}%
\index{BlockStore@#BlockStore#}%
様々なデバイスが外部メモリの概念には含まれる。
それぞれブロックサイズが定義されており、独自のシステムコールによってアクセスされる。
説明を単純にし、一般的なアイデアに焦点をあてるため、外部メモリのデバイスを#BlockStore#と呼ばれるオブジェクトに隠蔽する。
#BlockStore#はメモリブロックの集まりを格納する。
各ブロックの大きさは$B$である。
各ブロックはインデックスによって一意に特定される。
#BlockStore#は次の操作をサポートする。

\begin{enumerate}
  \item #readBlock(i)#:インデックス#i#のブロックの内容を返す。
  \item #writeBlock(i,b)#:インデックス#i#のブロックに#b#の内容を書く。
  \item #placeBlock(b)#:新たなインデックスを返し、そこに#b#の内容を書く。
  \item #freeBlock(i)#:インデックス#i#のブロックを開放する。これは指定したブロックの内容をもう使わず、このブロックに割り当てられていた外部メモリは別の用途に使ってよいことを示す。
\end{enumerate}

#BlockStore#は$B$バイトのブロックに分割されたディスク上のファイルだと考えるのが最も想像しやすいだろう。
このとき、#readBlock(i)#・#writeBlock(i,b)#は、このファイルのバイト列$#i#B,\ldots,(#i#+1)B-1$の読み・書きである。
さらに、#BlockStore#は利用可能なブロックからなる\emph{フリーリスト}を保持してもよい。
#freeBlock(i)#により解放されたブロックはフリーリストに追加される。
こうすれば、#placeBlock(b)#はフリーリストのブロックを使い、利用可能なブロックがないときだけファイルの末尾に新しいブロックを追加するようにできる。
