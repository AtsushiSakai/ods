\chapter{配列を使ったリスト}
\chaplabel{arrays}

この章では、\emph{backing array}と呼ばれる配列にデータを格納する、#List#・#Queue#インターフェースの実装について検討する。
\emph{backing array}.
\index{backing array}%
次の表は、この章で説明するデータ構造の操作時間を要約したものだ。

\newlength{\tabsep}
\setlength{\tabsep}{\itemsep}
\addtolength{\tabsep}{\parsep}
\addtolength{\tabsep}{-2pt}
\begin{center}
\vspace{\tabsep}
\begin{tabular}{|l|l|l|} \hline
 & #get(i)#/#set(i,x)# & #add(i,x)#/#remove(i)# \\ \hline
#ArrayStack# & $O(1)$ & $O(#n#-#i#)$ \\
#ArrayDeque# & $O(1)$ & $O(\min\{#i#,#n#-#i#\})$ \\
#DualArrayDeque# & $O(1)$ & $O(\min\{#i#,#n#-#i#\})$ \\
#RootishArrayStack# & $O(1)$ & $O(#n#-#i#)$ \\ \hline
\end{tabular}
\vspace{\tabsep}
\end{center}

データをひとつの配列に入れるデータ構造には以下のような利点・欠点がある。

\begin{itemize}
  \item 配列の任意の要素には一定の時間でアクセスできる。
  そのため#get(i)#・#set(i,x)#はいずれも定数時間で実行される。

  \item 配列は動的ではない。リストの真ん中付近に要素を追加・削除するためには、隙間を作ったり埋めたりするために多くの要素が移動することになる。
  #add(i,x)#・#remove(i)#操作の実行時間が#n#・#i#に依存するのはこのためだ。

  \item 配列は伸び縮みしない。
  backing arrayのサイズより多くの要素をデータ構造に入れるには、新しい配列を割当て、古い配列の要素をそちらにコピーする必要がある。
  この操作のコストは大きい。
\end{itemize}

3つめの点が重要だ。
上記の表に記載された実行時間はbacking arrayの拡大・縮小にかかるコストは含まれていない。
後述するように、注意深く設計すれば、backing arrayの拡大・縮小のコストを加味しても\emph{平均的な}実行時間はほとんど増えない。
より正確に言うと、空のデータ構造からはじめて、#add(i,x)#・#remove(i)#を#m#回実行するときの、backing arrayの拡大・縮小のための合計コストは$O(m)$である。
個々の操作のコストは大きいが、#m#個の操作にわたる償却コストを考えれば、ひとつの操作あたりのコストは$O(1)$なのだ。

\section{ArrayStack:配列を使った高速なスタック操作}
\seclabel{arraystack}

\index{ArrayStack@#ArrayStack#}%

#ArrayStack#は\emph{backing array}#a#を使ったリストインターフェースの実装だ。
リストの#i#番目の要素を#a[i]#とする。
ほとんどの場合#a#は実際に必要な値よりも大きい。
そのため整数#n#によって実際に#a#に入っている要素数を表す。
つまり、リストの要素は#a[0]#,\ldots,#a[n-1]#に格納される。
また、関係$#a.length# \ge #n#$が常に成り立つ。

\codeimport{ods/ArrayStack.a.n.size()}

\subsection{基本}
#get(i)#や#set(i,x)#を使って#ArrayStack#の要素を読み書きする方法は簡単である。
境界チェックをしたあと単に#a[i]#を返すか、#a[i]#を書き換えるかすればよいのだ。

\codeimport{ods/ArrayStack.get(i).set(i,x)}

#ArrayStack#に要素を追加・削除するための実装を\figref{arraystack}に示す。
#add(i,x)#では、まず#a#が既に一杯かどうかを調べる。
もしそうなら#resize()#を呼び出して、#a#を大きくする。
#resize()#の実装方法については後述する。
#resize()#の直後では$#a.length# > #n#$であることだけ知っていれば今は十分である。
あとは#x#が入るように$#a[i]#,\ldots,#a[n-1]#$をひとつずつ右に移動させ、#a[i]#をxにし、#n#を1増やせばよい。

\begin{figure}
  \begin{center}
    \includegraphics[scale=0.90909]{figs/arraystack}
  \end{center}
  \caption[Adding to an ArrayStack]{
  #add(i,x)#・#remove(i)#を#ArrayStack#に実行する。
  矢印は要素のコピーを表す。
  #resize()#を呼ぶ操作にはアスタリスクを付した。}
  \figlabel{arraystack}
\end{figure}

\codeimport{ods/ArrayStack.add(i,x)}
#resize()#が呼ばれるかもしれないが、このコストを無視すれば#add(i, x)#のコストは#x#を入れる場所を作るためにシフトする要素数に比例する。つまり、この操作のコストは(リサイズのことを無視すれば)$O(#n#-#i#)$である。

#remove(i)#の実装も似ている。
$#a[i+1]#,\ldots,#a[n-1]#$を左にひとつシフトし(#a[i]#は書き換えられる))、#n#の値をひとつ小さくする。
その後#n#が#a.length#より小さすぎないか、具体的には$#a.length# \ge 3#n#$を確認する。
もしそうなら、#resize()#を呼んで#a#を小さくする。

\codeimport{ods/ArrayStack.remove(i)}
#resize()#が呼ばれるかもしれないが、このコストを無視すれば#remove(i)#のコストはシフトする要素数に比例し、$O(#n#-#i#)$である。

\subsection{拡張と収縮}

#resize()#の実装は単純だ。
大きさ$2#n#$の新しい配列#b#を割当て、#n#個の#a#の要素を#b#のはじめの#n#個としてコピーする。そして#a#を#b#に置き換える。
よって#resize()#の呼び出し後は$#a.length# = 2#n#$が成り立つ。

\codeimport{ods/ArrayStack.resize()}

#resize()#の実際のコストの計算も簡単だ。
大きさ$2#n#$の配列#b#を割当て、#n#要素をコピーする。
これは$O(#n#)$の時間がかかる。

前節からの実行時間分析は#resize()#のコストを無視していた。
この節では\emph{償却解析}として知られる方法でこれを解決する。
この手法は個々の#add(i, x)#・#remove(i)#における#resize()#のコストを求めるわけではない。
代わりに、#m#個の#add(i, x)#・#remove(i)#からなる操作列の間に呼ばれる#resize()#すべてのことを考える。

次の補題を示す。
\begin{lem}\lemlabel{arraystack-amortized}
  空の#ArrayStack#が作られたあと、$m\ge 1$個の#add(i, x)#・#remove(i)#からなる操作の列が順に実行されるとき、この間に呼ばれる#resize()#の合計実行時間は#O(m)#である。
\end{lem}

%XXX: ここまで完了
\begin{proof}
#resize()#が呼ばれるとき、その前の#resize()#呼び出しから#add#・#remove#が実行された回数は$#n#/2-1$以下である。
$i$回目の#resize()#呼び出しの際の#n#を$#n#_i$とする。
$r$を#resize()#の呼び出し回数とする。
このとき、#add(i,x)#と#remove(i)#の合計呼び出し回数は次の関係を満たす。
\[
  \sum_{i=1}^{r} (#n#_i/2-1) \le m \enspace
\]

これを変形すると次の式が得られる。
\[
  \sum_{i=1}^{r} #n#_i \le 2m + 2r  \enspace
\]

$r \leq m$より#resize()#呼び出しのための実行時間の合計は次のようになる。
\[
\sum_{i=1}^{r} O(#n#_i) \le O(m+r) = O(m)  \enspace
\]

あとは$(i-1)$から$i$回目の#resize()#の間に#add(i,x)#か#remove(i)#が呼ばれる回数が$#n#_i/2$以下であることを示す。

2つの場合が考えられる。
ひとつは#resize()#が#add(i,x)#に呼ばれる場合で、これはbacking arrayが一杯になるとき、つまり$#a.length# = #n#=#n#_i$が成り立つときだ。
直前の#resize()#を考えよう。この#resize()#の直後、#a#の大きさは#a.length#だが#a#の要素数は$#a.length#/2=#n#_i/2$以下であった。
しかし#a#の要素数は今では$#n#_i=#a.length#$なのだから、前の#resize()#から$#n#_i/2$回以上は#add(i,x)#が呼ばれたことがわかる。

もうひとつ考えられるのは、#resize()#が#remove(i)#に呼ばれる場合で、このとき$#a.length# \ge 3#n#=3#n#_i$である。
前の#resize()#の直後では#a#の要素数は$#a.length/2#-1$以下であった。
\footnote{この数式における${}-1$は、特別なケース$#n#=0$かつ$#a.length# = 1$を考慮したものだ。}
今#a#には$#n#_i\le#a.length#/3$個の要素が入っている。
よって、直前の#resize()#以降に実行された#remove(i)#の回数の下界は次のように計算できる。

  \begin{align*}
      R & \ge #a.length#/2 - 1 - #a.length#/3 \\
        & = #a.length#/6 - 1 \\
        & = (#a.length#/3)/2 - 1 \\
        & \ge #n#_i/2 -1\enspace .
  \end{align*}

いずれの場合でも、$(i-1)$から$i$回目の#resize()#の間に#add(i,x)#か#remove(i)#が呼ばれる回数の合計は$#n#_i/2-1$以上である。
\end{proof}

\subsection{要約}

次の定理は#ArrayStack#の性能を整理するものだ。

\begin{thm}\thmlabel{arraystack}
  #ArrayStack#は#List#インターフェースを実装する。
  #resize()#のコストを無視すると、#ArrayStack#における各操作の実行時間は、
  \begin{itemize}
    \item #get(i)#・#set(i,x)#の実行時間は$O(1)$である。
    \item #add(i,x)#・#remove(i)#の実行時間は$O(1+#n#-#i#)$である。
  \end{itemize}
  空の#ArrayStack#から任意の$m$個の#add(i,x)#・#remove(i)#からなる操作の列を実行する。このときすべての#resize()#にかかる時間の合計は$O(m)$である。
\end{thm}

#ArrayStack#は#Stack#を実装する効率的な方法である。
特に#push(x)#は#add(n,x)#、#pop()#は#remove(n-1)#のようにそれぞれ実装できる。
またいずれの操作も償却実行時間$O(1)$である。

\section{#FastArrayStack#:最適化されたArrayStack}

#ArrayStack#が主にやっていることは、データの(#add(i,x)#と#remove(i)#のための)シフトと(#resize()#のための)コピーである。

素朴な実装では#for#ループを使うだろう。
しかしデータのコピーや移動用の効率的な機能があるプログラミング環境も多いだろう。
C言語には#memcpy(d,s,n)#・#memmove(d,s,n)#関数がある。
C++言語には#std::copy(a0,a1,b)#アルゴリズムがある。
Javaには#System.arraycopy(s,i,d,j,n)#メソッドがある。
\index{memcpy@#memcpy(d,s,n)#}%
\index{std::copy@#std::copy(a0,a1,b)#}%
\index{System.arraycopy@#System.arraycopy(s,i,d,j,n)#}%

\cppimport{ods/FastArrayStack.add(i,x).remove(i).resize()}
\javaimport{ods/FastArrayStack.add(i,x).remove(i).resize()}

これらは最適化されており、#for#ループを使うよりかなり速くコピーができる特殊な機械命令を使うかもしれない。
これらを使っても漸近的な実行時間は減らないのだが、やる価値のある最適化ではある。

C++やJavaの実装で高速な配列コピーを使って2〜3倍の高速化できたこともある。どのくらい速くなるかは環境によるので是非試してみてほしい。

\section{ArrayQueue:配列を使ったキュー}
\seclabel{arrayqueue}

\index{ArrayQueue@#ArrayQueue#}%
この節ではFIFO(先入れ先出し)キューを実装するデータ構造#ArrayQueue#を紹介する。
(#add(x)#によって)要素が追加されたのと同じ順番で、(#remove()#によって)キューから要素が削除される。

FIFOキューの実装に#ArrayStack#はあまり向いていない。
一方の端から要素を追加し他方から要素を削除することになるので、賢明な選択ではないのだ。
2つの操作のうち一方はリストの先頭を変更することになる。つまり、$#i#=0$で#add(i,x)#か#remove(i)#を呼びだす。
このとき、#n#に比例する実行時間がかかってしまう。

配列を使ったキューの効率的な実装は、もし無限の配列#a#があれば簡単だろう。
次に削除する要素を追跡するインデックス#j#と、キューの要素数#n#を記録しておけばよい。
そうすればキューの要素は以下の場所に入っている。
\[ #a[j]#,#a[j+1]#,\ldots,#a[j+n-1]# \enspace \]
まず#i#, #j#を0に初期化する。
要素を追加するときは、#a[j+n]#に要素を入れて、#n#をひとつ増やす。
要素を削除するときは、#a[j]#から要素を取り出し、#j#をひとつ増やし、#n#をひとつ減らす。

もちろんこの方法の問題点は無限の配列が必要になることだ。
#ArrayQueue#は有限の配列#a#と\emph{剰余算術}で無限配列を模倣する。
\index{modular arithmetic}%
剰余算術は時間の計算をするときに使っているものだ。
例えば10:00に5時間を足すと3:00になる。
形式的に書けば次のようになる。
\[
    10 + 5 = 15 \equiv 3 \pmod{12} \enspace
\]
数式の後半は「12を法として15と3は合同である」と読む。
$\bmod$は次のように二項演算と考えてもよい。
\[
   15 \bmod 12 = 3 \enspace .
\]

より一般的には
整数$a$と正整数$m$について、ある整数kが存在し$a = r + km$をみたす$\{0, \ldots, m-1 \} $の一意な要素を$a \bmod m $と書く。
雑に言うと$ r $とは$ a $を$ m $で割った余りである。
CやC++、Ruby、Javaなど多くのプログラミング言語ではmod演算子は\%で表される。

剰余算術は無限配列を模倣するのに便利である。
$#i#\bmod #a.length#$は常に$0,\ldots,#a.length-1#$の値を取ることを利用して、配列の中にキューの要素をうまく入れられるのだ。

\[
#a[j%a.length]#,#a[(j+1)%a.length]#,\ldots,#a[(j+n-1)%a.length]# \enspace
\]

これは#a#を\emph{循環配列}として使っている。
\index{circular array}%
\index{array!circular}%
配列の添字が$#a.length#-1$を超えると、配列の先頭に戻ってくるのである。

残りの問題は、#ArrayQueue#の要素数が#a#の大きさを超えてはならないことだ。

\codeimport{ods/ArrayQueue.a.j.n}

#ArrayQueue#に対して#add(x)#・#remove()#からなる操作の列を実行する様子を\figref{arrayqueue}に示す。
#add(x)#はまず#a#が一杯かどうかを確認し、必要に応じて#resize()#を呼んで#a#の容量を増やす。
続いて#x#を#a[(j+n)%a.length]#に入れて、#n#をひとつ増やせばよい。

\begin{figure}
  \begin{center}
    \includegraphics[scale=0.90909]{figs/arrayqueue}
  \end{center}
  \caption[Adding and removing from an ArrayQueue]{A sequence of #add(x)# and #remove(i)# operations on an
  #ArrayQueue#.  Arrows denote elements being copied.  Operations that
  result in a call to #resize()# are marked with an asterisk.}
  \figlabel{arrayqueue}
\end{figure}

\codeimport{ods/ArrayQueue.add(x)}

最後になるが、#resize()#操作は#ArrayStack#のものとよく似ている。
大きさ$2#n#$の新しい配列#b#を割当て、
\[
   #a[j]#,#a[(j+1)%a.length]#,\ldots,#a[(j+n-1)%a.length]#
\]
を
\[
   #b[0]#,#b[1]#,\ldots,#b[n-1]#
\]
にコピーし、$#j#=0$とする。

\codeimport{ods/ArrayQueue.resize()}

\subsection{要約}

次の定理は#ArrayQueue#の性能を整理するものだ。

\begin{thm}
  #ArrayStack#は#List#インターフェースを実装する。
  #resize()#のコストを無視すると、#ArrayStack#における各操作の実行時間は、
  \begin{itemize}
    \item #get(i)#・#set(i,x)#の実行時間は$O(1)$である。
    \item #add(i,x)#・#remove(i)#の実行時間は$O(1+#n#-#i#)$である。
  \end{itemize}
  空の#ArrayStack#から任意の$m$個の#add(i,x)#・#remove(i)#からなる操作の列を実行する。このときすべての#resize()#にかかる時間の合計は$O(m)$である。
  #ArrayQueue#は(FIFO)#Queue#インターフェースの実装である。
  #resize()#のコストを無視すると、#ArrayStack#は#add(x)#・#remove()#を$O(1)$の時間で実行できる。
  さらに、空の#ArrayStack#に対して長さ#m#の任意の#add(i,x)#・#remove(i)#からなる操作の列を実行するとき、#resize()#に使われる実行時間の合計は$O(m)$である。
\end{thm}

\section{ArrayDeque:配列を使った高速な双方向キュー}
\seclabel{arraydeque}

\index{ArrayDeque@#ArrayDeque#}%
前節の#ArrayQueue#は、一方の端からは追加が他方の端から削除が効率的にできる列を表現するデータ構造だった。#ArrayDeque#は両方の端で効率的な追加と削除ができるデータ構造である。
#ArrayQueue#を表現するために使った循環配列をここでもまた使って#List#インタフェースを実装する。

\codeimport{ods/ArrayDeque.a.j.n}

#ArrayDeque#における#get(i)#と#set(i,x)#の実装は難しくない。
配列の要素$#a[#{#(j+i)#\bmod #a.length#}#]#$を読み書きすればよいのだ。

\codeimport{ods/ArrayDeque.get(i).set(i,x)}

#add(i,x)#の実装はもう少し興味深い。
まず、#a#が一杯かどうかを確認し、必要に応じて#resize()#を呼ぶ。
ここで、#i#が小さいとき(0に近いとき)と大きいとき(#n#に近いとき)に、特に効率的に操作したいのだということを覚えておいてほしい。
つづいて、$#i#<#n#/2$かどうかを確認する。
もしそうなら、$#a[0]#,\ldots,#a[i-1]#$をそれぞれひとつずつ左にずらす。
そうでないなら、$#a[i]#,\ldots,#a[n-1]#$をそれぞれひとつずつ右にずらす。
#add(i,x)#と#remove(x)#の説明として\figref{arraydeque}を見てほしい。

\begin{figure}
  \begin{center}
    \includegraphics[scale=0.90909]{figs/arraydeque}
  \end{center}
  \caption[Adding and removing from an ArrayDeque]{A sequence of #add(i,x)# and #remove(i)# operations on an
  #ArrayDeque#.  Arrows denote elements being copied.}
  \figlabel{arraydeque}
\end{figure}

\codeimport{ods/ArrayDeque.add(i,x)}

このようにシフトを行うことで、#add(i,x)#は$\min\{ #i#, #n#-#i# \}$より多くの要素をシフトしなくてよい。
よって#add(i,x)#の(#resize()#のことを無視した)実行時間は$O(1+\min\{#i#,#n#-#i#\})$である。

#remove(i)#の実装も似たようなものだ。
$#a[0]#,\ldots,#a[i-1]#$をそれぞれ右にひとつずつシフトするか、$#a[i+1]#,\ldots,#a[n-1]#$をそれぞれ左にひとつずつシフトするか、$#i#<#n#/2$かどうかに応じてどちらかを行う。
やはり#remove(i)#も$O(1+\min\{#i#,#n#-#i#\})$だけの時間で要素を動かし終えることができる。

\codeimport{ods/ArrayDeque.remove(i)}

\subsection{要約}

次の定理は#ArrayDeque#の性能を整理するものだ。
  #ArrayDeque#は#List#インターフェースを実装する。
  #resize()#のコストを無視すると、#ArrayDeque#における各操作の実行時間は、
  \begin{itemize}
    \item #get(i)#・#set(i,x)#の実行時間は$O(1)$である。
    \item #add(i,x)#・#remove(i)#の実行時間は$O(1+\min\{#i#,#n#-#i#\})$である。
  \end{itemize}
  空の#ArrayDeque#から任意の$m$個の#add(i,x)#・#remove(i)#からなる操作の列を実行する。このとき#resize()#にかかる時間の合計は$O(m)$である。

\section{#DualArrayDeque#:2つのスタックから作った双方向キュー}
\seclabel{dualarraydeque}

\index{DualArrayDeque@#DualArrayDeque#}%

次は2つの#ArrayStack#を使って#ArrayDeque#に近い性能を示すデータ構造#DualArrayDeque#を紹介する。
#DualArrayDeque#の漸近的な性能は#ArrayDeque#より優れているわけではないのだが、2つのシンプルなデータ構造を組み合わせてより高度なデータ構造を作る良い例なのでここで扱う価値がある。

#DualArrayDeque#は、2つの#ArrayStack#を使ってリストを表現する。
#ArrayStack#では終端付近の要素を高速に修正できたことを思い出してほしい。
#DualArrayDeque#は#front#と#back#という名のふたつの#ArrayStack#を後ろ合わせに配置する。
そのため両端での高速な操作が可能だ。

\codeimport{ods/DualArrayDeque.front.back}

#DualArrayDeque#は要素数#n#を明示的に保持しない。
$#n#=#front.size()# + #back.size()#$なのでその必要がないのだ。
しかし#DualArrayDeque#の解析では相変わらず#n#で要素数を表すことにする。

\codeimport{ods/DualArrayDeque.front.back}

ひとつめの#ArrayStack#である#front#には$0,\ldots,#front.size()#-1$番目の要素を、逆さまの順番で格納する。
もうひとつの#ArrayStack#である#back#には$#front.size()#,\ldots,#size()#-1$番目の要素を普通の順番で格納する。
こうして、#front#か#back#に対する#get(i)#か#set(i,x)#を適切に呼び出すことで、#get(i)#・#set(i,x)#を$O(1)$の実行時間で実現できる。

\codeimport{ods/DualArrayDeque.get(i).set(i,x)}

#front#には逆順に要素を蓄えているので、インデックス$#i#<#front.size()#$は#front#の$#front.size()#-#i#-1$番目の要素である。

#DualArrayDeque#における要素の追加・削除は\figref{dualarraydeque}を見てほしい。
#add(i,x)#は#front#か#back#を必要に応じて操作する。

\begin{figure}
  \begin{center}
    \includegraphics[scale=0.90909]{figs/dualarraydeque}
  \end{center}
  \caption[Adding and removing in a DualArrayDeque]{A sequence of #add(i,x)# and #remove(i)# operations on a
  #DualArrayDeque#.  Arrows denote elements being copied.  Operations that
  result in a rebalancing by #balance()# are marked with an asterisk.}
  \figlabel{dualarraydeque}
\end{figure}

\codeimport{ods/DualArrayDeque.add(i,x)}

%XXX: ここまで
#add(i,x)#はふたつの#ArrayStack# #front#・#back#のバランスを調整するために#balance()#を呼び出す。
#balance()#の実装は後で説明するが、$#size()#<2$であるときを除いて#front.size()#と#back.size()#は三倍以上離れないことを#balance()#は保証することを知っておけば十分である。
具体的には$3\cdot#front.size()# \ge #back.size()#$と$3\cdot#back.size()# \ge #front.size()#$であることを保証する。

つづいて#add(i,x)#のうち#balance()#のことを無視したコストを求める。
$#i#<#front.size()#$のとき#add(i,x)#は$#front.add(front.size()-i-1,x)#$で実装できる。
#front#は#ArrayStack#なのでこのコストは次のようになる。
\begin{equation}
  O(#front.size()#-(#front.size()#-#i#-1)+1) = O(#i#+1) \enspace
  \eqlabel{das-front}
\end{equation}
一方
$#i#\ge#front.size()#$のとき#add(i,x)#は$#back.add(i-front.size(),x)#$で実装できる。
このコストは次のようになる。
\begin{equation}
  O(#back.size()#-(#i#-#front.size()#)+1) = O(#n#-#i#+1) \enspace
  \eqlabel{das-back}
\end{equation}

$#i#<#n#/4$のときはひとつめのケース\myeqref{das-front}に該当する。
$#i#\ge 3#n#/4$のときはふたつめのケース\myeqref{das-back}に該当する。
$#n#/4\le#i#<3#n#/4$のときは、#front#と#back#どちらを操作するかわからない。
しかし$#i#\ge #n#/4$かつ$#n#-#i#>#n#/4$なのでいずれ場合も実行時間は$O(#n#)=O(#i#)=O(#n#-#i#)$である。
以上をまとめると次のようになる。
\[
     \mbox{Running time of } #add(i,x)# \le 
          \left\{\begin{array}{ll}
            O(1+ #i#) & \mbox{if $#i#< #n#/4$} \\
            O(#n#) & \mbox{if $#n#/4 \le #i# < 3#n#/4$} \\
            O(1+#n#-#i#) & \mbox{if $#i# \ge 3#n#/4$}
          \end{array}\right.
\]
ゆえに#add(i,x)#の実行時間は#balance()#の呼び出しのことを無視すれば$O(1+\min\{#i#, #n#-#i#\})$である。

\codeimport{ods/DualArrayDeque.remove(i)}

\subsection{バランスの調整}

最後に#add(i,x)#と#remove(i)#によって実行される#balance()#の説明をする。
この操作は#front#・#back#のどちらも大きく(または小さく)なりすぎないことを保証するものだ。
要素数が2以上のとき、#front#も#back#も$#n#/4$以上の要素を含むようにするのだ。
そうでないときは要素を動かして、#front#・#back#がそれぞれちょうど$\lfloor#n#/2\rfloor$・$\lceil#n#/2\rceil$個の要素を持つようにする。

\codeimport{ods/DualArrayDeque.balance()}

ここまでで分析するようなことは特にないだろう。
#balance()#がバランス調整をするとき$O(#n#)$個の要素を動かすので$O(#n#)$の時間がかかる。
#balance()#は#add(i,x)#・#remove(i)#で毎回呼ばれるのでこれは一見好ましくない。
しかし次の補題より、#balance()#のための平均時間は定数であることがわかる。

\begin{lem}\lemlabel{dualarraydeque-amortized}
  空の#DualArrayDeque#に対して$m\ge 1$個の#add(i,x)#・#remove(i)#からなる操作の列を順に実行する。
  このとき#balance()#の実行時間の合計は$O(m)$である。
\end{lem}

\begin{proof}
  #balance()#が要素を動かすとき、前に#balance()#が要素を動かしたときから呼ばれた#add(i,x)#・#remove(i)#の合計数は$#n#/2-1$以下であることを示す。
  \lemref{arraystack-amortized}の証明と同様に、これを示せば#balance()#の合計時間が$O(m)$であることを示すには十分である。

  ここでは
  \emph{ポテンシャル法}.
  \index{potential}%
  \index{potential method}%
  として知られる手法を解析に用いる。
  #DualArrayDeque#の\emph{ポテンシャル}$\Phi$を#front#と#back#の要素数の差と定義する。
  \[  \Phi = |#front.size()# - #back.size()#| \enspace . \]
  このポテンシャルの興味深い性質は、バランス調整を行わない#add(i,x)#・#remove(i)#の呼び出しはポテンシャルを1増やすことだ。

  
  次の式が成り立つことから、#balance()#が要素を動かした直後にはポテンシャル$\Phi_0$は1以下であることがわかるだろう。
  \[ \Phi_0 = \left|\lfloor#n#/2\rfloor-\lceil#n#/2\rceil\right|\le 1  \enspace .\]

  要素を動かす#balance()#の直前には$3#front.size()# < #back.size()#$であったと仮定して一般性を失わない。
  次の式が成り立つ。
  \begin{eqnarray*}
   #n# & = & #front.size()#+#back.size()# \\
       & < & #back.size()#/3+#back.size()# \\
       & = & \frac{4}{3}#back.size()#
  \end{eqnarray*}
  このときのポテンシャルは次のように評価できる。
  \begin{eqnarray*}
  \Phi_1 & = & #back.size()# - #front.size()# \\
      &>& #back.size()# - #back.size()#/3 \\
      &=& \frac{2}{3}#back.size()# \\
      &>& \frac{2}{3}\times\frac{3}{4}#n# \\
      &=& #n#/2
  \end{eqnarray*}
  以上より、前に#balance()#によって要素を動かしてから、#add(i,x)#・#remove(i)#が呼ばれた回数は$\Phi_1-\Phi_0 > #n#/2-1$ 以上である。
\end{proof}

\subsection{要約}

次の定理は#DualArrayDeque#の性質をまとめるものだ。
\begin{thm}\thmlabel{dualarraydeque}
  #DualArrayDeque#は#List#インターフェースを実装する。
  #resize()#・#balance()#のコストを無視すると、#DualArrayDeque#における各操作の実行時間は、
  \begin{itemize}
    \item #get(i)#・#set(i,x)#の実行時間は$O(1)$である。
    \item #add(i,x)#・#remove(i)#の実行時間は$O(1+\min\{#i#,#n#-#i#\})$である。
  \end{itemize}
  空の#DualArrayDeque#から任意の$m$個の#add(i,x)#・#remove(i)#からなる操作の列を実行する。このときすべての#resize()#にかかる時間の合計は$O(m)$である。
  Furthermore, beginning with an empty #DualArrayDeque#, any sequence of $m$
  #add(i,x)# and #remove(i)# operations results in a total of $O(m)$
  time spent during all calls to #resize()# and #balance()#.
\end{thm}
  \begin{itemize}
    \item #get(i)#・#set(i,x)#の実行時間は$O(1)$である。
    \item #add(i,x)#・#remove(i)#の実行時間は$O(1+\min\{#i#,#n#-#i#\})$である。
  \end{itemize}
  空の#DualArrayDeque#から任意の$m$個の#add(i,x)#・#remove(i)#からなる操作の列を実行する。このとき#resize()#・#balance()#にかかる時間の合計は$O(m)$である。

\section{2.6 RootishArrayStack:空間効率に優れた配列スタック}
\seclabel{rootisharraystack}

\index{RootishArrayStack@#RootishArrayStack#}%

ここまで紹介してきたデータ構造には共通の欠点がある。
データは1つか2つの配列に入れ、配列のサイズを変更しないようにしているので、配列に隙間が多い傾向がある点だ。
例えば#resize()#直後の#ArrayStack#では配列#a#は半分しか埋まっていない。
#a#は3分の1しか埋まっていないことさえある。

この節ではこの無駄なスペースの問題を解決する#RootishArrayStack#というデータ構造を紹介する。
#RootishArrayStack#は#n#個の要素を$O(\sqrt{#n#})$個の配列に格納する。
この配列では使われていない場所は常に$O(\sqrt{#n#})$以下である。
残りのすべての場所にはデータが入っているのだ。
つまり#n#個の要素を入れるとき、無駄になるスペースは$O(\sqrt{#n#})$以下である。

#RootishArrayStack#は\emph{ブロック}と呼ばれる#r#個の配列に要素を格納する。この配列は$0,1,\ldots,#r#-1$と添字付けられる。
\figref{rootisharraystack}を見てほしい。
ブロック$b$は$b+1$個の要素を含む。
すなわち、#r#個のブロックが含む要素数の合計は次のように計算できる。
\[
  1+ 2+ 3+\cdots +#r# = #r#(#r#+1)/2
\]
この式が成り立つのは\figref{gauss}を見ればわかるだろう。

\begin{figure}
  \begin{center}
    \includegraphics[width=\ScaleIfNeeded]{figs/rootisharraystack}
  \end{center}
  \caption[Adding and removing in a RootishArrayStack]{A sequence of #add(i,x)# and #remove(i)# operations on a
  #RootishArrayStack#.  Arrows denote elements being copied. }
  \figlabel{rootisharraystack}
\end{figure}

\codeimport{ods/RootishArrayStack.blocks.n}

\begin{figure}
  \begin{center}
    \includegraphics[scale=0.90909]{figs/gauss}
  \end{center}
  \caption{The number of white squares is $1+2+3+\cdots+#r#$.  The number of
  shaded squares is the same.  Together the white and shaded squares make a
  rectangle consisting of $#r#(#r#+1)$ squares.}
  \figlabel{gauss}
\end{figure}

リストの要素はブロック内で順番に配置される。
0番目の要素はブロック0に、1・2番目の要素はブロック1に、3・4・5番目の要素はブロック2に格納される。
ここで問題になるのは、全体で#i#番目の要素が、どのブロックのどの位置に入っているかをどう知るかである。

#i#に対応する、ブロック内の位置は簡単に計算できる。
インデックス#i#の要素が#b#番目のブロックに入っているなら、$0,\ldots,#b#-1$番目のブロックにおける要素数の合計は$#b#(#b#+1)/2$である。
そのため、#i#は
\[
     #j# = #i# - #b#(#b#+1)/2
\]
として#b#番目のブロックの#j#番目の位置に入っている。
#b#を求めるのはもう少し難しい。
#i#以下のインデックスを持つ要素は$#i#+1$個ある。
一方で、$0,\ldots,b$番目のブロックに入っている要素数の合計は$(#b#+1)(#b#+2)/2$である。
よって、#b#は次の式を満たす最小の整数である。
\[
    (#b#+1)(#b#+2)/2 \ge #i#+1 \enspace .
\]
この式は次のように変形できる。
\[
    #b#^2 + 3#b# - 2#i# \ge  0 \enspace .
\]
対応する2次方程式$#b#^2 + 3#b# - 2#i# =  0$はふたつの解$#b#=(-3 + \sqrt{9+8#i#}) / 2$と$#b#=(-3 - \sqrt{9+8#i#}) / 2$を持つ。
ふたつめの解は常に負の値なので捨ててよい。
よって、解は$#b# = (-3 + \sqrt{9+8i}) / 2$である。
この解は一般に整数とは限らない。
しかし元の不等式に戻ると$#b# \ge (-3 + \sqrt{9+8i}) / 2$を満たす最小の#b#が欲しかったのであった。
これは次のように書ける。
\[
   #b# = \left\lceil(-3 + \sqrt{9+8i}) / 2\right\rceil \enspace .
\]

\codeimport{ods/RootishArrayStack.i2b(i)}

話を変えるが、#get(i)#と#set(i,x)#は簡単に実装できる。
まず#b#を計算し、そのブロック内のインデックス#j#を求め、適切な操作を実行すればよい。

\codeimport{ods/RootishArrayStack.get(i).set(i,x)}

この章のデータ構造のどれを使って#blacks#リストを表現すれば、#get(i)#も#set(i,x)#も定数時間で実行できる。

#add(i,x)#はもう手慣れたものだろう。
まずデータ構造が一杯かどうか、つまり$#r#(#r#+1)/2 = #n#$かどうかを確認する。
もしそうなら#grow()#を呼び出し新たなブロックを追加する。
その後$#i#,\ldots,#n#-1$番目の要素をそれぞれ右にひとつずらし、新たな#i#番目の要素を入れるための隙間を作る。

\codeimport{ods/RootishArrayStack.add(i,x)}

#grow()#メソッドはやってほしいことをしてくれる、つまり新しいブロックを追加してくれる。

\codeimport{ods/RootishArrayStack.grow()}

#grow()#のコストを無視すると、#add(i,x)#の操作はシフト操作のコストを考えれば十分で、これは$O(1+#n#-#i#)$である。
これは#ArrayStack#と同じだ。

#remove(i)#は#add(i,x)#に似ている。
$#i#+1,\ldots,#n#$番目の要素をそれぞれ左にひとつずつシフトし、ふたつ以上の空のブロックがあれば#shrink()#を呼び出し、使われていないブロックをひとつだけ残して削除する。

\codeimport{ods/RootishArrayStack.remove(i)}
\codeimport{ods/RootishArrayStack.shrink()}

ここでもまた、#shrink()#のコストを無視すれば#remove(i)#のコストはシフトのコストを考えれば十分で、これは$O(#n#-#i#)$である。

\subsection{拡張・収縮の分析}

上の#add(i,x)#・#remove(i)#の解析では#grow()#・#shrink()#のことを考慮していなかった。
まず、#ArrayStack.resize()#とは違い、#grow()#は#shrink()#要素をコピーしないことに注意する。
つまり大きさ#r#の配列を割り当て・解放するだけである。
環境によって、これは定数時間で実行できたり、#r#に比例する時間がかかったりする。

#grow()#・#shrink()#を呼んだ直後の状況はわかりやすい。
最後のブロックは空で、その他のブロックは一杯である。
そのため、次の#grow()#・#shrink()#が呼ばれるのは、少なくとも$#r#-1$回要素が追加・削除された後である。
よって、#grow()#・#shrink()#に$O(#r#)$だけ時間がかかっても、そのコストは$#r#-1$回の#add(i,x)#・#remove(i)#で償却され、#grow()#・#shrink()#の償却コストは$O(1)$である。

\subsection{領域使用量}
\seclabel{rootishspaceusage}

次に、#RootishArrayStack#が使用する余分な領域の量を分析する。
RootishArrayStackが使用する領域のうち、リストの要素を保持していないものを数えたい。
これを\emph{無駄な領域}ということにする。
\index{wasted space}%

XXX: 以下、原著怪しい(?)ので確認

#remove(i)#の後には#RootishArrayStack#のうち一杯でないブロックは2つまでである。
よって#n#要素を含む#RootishArrayStack#のブロック数を#r#とすれば次の関係が成り立つ。
\[
    (#r#-2)(#r#-1)/2 \le #n# \enspace
\]
ここでまた二次式の解を考えれば次の式が成り立つ。
\[
   #r# \le (3+\sqrt{1+4#n#})/2 = O(\sqrt{#n#}) \enspace
\]
末尾のブロックふたつの大きさは#r#と#r-1#なので、これらのブロックによって生じる無駄な領域の量は$2#r#-1 = O(\sqrt{#n#})$以下である。
もしこれらのブロックを(例えば)#ArrayStack#に入れれば、#r#個のブロックを入れる#List#による無駄な領域の量も$O(#r#)=O(\sqrt{#n#})$である。
#n#個の要素と関連情報を保持するのに必要なその他の領域は$O(1)$である。
以上より、#RootishArrayStack#の無駄な領域の量は合計$O(\sqrt{#n#})$である。

この空間領域量は空からはじまり、要素をひとつずつ追加できるデータ構造の中で最適であることを示す。
正確にいうと、#n#個の要素を追加する際にはどこかのタイミングで(ほんの一瞬かもしれないが)$\sqrt{#n#}$以上の無駄な領域が生じることを示す。

空のデータ構造に#n#個の要素をひとつずつ追加していくとする。
完了したときには、#r#個のブロックに分散して#n#個のアイテムがデータ構造に入っている。
$#r#\ge \sqrt{#n#}$なら、#r#個のブロックを追跡するために#r#個のポインタ(参照)を使い、ポインタは無駄な領域である。
一方で$#r# < \sqrt{#n#}$なら鳩の巣原理より大きさ$#n#/#r# > \sqrt{#n#}$のブロックが存在する。
このブロックがはじめて割当てられた瞬間を考える。
このブロックは割当てられたとき空なので、$\sqrt{#n#}$の無駄な領域が生じている。
以上より、#n#個の要素を挿入するまでのあるタイミングでデータ構造は$\sqrt{#n#}$の無駄な領域を生じることが示された。

\subsection{要約}

次の定理は#RootishArrayStack#のについての議論をまとめたものだ。
\begin{thm}\thmlabel{rootisharraystack}
  #RootishArrayStack#は#List#インターフェースを実装する。
  #grow()#・#shrink()#のコストを無視すると、#RootishArrayStack#における各操作の実行時間は、
  \begin{itemize}
    \item #get(i)#・#set(i,x)#の実行時間は$O(1)$である。
    \item #add(i,x)#・#remove(i)#の実行時間は$O(1+#n#-#i#)$である。
  \end{itemize}
  空の#RootishArrayStack#から任意の$m$個の#add(i,x)#・#remove(i)#からなる操作の列を実行する。このときすべての#grow()#・#shrink()#にかかる時間の合計は$O(m)$である。

  要素数#n#の#RootishArrayStack#が使う(ワード単位で測った)使用領域量\footnote{\secref{model}で説明した、どのようにメモリ量を測るかという話を思い出してほしい。}は$#n# +O(\sqrt{#n#})$である。
\end{thm}

\subsection{Computing Square Roots}
XXX: Pseudo-code editionでは無い節だが、Ruby editionでは書くか?

\section{ディスカッションと練習問題}

この章で説明したデータ構造は民俗学のようなものだ。
30年以上前の実装さえ見つかる。
例えば、ここで扱った#ArrayStack#・#ArrayQueue#・#ArrayDeque#の実装を簡単に一般化できるスタック・キュー・双方向キューがKnuth \cite[Section~2.2.2]{k97v1}により議論されている。

Brodnik \etal\ \cite{bcdms99}が#RootishArrayStack#についての記述であり、\secref{rootishspaceusage}で述べたような下界$\sqrt{n}$を示したようである。
彼らは他の洗練されたブロックサイズの選び方も示しており、これは#i2b(i)#の中で冪根の計算をせずに済むものだ。
このやり方では#i#番目の要素を含むブロックは$\lfloor\log (#i#+1)\rfloor$番目のもので、これは単に$#i#+1$の二進表現における最高位の桁である。
この計算をするための命令を提供するコンピュータ・アーキテクチャもある。

#RootishArrayStack#に関連するデータ構造として、Goodrich and Kloss \cite{gk99}の二段階の\emph{階層ベクトル}がある。
\index{tiered-vector}%
この構造体は#get(i,x)#・#set(i,x)#を定数時間で実行できる。
#add(i,x)#・#remove(i)#の実行時間は$O(\sqrt{#n#})$である。
これと同じような実行時間は\excref{rootisharraystack-fast}で扱う#RootishArrayStack#のより練られた実装によっても達成できる。

\begin{exc}
  #List#の#addAll(i,c)#操作は#Collection# #c#の要素をすべてリストの#i#番目の位置に順に挿入する。
  (#add(i,x)#は$#c#=\{#x#\}$とした特殊な場合である。)
  この章で説明したデータ構造において
  #addAll(i,c)#を#add(i,x)#繰り返し実行して実装するのはなぜ効率がよくないのかを説明せよ。
  またより効率的な実装を考え、実装せよ。
\end{exc}

\begin{exc}
  \emph{#RandomQueue#}を設計・実装せよ。
  \index{RandomQueue@#RandomQueue#}%
  これは#Queue#インターフェースの実装で、#remove()#操作はそのときキューに入っている要素から一様な確率でひとつ選んで取り出すものである。
  (#RandomQueue#はカバンに要素を入れておき、中を見ずに適当に要素を取り出すようなものだと考えればよい。)

  #RandomQueue#における#add(x)#・#remove()#の償却実行時間は定数でなければならない。
\end{exc}

\begin{exc}
  #Treque# (triple-ended queue)を設計・実装せよ。
  \index{Treque@#Treque#}%
  これは#List#の実装であって、
  #get(i)#・#set(i,x)#は定数時間で実行でき、
  #add(i,x)#・#remove(i)#の実行時間は次のように表せるものだ。
  \[
     O(1+\min\{#i#, #n#-#i#, |#n#/2-#i#|\}) \enspace
  \]
  つまり、両端あるいは中央に近い位置の修正が高速なデータ構造である。
\end{exc}

\begin{exc}
  #rotate(a,r)#操作を実装せよ。
  配列#a#を「回転」する、すなわち$#i#\in\{0,\ldots,#a.length#\}$のすべてについて#a[i]#を$#a#[(#i#+#r#)\bmod #a.length#]$に動かすものだ。
\end{exc}

\begin{exc}
  #List#の回転操作#rotate(r)#を実装せよ。
  これはリストの#i#番目の要素を$(#i#+#r#)\bmod #n#$番目に移す。
  なお#ArrayDeque#や#DualArrayDeque#に対しての#rotate(r)#の実行時間は$O(1+\min\{#r#,#n#-#r#\})$である必要がある。
\end{exc}

\begin{exc}
  #ArrayDeque#を実装せよ。
  ただし、#add(i,x)#・#remove(i)#・#resize()#におけるシフト処理は高速な#System.arraycopy(s,i,d,j,n)#を利用して実現すること。
\end{exc}

\begin{exc}
  #%#演算を用いずに#ArrayDeque#を実装せよ。(この演算に多くの時間がかかる環境があるのだ。)
  #a.length#が2の冪なら次の式が成り立つことを利用してよい。
  \[  #k%a.length#=#k&(a.length-1)# \enspace
  \]
  なお#&#はビット単位のand演算オペレータである。
\end{exc}

\begin{exc}
  剰余演算を使わない#ArrayDeque#の実装を考えよ。
  すべてのデータは配列内の連続した領域に順番に並んでいることを利用してよい。
  データがこの配列の先頭・末尾の外にはみ出たときは、#rebuild()#操作を実行する。
  全ての操作の償却実行時間は#ArrayDeque#と同じになるように注意すること。

  \noindent ヒント:#rebuild()#の実装方法がポイントだ。
  データがどちらの端からもハミ出ない状態に$#n#/2$回以下の操作で辿りつかなければならない。

  実装したプログラムの性能を元の#ArrayDeque#と比較せよ。
  実装を(#System.arraycopy(a,i,b,i,n)#を使って)最適化し、#ArrayDeque#の性能を上回るかどうか確認せよ。
\end{exc}

\begin{exc}
  #RootishArrayStack#を修正し、無駄な領域量は$O(\sqrt{#n#})$だが、#add(i,x)#・#remove(i,x)#の実行時間が$O(1+\min\{#i#,#n#-#i#\})$であるデータ構造を設計・実装せよ。
\end{exc}

\begin{exc}\exclabel{rootisharraystack-fast}
  #RootishArrayStack#を修正し、無駄な領域量は$O(\sqrt{#n#})$だが、#add(i,x)#・#remove(i,x)#の実行時間が$O(1+\min\{\sqrt{#n#},#n#-#i#\})$であるデータ構造を設計・実装せよ。(\secref{selist}が参考になるだろう。)
\end{exc}

\begin{exc}
  #RootishArrayStack#を修正し、無駄な領域量は$O(\sqrt{#n#})$だが、#add(i,x)#・#remove(i,x)#の実行時間が$O(1+\min\{#i#,\sqrt{#n#},#n#-#i#\})$であるデータ構造を設計・実装せよ。(\secref{selist}が参考になるだろう。)
\end{exc}

\begin{exc}
  #CubishArrayStack#を設計・実装せよ。
  \index{CubishArrayStack@#CubishArrayStack#}%
  これは#List#インターフェースを実装する三段階のデータ構造であって、無駄な領域量が$O(#n#^{2/3})$であるものだ。
  #get(i)#・#set(i,x)#は定数時間で実行できる。
  #add(i,x)#・#remove(i)#の償却実行時間は$O(#n#^{1/3})$である。
\end{exc}
