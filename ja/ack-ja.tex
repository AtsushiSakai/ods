\chapter*{この本の読み方(翻訳者まえがき)}
\addcontentsline{toc}{chapter}{この本の読み方(翻訳者まえがき)}

実用上極めて重要で息をするように取り出せるようにしておくべきものと、そうではないものとを明確に区別しておくことは有益だと我々は考えた。以下に列挙するデータ構造(とそれらの上に成り立つアルゴリズム)は学術研究やプログラマーの実務で頻繁に登場し、全ての学習者が深く理解しておくことが望ましいと訳者3人の意見が一致したものだ。
%これらは、この本を教科書として指定した約4ヶ月の講義\footnote{2017年度の東京大学学際科学科総合情報学コースにおける学部生向け講義「情報数理科学2」。正確には、幅優先探索と深さ優先探索については別の授業で扱われる。https://lecture.ecc.u-tokyo.ac.jp/~ktanaka/mis2-2017/index.html}で扱われる事項の部分集合になっている。
% YJ: この情報は必要か?もし必要ならば駒場だけでなく様々な大学のカリキュラムを調査するべきでは。大学の授業に言及することで高校生以下の学生が、自分にはまだ難しい内容だと勘違いしないだろうか。
%% spinute: 読者目線ではIDSが出てくる必要を感じなかったので削りました。

\begin{itemize}
  % \item 第1章: (数学的基礎の確認の章のため、該当なし)
  \item 第2章: ArrayStack, ArrayQueue, ArrayDeque
  \item 第3章: SLList, DLList
  % \item 第4章: (該当なし)
  \item 第5章: ChainedHashTable
  \item 第6章: BinaryTree, BinarySearchTree
  % \item 第7章: (該当なし)
  % \item 第8章: (該当なし)
  \item 第9章: RedBlackTree(複雑なため、\secref{left-redblack}から\secref{redblack-elem-remove}は飛ばしてよい)
  \item 第10章: BinaryHeap
  \item 第11章: MergeSort, QuickSort
  \item 第12章: 幅優先探索, 深さ優先探索
\end{itemize}

ここに並べなかったややマイナーなデータ構造にも別の点で学ぶ価値がある。
それは、新しいデータ構造に用いられるアイデアを知ること、そしてそのデータ構造を解析する道具立てを学ぶことである。
マイナーかは必ずしも重要ではない。どの章を重点的に読むか、興味に応じて適宜調整して頂ければ幸いである。
% YJ: Introduction to Algorithm, Algorithm Designなどに言及する?

\chapter*{翻訳者謝辞}
\addcontentsline{toc}{chapter}{翻訳者謝辞}
% What should we put here?
%% XXX: ソースコードやプロジェクトページの在り処が謝辞に入っているのは変ではないか。まえがきに移す。あと、謝辞はもう少し書く。
クラウドファンディングに参加して頂いた皆様に感謝を捧げたい。
皆様のおかげで本書のソースコードを原著と同様にCreative Commons Attributionライセンスで公開することが出来た。
本書のプロジェクトページ \footnote {\url{https://sites.google.com/view/open-data-structures-ja}} には他言語版やプロジェクトに関する情報がある。
本書のソースコードはGithub \footnote {\url{https://github.com/spinute/ods}}にある。
