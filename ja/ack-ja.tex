\chapter*{翻訳者まえがき}
\addcontentsline{toc}{chapter}{翻訳者まえがき}

実用上極めて重要で息をするように取り出せるようにしておくべきものと、そうではないものとを明確に区別するのは良いことだと我々は考える。以下に挙げたデータ構造(とそれらの上に成り立つアルゴリズム)は学術研究やプログラマーの実務で頻繁に登場し、全ての学習者が知っておくべきだと訳者3人の意見が一致した。以下の表は、この本を教科書として指定した約4ヶ月の講義
\footnote{2017年度の東京大学学際科学科総合情報学コースにおける講義「情報数理科学2」。正確には、幅優先探索と深さ優先探索については別の授業で扱われる。https://lecture.ecc.u-tokyo.ac.jp/~ktanaka/mis2-2017/index.html}
で扱われる事項の部分集合になっている。


\begin{itemize}
  \item 第1章: (数学的基礎の確認の章のため、該当なし)
  \item 第2章: ArrayStack, ArrayQueue, ArrayDeque
  \item 第3章: SLList, DLList
  \item 第4章: (該当なし)
  \item 第5章: ChainedHashTable
  \item 第6章: BinaryTree, BinarySearchTree
  \item 第7章: (該当なし)
  \item 第8章: (該当なし)
  \item 第9章: RedBlackTree
  \item 第10章: BinaryHeap
  \item 第11章: MergeSort, QuickSort
  \item 第12章: 幅優先探索, 深さ優先探索
\end{itemize}

上表にないマイナーなデータ構造は別の点で学ぶ価値がある。
それは、新しいデータ構造に用いられるアイデアを知ること、そしてそのデータ構造を解析する道具立てを学ぶことである。
マイナーかは必ずしも重要ではない。どの章を重点的に読むか、興味に応じて適宜調整して頂ければ幸いである。


\chapter*{翻訳者謝辞}
\addcontentsline{toc}{chapter}{翻訳者謝辞}
% What should we put here?
クラウドファンディングに参加して頂いた皆様に感謝を捧げたい。
皆様のおかげで本書のソースコードを原著と同様にCreative Commons Attributionライセンスで公開することが出来た。
本書のプロジェクトページ \footnote {\url{https://sites.google.com/view/open-data-structures-ja/home}}
本書のソースコードはGithub \footnote {\url{https://github.com/spinute/ods}}にある。
