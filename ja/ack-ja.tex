\chapter*{この本の読み方(翻訳者まえがき)}
\addcontentsline{toc}{chapter}{この本の読み方(翻訳者まえがき)}

本書「みんなのデータ構造」の想定読者は初学者からベテランのエンジニアまで、データ構造に関わる全ての人である。
翻訳者が読者に伝えたいことは次の3つである。

\begin{enumerate}
\item {\bf ソフトウェアのほとんどはシンプルなデータ構造の組み合わせでできている} \\
本書で紹介するデータ構造はシンプルなものである。よくある誤解は、これらのデータ構造は理論上のものであり、実際のソフトウェアはもっと複雑なデータ構造を使っているというものだ。これは全くの間違いである。OSやブラウザなどの複雑なソフトウェアもその実シンプルなデータ構造の組み合わせでできている。本書で紹介するデータ構造が理解できれば、多くのソフトウェアの骨子が理解できるようになるだろう。言い換えれば、本書が紹介するのはおもちゃのデータ構造ではなく、現実のプログラムの中で実際に使われているデータ構造である。% spinute: まえがきなので、高校生をちょっと意識して、実務という表現を避けました

\item {\bf 本書の内容がだいたい分かるようになれば良いエンジニアになれる} \\
ソフトウェアのほとんどが基本的なデータ構造の組み合わせで出来ているということは、基本的なデータ構造を理解すれば、新しいソフトウェアをデザインし、既存のソフトウェアの改良が出来るようになるということである。

\item {\bf 分からない部分は飛ばしても良い} \\
本書には数学の理解を必要とする解析がある。
理解できない部分があったら読み飛ばしても差し支えない。
あるいは詳しい知り合いや翻訳者、著者に質問するべきである。
理解できないことは読み手が原因ではなく書き手が問題だと考えるべきである。
いずれにせよ、分からない箇所があったらそこで立ち止まるのではなく、そのまま先に進めるところまで進んでみることを勧めたい。

\end{enumerate}

2.に関連して、本書で扱われているデータ構造のうち、実用上極めて重要な項目と、そうではない項目とを、明確に区別しておくことが有益だと考えた。
以下に列挙する項目は、本書の中でも特に重要だと訳者3人ともが判断したものだ。
学術研究やプログラマーの実務で頻繁に登場する内容なので、すべての学習者が深く理解しておくことが望ましいだろう。
%これらは、この本を教科書として指定した約4ヶ月の講義\footnote{2017年度の東京大学学際科学科総合情報学コースにおける学部生向け講義「情報数理科学2」。正確には、幅優先探索と深さ優先探索については別の授業で扱われる。https://lecture.ecc.u-tokyo.ac.jp/~ktanaka/mis2-2017/index.html}で扱われる事項の部分集合になっている。
% YJ: この情報は必要か?もし必要ならば駒場だけでなく様々な大学のカリキュラムを調査するべきでは。大学の授業に言及することで高校生以下の学生が、自分にはまだ難しい内容だと勘違いしないだろうか。
%% spinute: 読者目線ではIDSが出てくる必要を感じなかったので削りました。

\begin{itemize}
  % \item 第1章: (数学的基礎の確認の章のため、該当なし)
  \item 第2章: ArrayStack、ArrayQueue、ArrayDeque
  \item 第3章: SLList、DLList
  % \item 第4章: (該当なし)
  \item 第4章: ChainedHashTable
  \item 第6章: BinaryTree、BinarySearchTree
  % \item 第7章: (該当なし)
  % \item 第8章: (該当なし)
  \item 第9章: RedBlackTree(\secref{left-redblack}から\secref{redblack-elem-remove}は複雑なので読み飛ばしてよい)
  \item 第10章: BinaryHeap
  \item 第11章: MergeSort、QuickSort
  \item 第12章: 幅優先探索、深さ優先探索
\end{itemize}

上記に列挙しなかった、ややマイナーなデータ構造にも、別の意味で学ぶ価値はある。
マイナーゆえに直接役立つ機会は少ないかもしれないが、その背後にあるアイデアやその解析手法は多くの場面で役立つはずだ(単純に知的な面白さのある話題も多い)。
どの章を重点的に読むか、興味に応じて適宜調整してほしい。
% YJ: Introduction to Algorithm, Algorithm Designなどに言及する?

本書の日本語版のプロジェクトページ\footnote {\url{https://sites.google.com/view/open-data-structures-ja}}には、本書の他言語版やプロジェクトに関する情報がある。
本書の日本語版のソースコードはGitHub\footnote {\url{https://github.com/spinute/ods}}にある。

\chapter*{なぜ翻訳するのか}
\addcontentsline{toc}{chapter}{なぜ翻訳するのか}
本書はOpen Data Structuresを邦訳したものである。
まず、本を英語から日本語に翻訳する意味はあるのだろうか?

個人的には英語よりも日本語を読む方がかなり早い。
しかし、専門書を読むときはついつい翻訳は避けてしまう。
品質のばらつきが大きく、読みにくいものも多いと感じるのが自分の場合は主な原因だ。

しかし、この教科書は入門書である。
前提知識は中学高校レベルの数学をほんの少しだけである(簡単なプログラミング経験があった方が、実感が持てて、有り難みがわかり、楽しく読めるとは思うが)。
日本語で読める無料の教科書は、分野の裾野を広げ、楽しくプログラムを書ける・効率的なプログラムが書ける人を増やしてくれるだろう。

母国語で大学レベルの教科書が読める国は多くない。
より専門的な内容はもっぱら英語で読むことになるのだから、さっさと崖から突き落とせという意見ももちろんあるだろう。
自分自身も大学に入った頃は「英語で読む」のに抵抗があった(「英語を読む」ところまでは幸い受験で慣れた)。
この抵抗を可及的速やかに取り除くのが、アクセス可能な知識を押し拡げるためには極めて重要だろう。
大学生になっても日本語で教科書が読める恵まれた環境が、日本人の英語アレルギーを支えている可能性はある。

しかし、この教科書は専門への橋渡しの、その初っ端に位置している。
この教科書の前提とする知識は多くない。これに英語を加えるのは、対象読者を制限することになるだろう。
母国語でこのような入門書が読める、少なくともその選択肢があるのは望ましいことだと思う。

この教科書は、300ページ程度ながら、丁寧にゆっくりと、それでいて実用的な題材を納めている。
より本格的な教科書も出版されており、その中には翻訳されているものもある。
内容は素晴らしく、僕自身今でも度々読み返す。
少なくとも自分の持っている二冊は翻訳の質も良かった(Algorithm Design \footnote{Kleinberg, Jon, and Eva Tardos. Algorithm design. Pearson Education, 2006.}とIntroduction to Algorithms \footnote{Cormen, Thomas H., et al. Introduction to algorithms. MIT press, 2009.})。
ただし、文量は大判1000ページ程度、価格は1万円程度と、気軽なものではない。
こういった専門家向けの入門書や、あるいはより専門的な書籍への橋渡しであるこの教科書を、日本語で・フリーでアクセスできるようにするのがこの翻訳の目的である。

堀江 慧(@spinute)

\section*{翻訳者謝辞}
\addcontentsline{toc}{section}{翻訳者謝辞}

まずなにより、本書の原著者であるPat Morinに感謝する。
Patは、Open Data Structuresを立ち上げ、再配布、改変、販売を許容するライセンスで公開してくれた。
本書の日本語訳プロジェクトは、Patが書いた``My hope is that, by doing things this way, this book will continue to be a useful textbook long after my interest in the project, or my pulse, (whichever comes first) has waned.''という一文に惹かれて始めたものである。
Patは、翻訳、クラウドファンディング、出版のいずれの相談にも前向きな返事をくれ、また、そのたびにencourageしてくれた。

本プロジェクトのクラウドファンディングに参加して頂いた皆様にも感謝を捧げたい。
次の団体、企業、個人をはじめとして沢山の方々からご支援いただいた。(敬称略) % 掲載不要とした人も沢山おり、その方々への感謝もしたい
\begin{itemize}
\item 斉藤淳(J PREP)、東京大学瀧本哲史ゼミ、株式会社バオバブ、有限会社コンセントレーション
\item 上田真道、長谷川悠斗、赤野健悟、石田修平、瓜生英尚、斎藤清隆、桑田誠、小林元、石畠正和、t2nis、永浦 尊信、wtokuno、katsyoshi、ysaito、前原貴憲、髙木正弘、Yoshinari Takaoka(a.k.a mumumu)、T.Miyazawa、佐藤怜、田中哲朗、Masashi Fujiwara、雪村あおい、山崎哲也、永本、武平佑太、Yamachan0928、新海息吹、杉山大騎、早瀬元、山口駿人、長谷川拓也、okue、飛田晃介、大坂直人、松林祐、若杉武史、Hideki Hamada、落合哲治、鈴木 研吾、redfield920、山崎宏宇、宮田潔志
\end{itemize}

皆様のおかげで本書のソースコードを原著と同様にCreative Commons Attributionライセンスで公開できた。
また、本書の原稿は出版社でのレビューを経て、かなり読みやすくなった。
さらには、余剰金で情報オリンピックの日本代表選抜に参加する学生に、書籍を献本することも計画している。
この本が多くの人に読まれ、日本のプログラミングや情報科学を支える人たちの一助となることを強く願っている。

ラムダノート社の鹿野さん、高尾さんにもこの場をお借りしてお礼を申し上げたい。
ラムダノートは 2015 年に設立した新進気鋭の技術出版社である。
このプロジェクトのクラウドファンディングを見て、本書の編集作業を破格の条件で引き受けてくれた。
また、出版の企画を持ちかけ、書籍としての完成度を高めるための惜しみない援助をくれた。

% 「書籍」と「本」が紛らわしい
この翻訳プロジェクトは、2017年の春に堀江が個人的に始めたものである。
その時点では、ただ日本語訳を作成することしか考えていなかった。
陣内にはじめて原稿を見てもらったときに「この本は価値がある」と言われたことを受け、より多くの読者に読んでもらうためクラウドファンディングを企画した。
また、陣内は本書のすべての章をレビューし、プロジェクト自体の運営にも携わってくれた。
一目置く陣内からの後押しは、GitHubの隅に眠っていたかもしれない本書の運命を変えた。
堀江、陣内とは違った角度から問題を解決する能力のある田中も、本書のレビューとプロジェクトの運営との両面で協力してくれた。
クラウドファンディングでの成功には人望の厚い田中の協力が欠かせなかった。
また、本書の原稿の読みやすさを高める、多くの修正や訳注を加えてくれた。
大学を卒業した後も、またこの二人と仕事ができて嬉しい。

\noindent\hspace*{2em}
2018年6月

\hfill 堀江 慧(@spinute)
