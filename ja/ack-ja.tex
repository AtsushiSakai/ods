\chapter*{この本の読み方(翻訳者まえがき)}
\addcontentsline{toc}{chapter}{この本の読み方(翻訳者まえがき)}

本書「みんなのデータ構造」の想定読者は初学者からベテランのエンジニアまで、データ構造に関わる全ての人である。
翻訳者が読者に伝えたいことは次の3つである。

\begin{enumerate}
\item {\bf 存在するソフトウェアのほとんどはシンプルなデータ構造の組み合わせでできている} \\
本書で紹介するデータ構造はシンプルなものである。よくある誤解は、これらのデータ構造は理論上のものであり、実際の複雑なソフトウェアはもっと複雑なデータ構造を使っているというものだ。これは全くの間違いである。OSやブラウザなどの複雑なソフトウェアもその実シンプルなデータ構造の組み合わせでできている。本書で紹介するデータ構造が理解できれば多くのソフトウェアの骨子が理解できるようになるだろう。言い換えれば、本書が紹介するのはただの数理モデルではなく実務に役に立つ理論である。

\item {\bf 本書の内容がだいたい分かるようになれば良いエンジニアになれる} \\
ほとんどのソフトウェアが基本的なデータ構造の組み合わせで出来ているということは、基本的なデータ構造が理解できその使い方を覚えれば多くのソフトウェアをデザインし、あるいは既存のソフトウェアの改良が出来るようになるということである。

\item {\bf 分からない部分は飛ばしても良い} \\
本書には数学の理解を必要とする解析がある。
もし理解できない部分があったらその部分は読み飛ばしても差し支えない。
あるいは詳しい知り合いや翻訳者、著者に質問するべきである。
理解できないことがあるならばそれは読み手が原因ではなく書き手が問題だと考えるべきである。
いずれにせよ、分からない箇所があったらそこで立ち止まるのではなく、そのまま先に進めるところまで進んでみることを勧めたい。

\end{enumerate}

2.に関連して、実用上極めて重要で息をするように取り出せるようにしておくべき項目と、そうではない項目とを明確に区別しておくことは有益だと我々は考えた。
以下に列挙する項目は本書の中でも特に重要だと訳者3人ともが判断したものだ。
学術研究やプログラマーの実務で頻繁に登場するので、全ての学習者が深く理解しておくことが望ましいだろう。
%これらは、この本を教科書として指定した約4ヶ月の講義\footnote{2017年度の東京大学学際科学科総合情報学コースにおける学部生向け講義「情報数理科学2」。正確には、幅優先探索と深さ優先探索については別の授業で扱われる。https://lecture.ecc.u-tokyo.ac.jp/~ktanaka/mis2-2017/index.html}で扱われる事項の部分集合になっている。
% YJ: この情報は必要か?もし必要ならば駒場だけでなく様々な大学のカリキュラムを調査するべきでは。大学の授業に言及することで高校生以下の学生が、自分にはまだ難しい内容だと勘違いしないだろうか。
%% spinute: 読者目線ではIDSが出てくる必要を感じなかったので削りました。

\begin{itemize}
  % \item 第1章: (数学的基礎の確認の章のため、該当なし)
  \item 第2章: ArrayStack, ArrayQueue, ArrayDeque
  \item 第3章: SLList, DLList
  % \item 第4章: (該当なし)
  \item 第5章: ChainedHashTable
  \item 第6章: BinaryTree, BinarySearchTree
  % \item 第7章: (該当なし)
  % \item 第8章: (該当なし)
  \item 第9章: RedBlackTree(複雑なため、\secref{left-redblack}から\secref{redblack-elem-remove}は飛ばしてよい)
  \item 第10章: BinaryHeap
  \item 第11章: MergeSort, QuickSort
  \item 第12章: 幅優先探索, 深さ優先探索
\end{itemize}

ここに並べなかったややマイナーなデータ構造にも別の意味で学ぶ価値がある。
マイナーゆえにこれらのデータ構造が直接役立つ機会は少ないかもしれないが、その背後にあるアイデアやその解析手法は別の場面でも役立つはずだ。(あとは、単純に知的な面白さのあるものが多い。)

どの章を重点的に読むか、興味に応じて適宜調整してほしい。
% YJ: Introduction to Algorithm, Algorithm Designなどに言及する?

本書のプロジェクトページ \footnote {\url{https://sites.google.com/view/open-data-structures-ja}} には、本書の他言語版やプロジェクトに関する情報がある。
本書のソースコードは GitHub \footnote {\url{https://github.com/spinute/ods}}にある。

\chapter*{翻訳者謝辞}
\addcontentsline{toc}{chapter}{翻訳者謝辞}

まずなにより、本書の原著者である Pat Morin に感謝する。
Pat は Open Data Structures を立ち上げ、再配布・改変・販売を許容するライセンスで公開してくれた。
Pat が書いた "My hope is that, by doing things this way, this book will continue to be a useful textbook long after my interest in the project, or my pulse, (whichever comes first) has waned." という一文に惹かれ、本書の日本語訳プロジェクトは始まった。
Pat は翻訳・クラウドファンディング・出版のいずれの相談にも前向きな返事をくれ、またその度に encourage してくれた。

本プロジェクトのクラウドファンディングに参加して頂いた皆様にも感謝を捧げたい。

次の団体、企業、個人をはじめとして沢山の方々からご支援いただいた。(敬称略) % 掲載不要とした人も沢山おり、その方々への感謝もしたい
\begin{enumerate}
\item 斉藤淳 (J PREP) 、東京大学瀧本哲史ゼミ、株式会社バオバブ、有限会社コンセントレーション
\item 上田真道、長谷川悠斗、赤野健悟、石田修平、瓜生英尚、Kiyotaka Saito、桑田誠、小林元、石畠正和、t2nis、永浦 尊信、wtokuno、katsyoshi、ysaito、前原貴憲、髙木正弘、Yoshinari Takaoka (a.k.a mumumu)、T.Miyazawa、佐藤怜、田中哲朗、Masashi Fujiwara、雪村あおい、Tetsuya Yamazaki、永本、武平佑太、Yamachan0928、新海息吹、Daiki Sugiyama、早瀬元、山口駿人、長谷川拓也、okue、飛田晃介、大坂直人、松林祐、若杉武史、Hideki Hamada、落合哲治、鈴木 研吾、redfield920、山崎宏宇、宮田潔志
\end{enumerate}

皆様のおかげで本書のソースコードを原著と同様に Creative Commons Attribution ライセンスで公開できた。
また、本書の原稿は出版社でのレビューを経て、かなり読みやすくなった。
さらには、情報オリンピック日本代表選手への本書を献本することもできた。
この本が多くの人に読まれ、日本のプログラミングや情報科学を支える人たちの一助となることを強く願っている。

ラムダノート社の鹿野さん、高尾さんにもこの場をお借りしてお礼を申し上げたい。
ラムダノートは 2015 年に設立した新進気鋭の技術出版社である。
本プロジェクトのクラウドファンディングを見て、本書の編集作業を破格の条件で引き受けてくれた。
また、出版の企画を持ちかけ、書籍としての完成度を高めるための惜しみない援助をくれた。

本プロジェクトは 2017 年の春に個人的にはじめたものだ。
その時点では、ただ日本語訳を作成することしか考えていなかった。
陣内にはじめて原稿を見てもらったとき「この本は価値がある。」と言われたことを受け、より多くの読者に読んでもらうためクラウドファンディングを企画した。
また、陣内は本書のすべての章をレビューし、プロジェクト自体の運営にも携わってくれた。
一目置く陣内からの後押しは、GitHub の隅に眠っていたかもしれない本書の運命を変えた。
堀江、陣内とは違った角度から問題を解決する能力のある田中も、本書のレビューとプロジェクトの運営との両面で協力してくれた。
クラウドファンディングでの成功には人望の厚い田中の協力が欠かせなかった。
また、本書の原稿の読みやすさを高める、多くの修正や訳注を加えてくれた。

堀江 慧(@spinute)
