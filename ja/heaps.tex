\chapter{ヒープ}
\chaplabel{heaps}
この章では優先度付きキューという役に立つデータ構造のふたつの実装を説明する。
いずれも特殊な二分木であり、\emph{ヒープ}(乱雑に積まれたもの)と呼ばれている。
\index{heap}%
\index{binary heap}%
\index{heap!binary}%
これは二分探索木が高度に構造化されながら積み上げられていたのとは対照的である。

ひとつめのヒープの実装は配列を使って完全二分木をシミュレートする。
この効率的な実装はヒープソート(\secref{heapsort}参照)という名の、最速の整列アルゴリズムのうちのひとつの基礎になっている。
ふたつめの実装はより柔軟である。
これはある優先度付きキューの要素を別の優先度付きキューに取り込む#meld(h)#操作を提供する。

\section{#BinaryHeap#:暗黙の二分木}
\seclabel{binaryheap}

\index{BinaryHeap@#BinaryHeap#}%
(優先度付き)キューの最初の実装は400年以上前に発見された手法に基づく。
\emph{Eytzinger法}
\index{Eytzinger's method}%
では木のノードを幅優先順に配列に入れて完全二分木を表現する。
(\secref{bintree:traversal}を参照せよ。)
こうすると、根は0番目、根の左の子は1番目、右の子は2番目、根の左の子の左の子は3番目の位置に格納される。
\figref{eytzinger}を参照せよ。

\begin{figure}
  \begin{center}
    \includegraphics[scale=0.90909]{figs/eytzinger}
  \end{center}
  \caption{Eytzinger's method represents a complete binary tree as an array.}
  \figlabel{eytzinger}
\end{figure}

Eytzinger法を大きな木に適用してみると、規則性が見えてくる。
添え字#i#のノードの左の子の添え字は$#left(i)#=2#i#+1$であり、右の子の添え字は$#right(i)#=2#i#+2$である。
また、添え字#i#のノードの親の添え字は$#parent(i)#=(#i#-1)/2$である。
\codeimport{ods/BinaryHeap.left(i).right(i).parent(i)}

#BinaryHeap#はこの手法を使って完全二分木を暗黙に表現する。
特に、この木の中では要素は\emph{ヒープ順}に並んでいる。
\index{heap-ordered binary tree}%
\index{binary tree!heap-ordered}%
\index{heap order}%
すなわち、添え字#i#の位置に格納されている値は、#parent(i)#に格納されている値以上である。(ただし、$#i#=0$である根は除く。)
このとき、優先度付き#Queue#における最小値が0番目の位置(根)に格納されていることがわかる。

#BinaryHeap#では#n#個の要素が配列#a#に格納されている。
\codeimport{ods/BinaryHeap.a.n}

#add(x)#の実装は簡単である。
他の配列ベースのデータ構造と同じく、まずは#a#が一杯かどうかを確認する。
($#a.length#=#n#$かどうかを確認する。)
もしそうなら#a#を拡張する。
続いて#x#を#a[n]#に入れ、#n#を1増やす。
あとはヒープ性を保てばよい。
これは#x#とその親とを交換する操作を、#x#が親以上になるまで繰り返せばよい。
\figref{heap-insert}を参照せよ。
\codeimport{ods/BinaryHeap.add(x).bubbleUp(i)}

\begin{figure}
  \begin{center}
    \includegraphics[height=\QuarterHeightScaleIfNeeded]{figs/heap-insert-1} \\
    \includegraphics[height=\QuarterHeightScaleIfNeeded]{figs/heap-insert-2} \\
    \includegraphics[height=\QuarterHeightScaleIfNeeded]{figs/heap-insert-3} \\
    \includegraphics[height=\QuarterHeightScaleIfNeeded]{figs/heap-insert-4} \\
  \end{center}
  \caption[Adding to a BinaryHeap]{Adding the value 6 to a #BinaryHeap#.}
  \figlabel{heap-insert}
\end{figure}

#remove()#はヒープから最小の値を削除するが、この実装にはすこし工夫が必要だ。
根が最小値なのはわかっているが、これを削除したあとにもヒープ性が成り立つことを保証しなければならない。

もっとも簡単な方法は根と#a[n-1]#を交換し、交換後に#a[n-1]#にある値を削除し、#n#を1小さくすることだ。
しかし、その結果新たな根はおそらく最小値ではなくなっている。
そのためこれを下方向に動かす必要がある。
このため、この要素を子供と比較することを繰り返す。
もしこの要素が三つ(自分と子供達)のうち最小ならば処理を終了する。
そうでないなら、子供達のうち小さいものと、この要素とを入れ替え、処理を繰り返す。
\codeimport{ods/BinaryHeap.remove().trickleDown(i)}

\begin{figure}
  \begin{center}
    \includegraphics[height=\QuarterHeightScaleIfNeeded]{figs/heap-remove-1} \\
    \includegraphics[height=\QuarterHeightScaleIfNeeded]{figs/heap-remove-2} \\
    \includegraphics[height=\QuarterHeightScaleIfNeeded]{figs/heap-remove-3} \\
    \includegraphics[height=\QuarterHeightScaleIfNeeded]{figs/heap-remove-4} \\
  \end{center}
  \caption[Removing from a BinaryHeap]{Removing the minimum value, 4, from a #BinaryHeap#.}
  \figlabel{heap-remove}
\end{figure}

他の配列ベースのデータ構造と同様に、#resize()#のための時間を無視する。
この実行時間は\lemref{arraystack-amortized}の償却解析によってわかる。
すると、#add(x)#・#remove()#の実行時間は(暗黙の)二分木の高さに依存する。
嬉しいことに、これは\emph{完全}二分木である。
\index{binary tree!complete}%
\index{complete binary tree}%
最大の深さを除く各深さには、可能な最大数のノードがある。
よって、$h$を木の高さとすると、少なくとも$2^h$個のノードがある。
言い換えると、次の式が成り立つ。
\[
  #n# \ge 2^h
\]  
両辺の対数を取ると、次の式が成り立つ。
\[
   h \le \log #n#
\]
以上より、#add(x)#・#remove()#のいずれの実行時間も$O(\log #n#)$である

\subsection{要約}

次の定理は#BinaryHeap#の性能をまとめたものだ。

\begin{thm}\thmlabel{binaryheap}
  #BinaryHeap#は(優先度付き)#Queue#インターフェースの実装である。
  #BinaryHeap#は#add(x)#・#remove()#をサポートし、#resize()#のコストを無視すると、いずれの実行時間も$O(\log #n#)$である。

  空の#BinaryHeap#から任意の$m$個の#add(x)#・#remove()#からなる操作の列を実行する。このときすべての#resize()#にかかる時間の合計は$O(m)$である。
\end{thm}
