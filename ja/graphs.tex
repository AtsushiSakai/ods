\chapter{グラフ}
\chaplabel{graphs}

%\textbf{Warning to the Reader:} This chapter is still being actively
%developed, meaning that the code has not been thoroughly tested and/or
%the text has not be carefully proofread.

この章ではグラフの2つの表現方法を説明し、グラフを扱う基本的なアルゴリズムを紹介する。

% TALK 「組み」のほうが「組」(送り仮名なし)より良い? -- spinute: ただの僕の好みです(強い拘りはないので、揃っていれば良いと思います)
数学における\emph{有向グラフ(directed graph)}とは、
\ejindex{graph}{ぐらふ@グラフ}%
\ejindex{directed graph}{ゆうこうぐらふ@有向グラフ}%
\emph{頂点(vertex)}の集合$V$と、\emph{辺(edge)}の集合$E$からなる、組$G=(V,E)$である。
\ejindex{vertex}{ちょうてん@頂点}%
\ejindex{edge}{へん@辺}%
なお、辺は頂点の組#(i,j)#であり、#i#から#j#に向かっているものとする。
\ejindex{directed edge}{ゆうこうへん@有向辺}%
% XXX: source と target の訳語 % YJ: 一応始点、終点という訳がある
#i#は辺の\emph{始点(source)}と呼ばれ、
\ejindex{source}{始点}
#j#は\emph{終点(target)}と呼ばれる。
\ejindex{target}{終点}
頂点の列$v_0,\ldots,v_k$は、任意の$i\in\{1,\ldots,k\}$について辺$(v_{i-1},v_{i})$が$E$に含まれる場合、
$G$における\emph{経路(path)}と呼ばれる。
\ejindex{path}{けいろ@経路}%
経路$v_0,\ldots,v_k$は、$(v_k,v_0)$も$E$の要素であるとき、\emph{循環(cycle)}と呼ばれる。
\ejindex{cycle}{じゅんかん@循環}%
経路(または循環)に含まれる頂点が互いに異なる場合、その経路(または循環)は\emph{単純(simple)}であるという。
\ejindex{simple}{たんじゅん@単純}%
頂点$v_i$から頂点$v_j$への経路があるとき、
$v_j$は$v_i$から\emph{到達可能(reachable)}であるという。
\ejindex{reachable vertex}{とうたつかのうなちょうてん@到達可能な頂点}
\figref{graph}にグラフの例を示す。

\begin{figure}
  \begin{center}
    \includegraphics[scale=0.90909]{figs/graph}
  \end{center}
  \caption{12個の頂点からなるグラフ。頂点は番号付きの円で、辺はsourceからtargetに向かう矢印で描く}
  \figlabel{graph}
\end{figure}

グラフを使ってモデル化できる現象は多く、そのためグラフには多くの応用がある。
自明な例をいくつか挙げよう。
コンピュータのネットワークは、コンピュータを頂点、それらを繋ぐ(直接の)通信路を辺とみなせば、グラフとしてモデル化できる。
街道は、交差点を頂点、それらを繋ぐ通りを辺とみなせば、グラフとしてモデル化できる。

グラフが集合における二項関係のモデルであることに着目すると、もっと意外な例が見つかる。
例えば、大学の時間割における\emph{衝突グラフ(conflict graph)}というものが考えられる。
\ejindex{conflict graph}{しょうとつぐらふ@衝突グラフ}%
このグラフでは、頂点を大学の講義とする。辺#(i,j)#は、講義#i#と講義#j#の両方を受講する生徒がいることを表している。
よって、そのような辺があれば、講義#i#と講義#j#のテストを同じ時間に実施できないことがわかる。

この節を通じて、#n#は頂点の数、#m#は辺の数を表すことにする。
すなわち、$#n#=|V|$かつ$#m#=|E|$である。
さらに、$V=\{0,\ldots,#n#-1\}$と仮定する。
$V$の各頂点とひも付けられたデータを保存するには、大きさ#n#の配列にデータを入れておけばよい。

グラフに対する典型的な操作には次のようなものがある。
\begin{itemize}
  \item #addEdge(i,j)#:辺$(#i#,#j#)$を$E$に加える
  \item #removeEdge(i,j)#:辺$(#i#,#j#)$を$E$から除く
  \item #hasEdge(i,j)#:$(#i#,#j#)\in E$かどうかを調べる
  \item #outEdges(i)#:$(#i#,#j#)\in E$を満たす整数整数$#j#$のリストを返す
  \item #inEdges(i)#:$(#j#,#i#)\in E$を満たす整数整数$#j#$のリストを返す
\end{itemize}

これらの操作を効率的に実装するのはさほど難しくない。
例えば、はじめの3つの操作は#USet#を使って実装できる。\chapref{hashing}で説明したハッシュテーブルを使えば期待実行時間は定数である。
最後の2つの操作は、各頂点ごとに隣接する頂点のリストを保持すれば、定数時間で実行できる。

しかし、グラフを何に応用するかによって、各操作への要求は異なる。理想的には、そうした要求をすべて満たす中で最も単純な実装を使いたい。
そのため、この章ではグラフを表現する方法を大きく2つに分けて議論する。

\section{#AdjacencyMatrix#:行列によるグラフの表現}
\seclabel{adjacency-matrix}

\ejindex{adjacency matrix}{りんせつぎょうれつ@隣接行列}%
#n#個の頂点を持つグラフ$G=(V,E)$を、真偽値を並べた$#n#\times#n#$行列#a#を使って表現したものを、\emph{隣接行列(adjacency matrix)}と呼ぶ。
\codeimport{ods/AdjacencyMatrix.a.n.AdjacencyMatrix(n0)}

隣接行列の要素#a[i][j]#は次のように定義される。
\[  #a[i][j]#=
    \begin{cases}
      #true# & \text{$#(i,j)#\in E$のとき} \\
      #false# & \text{そうでないとき}
    \end{cases}
\]
\figref{graph}のグラフの隣接行列を\figref{graph-adj}に示す。

隣接行列による表現において、#addEdge(i,j)#、#removeEdge(i,j)#、#hasEdge(i,j)#の各操作は、いずれも行列の要素#a[i][j]#を読み書きするだけで実装できる。
\codeimport{ods/AdjacencyMatrix.addEdge(i,j).removeEdge(i,j).hasEdge(i,j)}
いずれの操作も明らかに定数時間で実行できる。

\begin{figure}
  \begin{center}
    \includegraphics[scale=0.90909]{figs/graph} \\[3ex]
    \begin{tabular}{c|cccccccccccc}
        &0&1&2&3&4&5&6&7&8&9&10&11 \\\hline
       0&0&1&0&0&1&0&0&0&0&0&0 &0\\
       1&1&0&1&0&0&1&1&0&0&0&0 &0\\
       2&1&0&0&1&0&0&1&0&0&0&0 &0\\
       3&0&0&1&0&0&0&0&1&0&0&0 &0\\
       4&1&0&0&0&0&1&0&0&1&0&0 &0\\
       5&0&1&1&0&1&0&1&0&0&1&0 &0\\
       6&0&0&1&0&0&1&0&1&0&0&1 &0\\
       7&0&0&0&1&0&0&1&0&0&0&0 &1\\
       8&0&0&0&0&1&0&0&0&0&1&0 &0\\
       9&0&0&0&0&0&1&0&0&1&0&1 &0\\
      10&0&0&0&0&0&0&1&0&0&1&0 &1\\
      11&0&0&0&0&0&0&0&1&0&0&1 &0\\
    \end{tabular}
  \end{center}
  \caption{グラフとその隣接行列}
  \figlabel{graph-adj}
\end{figure}

隣接行列による表現で効率がよくない操作は、#outEdges(i)#と#inEdges(i)#である。
これらを実装するには、#a#の対応する行または列にある#n#個の要素を順にすべて見て、各添字#j#についてそれぞれ#a[i][j]#と#a[j][i]#が真かどうかを確認しなければならない。
\javaimport{ods/AdjacencyMatrix.outEdges(i).inEdges(i)}
\cppimport{ods/AdjacencyMatrix.outEdges(i,edges).inEdges(i,edges)}
これらの操作には明らかに$O(#n#)$の時間がかかる。

隣接行列による表現のもう1つの短所は、行列がかさばることである。%、行列で表現したために、かさばることである。
真偽値からなる$#n#\times #n#$行列を格納するのに、$#n#^2$ビット以上のメモリが必要になる。
この実装では\javaonly{#boolean#}\cpponly{#bool#}という値を使っているので、実際に必要なメモリは$#n#^2$バイトのオーダーになる。
実装を工夫して#w#個の真偽値を各ワードに詰め込めば、領域使用量を$O(#n#^2/#w#)$ワードに減らせるだろう。

\begin{thm}
#AdjacencyMatrix#は#Graph#インターフェースを実装する。
#AdjacencyMatrix#は以下の各操作をサポートする。
\begin{itemize}
  \item #addEdge(i,j)#、#removeEdge(i,j)#、#hasEdge(i,j)#を定数時間で実行できる。
  \item #inEdges(i)#、#outEdges(i)#を$O(#n#)$の時間で実行できる。
\end{itemize}
#AdjacencyMatrix#の領域使用量は$O(#n#^2)$である。% 空間使用量、領域使用量
\end{thm}

メモリ使用量の多さと、#inEdges(i)#および#outEdges(i)#の性能の低さにもかかわらず、#AdjacencyMatrix#が有用な場合はある。
具体的には、グラフ$G$が\emph{密}なとき、つまり辺の数が$#n#^2$に近い場合には、$#n#^2$というメモリ使用量を許容できるだろう。

#AdjacencyMatrix#が広く使われる別の理由として、グラフ#G#の性質を効率的に計算するために行列#a#の代数的な操作を使えるという点が挙げられる。
これはアルゴリズムに関する話題だが、そのような性質を1つ紹介しよう。
#a#の要素を整数(#true#が1、#false#が0)であるとみなし、#a#同士の積を行列の掛け算を使って計算すると、行列$#a#^2$が求まる。
積の定義から成り立つ次の関係を思い出してほしい。
\[
    #a^2[i][j]# = \sum_{k=0}^{#n#-1} #a[i][k]#\cdot #a[k][j]#
\]
グラフ$G$の文脈で解釈すると、この和は$G$が辺#(i,k)#と辺#(k,j)#を共に持つ頂点$#k#$の個数になる。
つまり、この和は$#i#$から$#j#$への(中間頂点$#k#$を通る)経路のうち、長さがちょうど2であるものの個数である。
この観察に基づいて、$G$におけるすべての頂点の対についての最短経路を$O(\log #n#)$回だけの行列の積で計算するアルゴリズムが考案されている。

\section{#AdjacencyLists#:リストの集まりとしてのグラフ}
\seclabel{adjacency-list}

\ejindex{adjacency list}{りんせつりすと@隣接リスト}%
\emph{隣接リスト(adjacency list)}は、グラフの表現において辺を重視するアプローチである。
隣接リストの実装にはさまざまな方法がある。
この節では、その中でも単純な実装について説明し、
それ以外の方法については末尾で紹介する。
隣接リストによる表現では、グラフ$G=(V,E)$を、リストの配列#adj#で表現する。
リスト#adj[i]#には、頂点#i#と、隣接するすべての頂点が含まれる。
つまり、#adj[i]#は、$#(i,j)#\in E$を満たすすべての添字#j#からなるリストである。
\codeimport{ods/AdjacencyLists.adj.n.AdjacencyLists(n0)}
\figref{graph-adjlist}に例を示す。
この実装では、リスト#adj#は#ArrayStack#\cpponly{のサブクラス}とする。
なぜなら、添字を使って定数時間で要素にアクセスしたいからである。
他の選択肢もありうる。
特に、#adj#を#DLList#として実装してもよいだろう。

\begin{figure}
  \begin{center}
    \includegraphics[scale=0.90909]{figs/graph} \\[3ex]
    \begin{tabular}{|c|c|c|c|c|c|c|c|c|c|c|c|c|}\hline
        0&1&2&3&4&5&6 &7 &8&9 &10&11 \\\hline
        1&0&1&2&0&1&5 &6 &4&8 &9 &10 \\
        4&2&3&7&5&2&2 &3 &9&5 &6 &7 \\
         &6&6& &8&6&7 &11& &10&11& \\
         &5& & & &9&10&  & &  &  & \\
         & & & & &4&  &  & &  &  & \\
    \end{tabular}
  \end{center}
  \caption{グラフとその隣接リスト}
  \figlabel{graph-adjlist}
\end{figure}

#addEdge(i,j)#は、リスト#adj[i]#に#j#を加えるだけだ。これは定数時間で実行できる。
\codeimport{ods/AdjacencyLists.addEdge(i,j)}

#removeEdge(i,j)#では、リスト#adj[i]#から#j#を見つけ、それを削除する。
これには$O(\deg(#i#))$の時間がかかる。
ここで、$\deg(#i#)$は、$E$の要素のうち$#i#$から出ている辺の個数であり、$#i#$の\emph{次数(degree)}と呼ばれる
\ejindex{degree}{じすう@次数}%
\codeimport{ods/AdjacencyLists.removeEdge(i,j)}

#hasEdge(i,j)#も同様だ。
リスト#adj[i]#から#j#を探し、見つかれば真を、見つからなければ偽を返す。
これにかかる時間は$O(\deg(#i#))$である。
\codeimport{ods/AdjacencyLists.hasEdge(i,j)}

#outEdges(i)#は、単純に
\pcodeonly{リスト#adj[i]#を返す。}
\javaonly{リスト#adj[i]#を返す。}
\cpponly{リスト#adj[i]#の中身を出力リストにコピーする。}
\javaonly{これは定数時間で実行できる。}
\cpponly{これにかかる時間は$O(\deg(#i#))$である。}
\pcodeimport{ods/AdjacencyLists.outEdges(i)}
\javaimport{ods/AdjacencyLists.outEdges(i)}
\cppimport{ods/AdjacencyLists.outEdges(i,edges)}

#inEdges(i)#ではもう少し手順が増える。
すべての頂点$j$について#(i,j)#が存在するかどうかを確認し、もし存在したら#j#を出力リストに追加する。
この操作にはかなり時間が必要で、
すべての頂点の隣接リストを見て回る必要があるので、$O(#n# + #m#)$の時間がかかる。
\pcodeimport{ods/AdjacencyLists.inEdges(i)}
\javaimport{ods/AdjacencyLists.inEdges(i)}
\cppimport{ods/AdjacencyLists.inEdges(i,edges)}

このデータ構造の性能を次の定理にまとめる。

\begin{thm}
#AdjacencyLists#は#Graph#インターフェースを実装する。
#AdjacencyLists#は以下のように各操作をサポートする。
\begin{itemize}
  \item #addEdge(i,j)#は定数時間で実行できる。
  \item #removeEdge(i,j)#および#hasEdge(i,j)#にかかる時間は$O(\deg(#i#))$である。
  \javaonly{\item #outEdges(i)#は定数時間で実行できる。}
  \cpponly{\item #outEdges(i)#にかかる時間は$O(\deg(#i#))$である。}
  \item #inEdges(i)#にかかる時間は$O(#n#+#m#)$である。
\end{itemize}
#AdjacencyLists#の領域使用量は$O(#n#+#m#)$である。
\end{thm}

すでに触れたように、グラフを隣接リストとして実装する具体的な方法には多くの選択肢がある。
それらの中から実装方法を選ぶ際に考慮する点としては次のようなものがある。
\begin{itemize}
  \item #adj#の要素を格納するのに、どんなデータ構造を使うか。
  配列ベースか、ポインタベースか、あるいはハッシュテーブルか
  \item 各#i#について$#(j,i)#\in E$を満たす頂点#j#のリストを#inadj#とした場合、
  これを二次的な隣接リストとして利用するかどうか。
  二次的な隣接リストを利用することで、#inEdges(i)#の実行時間が劇的に改善するが、辺を追加、削除する際の仕事が少し増える
  \item #adj[i]#における辺#(i,j)#に、対応する#inadj[j]#のエントリへの参照を持たせるべきか
  \item 辺をオブジェクトとして明示的に実装し、関連データを持たせるべきか。
  その場合は、#adj#に、頂点のリストではなく辺のリストを持たせることになる
\end{itemize}
これらの選択肢の多くは、実装の時間的および領域的な複雑さと実装の性能とのトレードオフに帰結する。

\section{グラフの走査}

この節では、グラフの頂点#i#から開始して、#i#から到達可能なすべての頂点を探索するアルゴリズムを2つ紹介する。
2つとも、隣接リストで表現されたグラフに最適なアルゴリズムである。
そのため、この節の解析では、グラフの表現が#AdjacencyLists#であることを仮定する。

\subsection{幅優先探索}
\seclabel{general-bfs}

\ejindex{breadth-first-search}{はばゆうせんたんさく@幅優先探索}%
\emph{幅優先探索(breadth-first search)}では、頂点#i#から開始し、#i#に隣接する頂点、#i#の隣の隣、#i#の隣の隣の隣、という順番で訪問していく。

このアルゴリズムは、二分木における幅優先の走査アルゴリズム(\secref{bintree:traversal})の一般化であり、とてもよく似ている。
木の幅優先走査では、初期状態で#i#だけを含めたキュー#q#を使う。
それから、#q#の要素を取り出してその要素に隣接する要素を#q#に追加していく。
その際には、要素に隣接する要素について、いずれも過去に#q#に登場していないことを前提としていた。
グラフにおける幅優先探索アルゴリズムでは、木の走査の場合と違い、同じ頂点を#q#に2回以上追加しないように気をつける必要がある。
そのためには、真偽値の補助配列#seen#を使い、すでに見つかっている頂点を記録しておけばよい。
\codeimport{ods/Algorithms.bfs(g,r)}
\figref{graph}に対して#bfs(g,0)#を実行する様子の一例を\figref{graph-bfs}に示す。
この処理の順番は隣接リストの並び順によって異なる。
\figref{graph-bfs}の処理では\figref{graph-adjlist}の隣接リストを使った。

\begin{figure}
  \begin{center}
    \includegraphics[scale=0.90909]{figs/graph-bfs}
  \end{center}
  \caption{ノード0から始まる幅優先探索の例。ノードの数字はこの探索において#q#に追加される順番を表している。ノードが#q#に追加されるときに辿られる辺を黒、それ以外の辺を灰色で描いている}
  \figlabel{graph-bfs}
\end{figure}

#bfs(g,i)#の実行時間は簡単に解析できる。
#seen#のおかげで、同じ頂点が#q#に2回以上追加されることはない。
#q#に頂点を追加する(そして後で削除する)処理は定数時間で実行でき、合計$O(#n#)$だけの時間がかかる。
すべての頂点が内側のループで高々1回処理されるので、すべての隣接リストが高々1回処理される。
よって、$G$の辺は高々1回だけ処理される。
内部ループが一周するたびに辺が1つ処理され、一周は定数時間で実行できるので、合計$O(#m#)$だけの時間がかかる。
以上より、アルゴリズム全体の実行時間は$O(#n#+#m#)$である。

次の定理に#bfs(g,r)#の性能をまとめる。
\begin{thm}\thmlabel{bfs-graph}
#bfs(g,r)#の実行時間は、#AdjacencyLists#で実装された#Graph# #g#を入力とすると、$O(#n#+#m#)$である。
\end{thm}

幅優先の走査には特別な性質がある。
#bfs(g,r)#を呼ぶと、#r#からの有向経路が存在するすべての頂点が、最終的には#q#に追加(および後で削除)される。
また、#r#から距離0の頂点(#r#自身)は、#r#から距離1の頂点より先に#q#に追加され、距離1の頂点は距離2の頂点よりも先に#q#に追加される。以降も同様である。
そのため、#bfs(g,r)#では、#r#からの距離の昇順で頂点を訪問していく。そして、#r#から到達不可能な頂点を訪問することはない。

このような方法で走査を行う幅優先探索の便利な応用として、最短経路の計算がある。
#r#からすべての頂点への最短経路を求めるために、長さ#n#の補助配列#p#を利用する#bfs(g,r)#の変種が利用できるのである。
この変種では、頂点#j#を#q#に追加するとき、#p[j]=i#とする。
こうすると、#p[j]#が、#r#から#j#への最短経路における最後から2つめの頂点になる。
#p[p[j]#、#p[p[p[j]]]#、と繰り返し求めていくことで、#r#から#j#への最短経路を(逆順に)再構築できる。

\subsection{深さ優先探索}

グラフに対する\emph{深さ優先探索(depth-first search)}は、二分木を走査するときの標準的なアルゴリズムに似ている。
\ejindex{depth-first-search}{ふかさゆうせんたんさく@深さ優先探索}%
すなわち、ある部分木を完全に探索し終えてから根の方向に戻り、それから別の部分木の探索に進む。
深さ優先探索は、スタックの代わりにキューを使う幅優先探索であると考えてもよい。

各頂点#i#には、深さ優先探索の実行中、色#c[i]#を割り当てる。
未訪問の頂点の色を#white#、現在訪問中の頂点の色を#grey#、すでに訪問した頂点の色を#black#にする。
深さ優先探索の最も簡単な方法は、再帰的なアルゴリズムによるものである。
まず、#r#を訪問するところから処理を開始する。
頂点#i#を訪問するときには#i#の色を#grey#にする。
それから#i#の隣接リストを見て、その中の#white#な頂点を再帰的に訪問する。
#i#に対する処理が終わったら、最後に#i#の色を#black#にする。
\codeimport{ods/Algorithms.dfs(g,r).dfs(g,i,c)}
\figref{graph-dfs}にこのアルゴリズムの処理の例を示す。

\begin{figure}
  \begin{center}
    \includegraphics[scale=0.90909]{figs/graph-dfs}
  \end{center}
  \caption{ノード0から深さ優先探索を開始した例。各ノードに併記してある数字は、この探索において#q#に追加される順番。再帰的呼び出しになる辺を黒、それ以外の辺を灰色で描いている}
  \figlabel{graph-dfs}
\end{figure}

再帰は、深さ優先探索について考えるには最適だが、実装方法としては最善でない。
% XXX: stack overflow の訳語
上のコードは、スタックのオーバーフローが原因で、大きなグラフの探索に失敗してしまうことがある。% YJ: 訳はカタカナでスタックオーバーフローが多いようである
そこで、再帰のスタックを明示的なスタック#s#で置き換える実装が考えられる。
次の実装はこの方法を採用したものである。
% XXX: black にしなくていいのか? % caprice から spinute しないといけないはず % TODO: YJ: アルゴリズムとしては、greyとblackは分けて考える必要がない。分岐もwhiteか否かだけをチェックすれば良い(そうしてある)。なのでアルゴリズムに不備はない (dfsもdfs2も実装にあたってblackは必要ではない)。grey, blackを分けるメリットは現在ノードに至るための経路がgreyで表示されるという点であるが活用されていないようである。ただ説明文は展開済みノードをblackにすると書いてあるので、説明として良くないと思う。Patに確認するか、こちらで修正する必要があるだろう。
\codeimport{ods/Algorithms.dfs2(g,r)}
上のコードでは、次の頂点#i#が処理されるときに#i#の色を#grey#にし、#i#の隣接リストに入っていた頂点をスタックに積んでから、そのうちの1つを#i#にしている。

当然のことだが、#dfs(g,r)#および#dfs2(g,r)#の実行時間は#bfs(g,r)#と同じである。
\begin{thm}\thmlabel{dfs-graph}
#AdjacencyLists#で実装された#Graph# #g#を入力すると、#dfs(g,r)#および#dfs2(g,r)#の実行時間はいずれも$O(#n#+#m#)$である。
\end{thm}

幅優先探索と同様に、深さ優先探索にも、実行に対応する木が考えられる。
頂点$#i#\neq #r#$の色が#white#から#grey#になるのは、ある頂点#i'#を再帰的に処理する中で#dfs(g,i,c)#を呼び出したからである
(#dfs2(g,r)#の場合、#i#は#i'#をスタックで置き換えた頂点のうちの1つである)。
#i'#を#i#の親だと考えると、#r#を根とする木が得られる。
この木は、\figref{graph-dfs}だと、頂点0から頂点11への経路に相当する。

深さ優先探索には、次のような重要な性質がある。
すなわち、#i#の色が#grey#のときに#i#から他の頂点#j#への経路で白い頂点だけを辿るものがある場合、
#i#の色が#black#になるよりも前に、#j#の色が#grey#になってから#black#になる
(これは背理法で証明できる。#i#から#j#への任意の経路$P$を考えればよい)。

この性質は、例えば循環の検出に役立つ。
\ejindex{cycle detection}{じゅんかんけんしゅつ@循環検出}%
\figref{dfs-cycle}で、
#r#から到達可能なある循環$C$があるとする。
$C$の中で色が#grey#である最初の頂点を#i#とし、$C$において#i#の前にある頂点を#j#とする。
このとき、上の性質から#j#の色は#grey#になり、辺#(j,i)#を辿るときにも#i#の色はまだ#grey#である。
したがって、深さ優先探索木において#i#から#j#への経路$P$が存在すること、および辺#(j,i)#が存在することが、アルゴリズムによりわかる。
つまり、$P$が循環であるとわかる。

\begin{figure}
  \begin{center}
    \includegraphics[scale=0.90909]{figs/dfs-cycle}
  \end{center}
  \caption{深さ優先探索アルゴリズムにより、グラフ$G$の循環を検出できる。ノード#i#が灰色であるとき、ノード#j#も灰色である。そのため、深さ優先探索木において#i#から#j#への経路$P$が存在する。辺#(j,i)#が存在することから、$P$が循環であることもわかる}
  \figlabel{dfs-cycle}
\end{figure}

\section{ディスカッションと練習問題}

\ref{thm:bfs-graph}と\ref{thm:dfs-graph}から導かれる幅優先探索と深さ優先探索のアルゴリズムの実行時間は、ある意味で厳しすぎるものだといえる。
$G$における頂点#i#のうち、#i#から#r#への経路が存在するものの個数を、$#n#_{#r#}$と定義する。
また、そのような頂点から出る辺の本数を$#m#_#r#$とする。
このとき、幅優先探索と深さ優先探索のより正確な実行時間に関して、次の定理が成り立つ
(練習問題で扱うアルゴリズムのうちの一部でこの定理が役に立つ)。
\begin{thm}\thmlabel{graph-traversal}
#bfs(g,r)#、#dfs(g,r)#、#dfs2(g,r)#の実行時間は、#AdjacencyLists#で実装された#Graph# #g#を入力とすると、いずれも$O(#n#_{#r#}+#m#_{#r#})$である。
\end{thm}

幅優先探索は、MooreとLeeによって独立に考案されたようである\cite{m59, l61}。
前者では迷路の探索において、後者では回路における経路に関する文脈において発見された。

グラフの隣接リストによる表現は、それまで一般的であった隣接行列による表現の代替として、HopcroftとTarjanにより提案された\cite{ht73}。
隣接リストによる表現は、深さ優先探索のほか、Hopcroft-Tarjanの平面性テストのアルゴリズムにおいても重要な役割を果たす。
\ejindex{planarity testing}{へいめんせいてすと@平面性テスト}%
これは、辺が交差しないようにグラフを平面に描けるかどうかを$O(#n#)$の時間で調べるアルゴリズムである\cite{ht74}。

以下の練習問題において、無向グラフとは、辺$(#i#,#j#)$が存在するとき、またそのときに限り、辺$(#j#,#i#)$が存在するようなグラフであるとする。
\ejindex{undirected graph}{むこうぐらふ@無向グラフ}%
\ejindex{graph!undirected}{ぐらふ@グラフ!むこう@無向}%

\begin{exc}
\figref{graph-example2}のグラフの隣接リストによる表現および隣接行列による表現を書け。
\end{exc}

\begin{figure}
  \centering{\includegraphics[scale=0.90909]{figs/graph-example2}}
  \caption{例題用のグラフ}
  \figlabel{graph-example2}
\end{figure}

\begin{exc}
  \ejindex{incidence matrix}{せつぞくぎょうれつ@接続行列}%
  グラフ$G$の\emph{接続行列(incidence matrix)}とは、$#n#\times#m#$行列$A$であって、次のように定義されるものである。
  \[
     A_{i,j} = \begin{cases}
        -1 & \text{頂点$i$が辺$j$の始点であるとき} \\
        +1 & \text{頂点$i$が辺$j$の終点であるとき} \\
        0 & \text{それ以外のとき}
     \end{cases}
  \]
  \begin{enumerate}
    \item \figref{graph-example2}のグラフの接続行列を書け
    \item グラフの隣接行列による表現を設計、実装せよ。また、領域使用量を解析し、#addEdge(i,j)#、#removeEdge(i,j)#、#hasEdge(i,j)#、#inEdges(i)#、#outEdges(i)#の実行時間を求めよ。
  \end{enumerate}
\end{exc}

\begin{exc}
\figref{graph-example2}のグラフ$G$について、#bfs(G,0)#および#dfs(G,0)#を実行する様子を図示せよ。
\end{exc}

\begin{exc}
  \ejindex{connected graph}{れんけつぐらふ@連結グラフ}%
  \ejindex{graph!connected}{ぐらふ@グラフ!れんけつ@連結}%
  $G$を無向グラフとする。
  $G$が\emph{連結(connected)}であるとは、任意の相異なる頂点の組$(#i#, #j#)$について、$#i#$から$#j#$への辺があることをいう(なお、このとき$G$は無向グラフなので、#j#から#i#にも辺がある)。
  $G$が連結かどうかを$O(#n#+#m#)$の時間で確認する方法を示せ。
\end{exc}

\begin{exc}
  \ejindex{connected components}{れんけつせいぶん@連結成分}%
  $G$を無向グラフとする。
  $G$の\emph{連結成分ラベル付け(connected components labelling)}とは、それぞれが連結な部分グラフになるような$G$の分割方法のうち、分割後の集合が極大となるように$G$を分割することである。
  これを$O(#n#+#m#)$の時間で計算する方法を示せ。
\end{exc}

\begin{exc}
  \ejindex{spanning forest}{ぜんいきもり@全域森}%
  $G$を無向グラフとする。
  $G$の\emph{全域森(spanning forest)}は木の集まりであって、各木の辺が$G$の辺であり、$G$のすべての頂点をある木に含むようなものである。
  これを$O(#n#+#m#)$の時間で計算する方法を示せ。
\end{exc}

\begin{exc}
  \ejindex{strongly-connected graph}{きょうれんけつぐらふ@強連結グラフ}%
  \ejindex{graph!strongly-connected}{ぐらふ@グラフ!きょうれんけつ@強連結}%
  グラフ$G$が\emph{強連結(strongly-connected)}であるとは、$G$の任意の頂点の組$(#i#, #j#)$について、$#i#$から$#j#$への経路が存在することをいう。
  これを$O(#n#+#m#)$の時間で確認する方法を示せ。
\end{exc}

\begin{exc}
  グラフ$G=(V,E)$と、特別な頂点$#r#\in V$があるとき、$#r#$から全頂点$#i# \in V$への最短経路の長さを計算する方法を示せ。
\end{exc}

\begin{exc}
#dfs(g,r)#が#dfs2(g,r)#と異なる順番で頂点を訪問する単純な例を与えよ。
また、#dfs(g,r)#と常に同じ順番で頂点を訪問する#dfs2(g,r)#を実装せよ
(ヒント:#r#から2つ以上の辺が出ているグラフをいくつか作り、それぞれのアルゴリズムがどう動くか考えてみるといいだろう)。
\end{exc}

\begin{exc}
  \index{universal sink}%
  \ejindex{celebrity|see{universal sink}}{せれぶりてぃ@セレブリティ|see{universal sink}}%
  グラフ$G$の\emph{universal sink}とは、$#n#-1$個の辺の行き先になっており、かつそこから辺が出ていない頂点である
  \footnote{universal sink #v#を\emph{セレブリティ(celebrity)}と呼ぶこともある。部屋の中の全員が#v#のことを知っているが、#v#は部屋の中の人が誰だかまったく知らないため、有名人(セレブリティ)のような状態になっているからである。}
  #AdjacencyMatrix#で表現されるグラフ$G$がuniversal sinkを持つかどうかを判定するアルゴリズムを設計、実装せよ。
  ただし、実行時間は$O(#n#)$でなければならない。
\end{exc}
