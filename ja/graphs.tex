\chapter{グラフ}
\chaplabel{graphs}

%\textbf{Warning to the Reader:} This chapter is still being actively
%developed, meaning that the code has not been thoroughly tested and/or
%the text has not be carefully proofread.

この章ではグラフのふたつの表現方法を説明し、それらを使う基本的なアルゴリズムを紹介する。

数学的には、\emph{(有向)グラフ}とは
\index{graph}%
\index{directed graph}%
組み$G=(V,E)$である。
ここで$V$は\emph{頂点}の集合であり、
\index{vertex}%
$E$は\emph{辺}と呼ばれる頂点の組みの集合である。
\index{edge}%
辺#(i,j)#は#i#から#j#に向いている。
\index{directed edge}%
XXX: source と target の訳語
#i#は辺の\emph{source}と呼ばれ、
\index{source}
#j#は\emph{target}と呼ばれる。
\index{target}
$G$における\emph{経路}%
\index{path}
とは頂点の列$v_0,\ldots,v_k$であって 任意の$i\in\{1,\ldots,k\}$について辺$(v_{i-1},v_{i})$が$E$に含まれるものである。
経路$v_0,\ldots,v_k$が
\emph{循環}である(循環している)とは、
\index{cycle}%
$(v_k,v_0)$も$E$の要素であることをいう。
経路(または循環)が\emph{単純}であるとは、
\index{simple path/cycle}%
経路に含まれる頂点が互いに異なることをいう。
頂点$v_i$から頂点$v_j$への経路があるとき、
$v_j$は$v_i$から\emph{到達可能}であるという。
\index{reachable vertex}
\figref{graph}にグラフの例を示した。

\begin{figure}
  \begin{center}
    \includegraphics[scale=0.90909]{figs/graph}
  \end{center}
  \caption{A graph with twelve vertices.  Vertices are drawn as numbered
    circles and edges are drawn as pointed curves pointing from source
    to target.}
  \figlabel{graph}
\end{figure}

グラフは多くの現象をモデル化できるので多くの応用を持つ。
自明な例がいくつかある。
コンピュータのネットワークはコンピュータを頂点、それらを繋ぐ(直接の)通信路を辺と見なせばグラフとしてモデル化できる。
街道は交差点を頂点、それらを繋ぐ通りを辺と見なせばグラフとしてモデル化できる。

もうすこし巧みな例は、グラフが集合における二項関係のモデルであることに着目すると見つかる。
例えば大学の時間割りにおける\emph{衝突グラフ}を考えられる。
\index{conflict graph}%
ここで頂点は大学の講義で、辺#(i,j)#は#i#と#j#の両方を受講する生徒がいることを表している。
よってこの辺から講義#i#・#j#のテストは同じ時間に割当てられてはならないことがわかる。

この節を通じて#n#は頂点の数を、#m#は辺の数を表すことにする。
すなわち$#n#=|V|$かつ$#m#=|E|$である。
さらに$V=\{0,\ldots,#n#-1\}$と仮定する。
他のデータを扱いたければ、大きさ#n#の配列にデータを入れて於けば良い。

グラフに対する典型的な操作は次のものだ。
\begin{itemize}
  \item #addEdge(i,j)#:辺$(#i#,#j#)$を$E$に加える。
  \item #removeEdge(i,j)#:辺$(#i#,#j#)$を$E$から除く。
  \item #hasEdge(i,j)#:$(#i#,#j#)\in E$ かどうかを調べる。
  \item #outEdges(i)#:$(#i#,#j#)\in E$を満たす整数整数$#j#$のリストを返す。
  \item #outEdges(i)#:$(#j#,#i#)\in E$を満たす整数整数$#j#$のリストを返す。
\end{itemize}

これらの操作を効率的に実装するのはさほど難しくない。
例えばはじめの3つの操作は#USet#を使って実装でき、\chapref{hashing}で説明したハッシュテーブルを使えば期待実行時間は定数である。
最後のふたつの操作は頂点を隣接行列のリストに入れれば定数時間で実行できる。

しかし、グラフにおける応用によって、各操作への要求が異なり、理想的にはこの要求をすべて満たす中で最も単純な実装を使いたい。
そのため、ふたつのグラフの表現方法のカテゴリについての説明をする。
