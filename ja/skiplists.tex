\chapter{スキップリスト}
\chaplabel{skiplists}

この章ではスキップリストというオシャレで実用的なデータ構造を紹介する。
スキップリストは#List#の実装であり、#get(i)#・#set(i,x)#・#add(i,x)#・#remove(i)#の実行時間はいずれも$O(\log n)$である。
#SSet#の実装でもあり、いずれの操作の期待実行時間も$O(\log #n#)$である。

スキップリストの効率性のキモはランダム性である。
新しい要素を追加するとき、スキップリストではランダムなコイントスの結果に応じて要素の高さを決める。
スキップリストの性能を要素を見つけるための経路の長さの期待値で表現する。
コイントスにより決まる確率からこの期待値を計算する。
ランダムなコイントスは擬似乱数生成器を使ってシミュレーションする。

\section{基本的な構造}

\ejindex{skiplist}{すきっぷりすと@スキップリスト}%
スキップリストは単方向連結リスト$L_0,\ldots,L_h$を並べたものだと考えられる。
#n#個の要素を含むスキップリストでは、$L_0$は#n#個の要素すべてを含む。
$L_0$から$L_1$を作り、$L_1$から$L_2$を作り、という作業を次のように繰り返す。
リスト$L_r$の要素は$L_{r-1}$の要素の部分集合である。
$L_{r-1}$のどの要素を$L_r$に含むかを決めるために、各要素についてコインを投げる。
表が出た要素を$L_r$に含める。
リスト$L_r$が空なら繰り返しを終える。
スキップリストの例を\figref{skiplist}に示した。

\begin{figure}
  \begin{center}
    \includegraphics[width=\ScaleIfNeeded]{figs/skiplist}
  \end{center}
  \caption{7つの要素を含むskiplistの例}
  \figlabel{skiplist}
\end{figure}

スキップリストのノード#x#について、#x#の\emph{高さ (height)}を
\ejindex{height!of a skiplist}{たかさ@高さ!すきっぷりすと@スキップリスト}%
#x#を含むリスト$L_r$の添え字#r#のうち最大のものと定義する。
例えば#x#が$L_0$だけに含まれているなら#x#の高さは$0$である。
少し考えると#x#は次の試行と関連していることがわかるだろう。

裏が出るまでコインを繰り返し投げる。表は何回出るだろうか。
この問いの答え、そしてxの高さの期待値は1である。(コイントスの回数の期待値は2回だが、最後のトスは表でないため表が出る回数の期待値は1だ。)
スキップリストの\emph{高さ}とは、最も高いノードの高さである。

すべてのリストの先頭は特別で、\emph{番兵}と呼ばれる。
\ejindex{sentinel node}{ばんぺい@番兵}%
これはリストのダミーノードのようなものだ。
スキップリストの重要な性質は、\emph{探索経路 (search path)}と呼ばれる
\ejindex{search path!in a skiplist}{たんさくけいろ@探索経路!すきっぷりすと@スキップリスト}%
$L_h$の番兵から$L_0$の各ノードまでの短いパスが存在することだ。
ノード#u#へのパスの作り方は簡単だ。(\figref{skiplist-searchpath}を参照のこと。)
左上の端($L_h$の番兵)からスタートし、#u#に到達したり#u#を通り越したりしない限り右に進み続ける。
#u#に到達する、または#u#を通し越してしまうときは右ではなく下に進む。

もうすこし正確に説明する。
$L_h$の番兵#w#から$L_0$のノード#u#への探索経路を見つける。
まず#w.next#を見て、これが$L_0$の中で#u#より前にあれば$#w#=#w.next#$とする。
そうでなければ、ひとつ下のリストに下がり、$L_{h-1}$の#w#から処理を続ける。
これを$L_0$における#u#の直前の要素にたどり着くまで繰り返す。
\begin{figure}
  \begin{center}
    \includegraphics[width=\ScaleIfNeeded]{figs/skiplist-searchpath}
  \end{center}
  \caption{あるskiplistにおける、$4$を含むノードの探索経路}
  \figlabel{skiplist-searchpath}
\end{figure}

次の補題は探索経路が非常に短いことを主張する。(\secref{skiplist-analysis}で証明する。)

\begin{lem}\lemlabel{skiplist-searchpath}
$L_0$のノード#u#について、#u#の探索経路長の期待値は$2\log #n# + O(1) = O(\log #n#)$である。
\end{lem}

空間効率のよいスキップリストの実装方法を説明する。
ノード#u#はデータ#x#・ポインタの配列#next#を含む。
#u.next[i]#で$L_{#i#}$における#u#の次のノードを指せばよい。
こうすると#x#は複数のリストに現れるかもしれないが、ノードとしての実体はひとつだけあれば済む。

% XXX: height は length of next - 1 ではないか
\javaimport{ods/SkiplistSSet.Node<T>}
\cppimport{ods/SkiplistSSet.Node}

この章の続くふたつの節ではスキップリストの応用を紹介する。
いずれの場合でも$L_0$に主な構造(リストや整列された集合)を入れる。
違いは探索経路の辿り方である。
$L_r$にいるとき下($L_{r-1}$)に向かうか($L_r$のまま)右に進むかが異なることがある。

\section{#SkiplistSSet#:効率的な#SSet#}
\seclabel{skiplistset}

\index{SkiplistSSet@#SkiplistSSet#}%
#SkiplistSSet#はスキップリストを使った#SSet#インターフェースの実装である。
ここでは、$L_0$は#SSet#の要素を整列して格納する。
#find(x)#は探索経路に沿って$#y#\ge#x#$を満たす最小の#y#を探す。

\codeimport{ods/SkiplistSSet.find(x).findPredNode(x)}

#y#の探索経路を辿るのは簡単だ。
$L_{#r#}$の中のノード#u#にいるとすると、まず右隣#u.next[r].x#を見る。
$#x#>#u.next[r].x#$なら$L_{#r#}$の中で右に進む。
そうでないなら$L_{#r#-1}$に下がる。
各ステップ(右または下に進む)は一定の時間で実行できる。
よって\lemref{skiplist-searchpath}より#find(x)#の期待実行時間は$O(\log #n#)$である。

#SkipListSSet#に要素を追加する方法の前に、新しいノードの高さ#k#を決めるためのコイントスをシミュレートする方法を考える。
ランダムな整数#z#を生成し、#z#の2進数表現において連続する1の数を数える。
\footnote{この方法はコイントスを完全に再現しているわけではない。なぜなら#k#は#int#のビット数より常に小さいからである。しかし要素数が$2^{32}=4294967296$を越える場合でもない限り、この影響は無視できるほど小さい。}

\codeimport{ods/SkiplistSSet.pickHeight()}

#SkiplistSSet#の#add(x)#の実装は、#x#を入れる場所を見つけ、高さ#k#を#pickHeight()#で決め、#x#を$L_0$,\ldots,$L_{#k#}$に継ぎ合せる。
これを実現する最も簡単な方法は、リスト$L_{#r#}$からリスト$L_{#r#-1}$に下がるノードを記録する配列#stack#を使うことだ。
より正確にいうと、#stack[r]#には探索経路において$L_{#r#}$から$L_{#r#-1}$に下がるノードが記録されている。
#x#を挿入する時に修正する必要のあるノードはちょうど$#stack[0]#,\ldots,#stack[k]#$である。
次のコードはこの#add(x)#アルゴリズムの実装である。
\label{pg:skiplist-add}
\codeimport{ods/SkiplistSSet.add(x)}

\begin{figure}
  \begin{center}
    \includegraphics[width=\ScaleIfNeeded]{figs/skiplist-add}
  \end{center}
  \caption{値$3.5$を含むノードをskiplistに追加する。#stack#に格納されるノードを強調している。}
  \figlabel{skiplist-add}
\end{figure}

要素#x#の削除も同様に行える。ただし#stack#を使って探索経路を記録する必要はない。
削除は探索経路を辿りながら行える。
#x#を探す途中にノード#u#から下に向かうとき、$#u.next.x#=#x#$なら#u#を繋ぎ替える。
\codeimport{ods/SkiplistSSet.remove(x)}

\begin{figure}
  \begin{center}
    \includegraphics[width=\ScaleIfNeeded]{figs/skiplist-remove}
  \end{center}
  \caption{値$3$を含むノードをskiplistから削除する}
  \figlabel{skiplist-remove}
\end{figure}

\subsection{要約}

次の定理はスキップリストを使った整列集合の性能をまとめたものだ。

\begin{thm}\thmlabel{skiplist}
#SkiplistSSet#は#SSet#インターフェースの実装である。
#SkiplistSSet#における#add(x)#・#remove(x)#・#find(x)#の実行時間の期待値はいずれも$O(\log #n#)$である。
\end{thm}

\section{#SkiplistList#:効率的なランダムアクセス#List#}
\seclabel{skiplistlist}

\index{SkiplistList@#SkiplistList#}%
#SkiplistList#はスキップリストを使った#List#インターフェースの実装だ。
#SkiplistList#では、$L_0$はリストの要素をリストにおける順序通りに含む。
#SkiplistSSet#と同様に、要素の追加・削除・読み書きのいずれの実行時間も$O(\log #n#)$である。

これを可能にするためにはまず$L_0$における#i#番目の要素を見つける方法が必要だ。
このための最も簡単な方法はリスト$L_{#r#}$における\emph{辺の長さ}を定義することだ。
$L_{0}$における辺の長さをいずれも1とする。
$L_{#r#} (#r#>0)$の辺#e#の辺の長さを、$L_{#r#-1}$において#e#の下にある辺の長さの和とする。
これは$L_0$において#e#の下にある辺の数を#e#の長さとするのと等価な定義である。
この定義の例として\figref{skiplist-lengths}を参照せよ。
スキップリストの辺は配列に格納されており、その長さも同様に格納すればよい。

\begin{figure}
  \begin{center}
    \includegraphics[width=\ScaleIfNeeded]{figs/skiplist-lengths}
  \end{center}
  \caption{skiplistにおける辺の長さ}
  \figlabel{skiplist-lengths}
\end{figure}

\javaimport{ods/SkiplistList.Node}
\cppimport{ods/SkiplistList.Node}

この定義の良い性質は、$L_0$においてj番目のノードから長さ$\ell$の辺を辿ると、$L_0$において$#j#+\ell$のノードに移ることだ。
こうして、探索パスを辿りながら$L_0$におけるインデックス#j#を算出できる。
$L_{#r#}$のノード#u#にいるとき、辺#u.next[r]#の長さと#j#の和が#i#より小さいなら右に進む。
そうでないなら下($L_{#r#-1}$)に進む。

\codeimport{ods/SkiplistList.findPred(i)}
\codeimport{ods/SkiplistList.get(i).set(i,x)}

#get(i)#・#set(i,x)#において最も計算時間がかかるのは$L_0$の#i#番目のノードを見つける処理なので、これらの操作の実行時間は$O(\log #n#)$である。

#SkiplistList#の#i#番目の位置に要素を追加するのは簡単だ。
#SkiplistSSet#とは違い新しいノードが必ず追加されるので、ノードの位置を見つける処理とノードを追加する処理とを同時に実行できる。
まずは新たに挿入するノード#w#の高さ#k#を決め、#i#の探索経路を辿る。
$L_{#r#}$から下に進むのは$#r#\le #k#$のときで、このとき#w#を$L_{#r#}$とを継ぎ合わせる。
このとき辺の長さも適切に更新する必要があることに注意する。
\figref{skiplist-addix}を見よ。

\begin{figure}
  \begin{center}
    \includegraphics[width=\ScaleIfNeeded]{figs/skiplist-addix}
  \end{center}
  \caption{#SkiplistList#への要素の追加}
  \figlabel{skiplist-addix}
\end{figure}

探索経路上でリスト$L_{#r#}$のノード#u#に降りたとき、
#i#番目の位置に要素を追加するため辺#u.next[r]#の長さをひとつ大きくする。
ノード#w#をふたつのノード#u#と#z#の間に追加する様子を\figref{skiplist-lengths-splice}に示す。
探索経路を辿りながら$L_0$において#u#が何番目なのかを数えられる。
そのため#u#から#w#までの辺の長さは$#i#-#j#$とわかる。
さらに、#u#から#z#への辺の長さ$\ell$から、#w#から#z#への辺の長さを計算できる。
こうして#w#を挿入し、関連する辺の長さの更新を定数時間で終えることができる。

\begin{figure}
  \begin{center}
    \includegraphics[scale=0.90909]{figs/skiplist-lengths-splice}
  \end{center}
  \caption{ノード#w#をskiplistに追加するときの、辺の長さの更新}
  \figlabel{skiplist-lengths-splice}
\end{figure}

複雑そうに聞こえるかもしれないが、実際のコードはとても単純である。

\codeimport{ods/SkiplistList.add(i,x)}
\codeimport{ods/SkiplistList.add(i,w)}

ここまでの話から#SkiplistList#における#remove(i)#の実装は明らかである。
#i#番目の位置への探索経路を辿る。
高さ#r#のノード#u#から経路が下に向かうとき、その高さにおける#u#から出る辺の長さをひとつ小さくする。
また、#u.next[r]#が高さ#i#の要素であるかどうかを確認し、もしそうならリストからそれを除く。
\figref{skiplist-removei}に例が描かれている。

\begin{figure}
  \begin{center}
    \includegraphics[width=\ScaleIfNeeded]{figs/skiplist-removei}
  \end{center}
  \caption{#SkiplistList#から要素を削除する。}
  \figlabel{skiplist-removei}
\end{figure}
\codeimport{ods/SkiplistList.remove(i)}

\subsection{要約}

次の定理は#SkiplistList#の性能をまとめたものだ。

\begin{thm}\thmlabel{skiplistlist}
  #SkiplistList#は#List#インターフェースを実装する。
  #SkiplistList#における#get(i)#・#set(i,x)#・#add(i,x)#・#remove(i)#の実行時間の期待値はいずれも$O(\log #n#)$である。
\end{thm}

\section{スキップリストの解析}
\seclabel{skiplist-analysis}

この節では高さ・大きさ・探索経路の長さの期待値を解析する。
ここでは基本的な確率論の知識を前提とする。
いくつかの証明は次に述べるコイントスについての考察を利用する。

\begin{lem}\lemlabel{coin-tosses}
  \ejindex{coin toss}{こいんなげ@コイン投げ}%
  $T$を表裏が等しい確率で出るコインを投げて、表が出るまでに要するコイントスの回数とする。(表が出た回も含めて数えることに注意する。)このとき、$\E[T]=2$である。
\end{lem}

\begin{proof}
表が出るまでコイントスを繰り返すとき、次の指示変数を定義する。
  \[ I_{i} = \left\{\begin{array}{ll}
     0 & \mbox{コイントスが$i$回よりも少ないとき} \\
     1 & \mbox{コイントスが$i$回以上のとき}
     \end{array}\right. % "\right."はドットまで含めてlatexの文法なので消さないで
  \]
  $I_i=1$なのは最初の$i-1$回の結果がいずれも裏であることと同値である。
  よって$\E[I_i]=\Pr\{I_i=1\}=1/2^{i-1}$である。
  コイントスの合計回数$T$は$T=\sum_{i=1}^{\infty} I_i$と書ける。
  以上より、次のことがわかる。
  \begin{align*}
    \E[T] & =  \E\left[\sum_{i=1}^\infty I_i\right] \\
     & =  \sum_{i=1}^\infty \E\left[I_i\right] \\
     & =  \sum_{i=1}^\infty 1/2^{i-1} \\
     & =  1 + 1/2 + 1/4 + 1/8 + \cdots \\
     & =  2 \enspace \qedhere
  \end{align*}
\end{proof}

次のふたつの補題ではスキップリストにおけるノード数は要素数に概ね比例することを示す。

\begin{lem}\lemlabel{skiplist-size1}
  $#n#$要素からなるスキップリストにおける(番兵を除く)ノード数の期待値は $2#n#$である。
\end{lem}

\begin{proof}
  要素#x#がリスト$L_{#r#}$に含まれる確率は$1/2^{#r#}$である。
  よって$L_{#r#}$のノード数の期待値は$#n#/2^{#r#}$である。
  \footnote{指示変数と期待値の線形性からこの結果を得る方法は\secref{randomization}を参照せよ。}
  以上よりすべてのリストに含まれるノードの総数の期待値が求まる。
  \[ \sum_{#r#=0}^\infty #n#/2^{#r#} = #n#(1+1/2+1/4+1/8+\cdots) = 2#n# \enspace \qedhere \]
\end{proof}

\begin{lem}\lemlabel{skiplist-height}
  #n#要素を含むスキップリストの高さの期待値は$\log #n# + 2$以下である。
\end{lem}

\begin{proof}
  $#r#\in\{1,2,3,\ldots,\infty\}$について次の確率変数を定義する。
  \[ I_{#r#} = \left\{\begin{array}{ll}
     0 & \mbox{$L_{#r#}$が空のとき} \\
     1 & \mbox{$L_{#r#}$が空でないとき}
     \end{array}\right.
  \]
  スキップリストの高さ#h#は次のように計算できる。
  \[
       #h# = \sum_{r=1}^\infty I_{#r#}
  \]
  $I_{#r#}$はリスト$L_{#r#}$の長さ$|L_{#r#}|$を越えないことに注意する。
  \[
     \E[I_{#r#}] \le \E[|L_{#r#}|] = #n#/2^{#r#} \enspace
  \]
  よって
  \begin{align*}
       \E[#h#] &= \E\left[\sum_{r=1}^\infty I_{#r#}\right] \\
        &= \sum_{#r#=1}^{\infty} E[I_{#r#}] \\
        &= \sum_{#r#=1}^{\lfloor\log #n#\rfloor} E[I_{#r#}]
                 + \sum_{r=\lfloor\log #n#\rfloor+1}^{\infty} E[I_{#r#}]  \\
        &\le \sum_{#r#=1}^{\lfloor\log #n#\rfloor} 1
                 + \sum_{r=\lfloor\log #n#\rfloor+1}^{\infty} #n#/2^{#r#} \\
        &\le \log #n#
                 + \sum_{#r#=0}^\infty 1/2^{#r#} \\
        &= \log #n# + 2 \enspace \qedhere
  \end{align*}
\end{proof}

\begin{lem}\lemlabel{skiplist-size2}
  $#n#$要素からなるスキップリストのノード数の期待値は、番兵を含めて$2#n#+O(\log #n#)$である。
\end{lem}

\begin{proof}
  \lemref{skiplist-size1}より番兵を除いたノード数の期待値は$2#n#$である。
  番兵の数の期待値はスキップリストの高さ$#h#$に等しく、
  これは\lemref{skiplist-height}より$\log #n#+2 = O(\log #n#)$以下である。
  \end{proof}



\begin{lem}
スキップリストにおける探索経路の長さの期待値は$2\log #n# + O(1)$以下である。
\end{lem}

\begin{proof}
最も簡単な証明方法はノード#x#の\emph{逆探索経路}を考えることだ。
この経路は$L_0$における#x#の直前のノードから始まる。
パスが上に向かえるときはそうする。
そうでないなら左に進む。
少し考えると、#x#の逆探索経路は探索経路と方向が逆であることを除いて同じであることがわかるだろう。

ある高さで逆探索経路が通過するノードの数#r#は次の試行と関連している。
コインを投げる。
表が出れば上に向かい停止する。
裏が出れば左に向かい試行を続ける。
このとき、表が出るまでにコインを投げる回数は逆探索経路のある高さで左に向かうステップの数に対応している。
\footnote{これは左に向かう回数を大きく数えてしまうかもしれない。なぜなら実際には試行は表が出るか番兵に出くわすかのどちらかが起きたときに終えるべきだからである。しかし今考えている補題は上界に関するものであるので、ここでは問題にならない。}
\lemref{coin-tosses}よりはじめて表が出るまでのコイントスの回数の期待値は1である。

$S_{#r#}$を(順方向の)探索経路における高さ$#r#$で右に進む回数を表す。
$\E[S_{#r#}]\le 1$である。
さらに$L_{#r#}$では$L_{#r#}$の長さより多く右に進むことはないので$S_{#r#}\le |L_{#r#}|$である。
よって次の式が成り立つ。
  \[
    \E[S_{#r#}] \le \E[|L_{#r#}|] = #n#/2^{#r#} \enspace
  \]
あとは\lemref{skiplist-height}と同様に証明を完成できる。
$S$をスキップリストにおけるノード#u#の探索経路の長差とする。
また、#h#をそのスキップリストの高さとする。
このとき、次の式が成り立つ。
  \begin{align*}
      \E[S] 
         &= \E\left[ #h# + \sum_{#r#=0}^\infty S_{#r#} \right] \\
         &= \E[#h#] + \sum_{#r#=0}^\infty \E[S_{#r#}]  \\
         &= \E[#h#] + \sum_{#r#=0}^{\lfloor\log #n#\rfloor} \E[S_{#r#}] 
              + \sum_{#r#=\lfloor\log #n#\rfloor+1}^\infty \E[S_{#r#}] \\
         &\le \E[#h#] + \sum_{#r#=0}^{\lfloor\log #n#\rfloor} 1
              + \sum_{r=\lfloor\log #n#\rfloor+1}^\infty #n#/2^{#r#} \\
         &\le \E[#h#] + \sum_{#r#=0}^{\lfloor\log #n#\rfloor} 1
              + \sum_{#r#=0}^{\infty} 1/2^{#r#} \\
         &\le \E[#h#] + \sum_{#r#=0}^{\lfloor\log #n#\rfloor} 1
              + \sum_{#r#=0}^{\infty} 1/2^{#r#} \\
         &\le \E[#h#] + \log #n# + 3 \\
         &\le 2\log #n# + 5  \enspace \qedhere
  \end{align*}
\end{proof}


次の定理はこの節の結果をまとめるものだ。
\begin{thm}
$#n#$要素を含むスキップリストの大きさの期待値は$O(#n#)$である。
ある要素の探索経路の長さの期待値は$2\log #n# + O(1)$以下である。
\end{thm}

\section{ディスカッションと練習問題}

スキップリストを提案したのはPugh \cite{p91}であり、多くの拡張や応用も提案されている\cite{p89}。
その後さらに多くの研究が行われている。
スキップリストの#i#番目の要素を見つける探索経路の長さの期待値や分散はより正確に求められている。 \cite{kp94,kmp95,pmp92}
決定的な(ランダム性を用いない)変種や\cite{mps92}偏りのある変種\cite{bbg02,esss01}、適応的な変種\cite{bdl08}も開発されている。
スキップリストは様々な言語やフレームワークで書かれ、またオープンソースのデータベースシステムで使われている。\cite{skipdb,redis}
スキップリストの変種がオペレーティング・システムHP-UXにおけるカーネルのプロセス制御の構造として使われている。\cite{hpux}
\javaonly{SkiplistsはJava 1.6 APIの一部として含まれている。 \cite{oracle_jdk6}}

\begin{exc}
\figref{skiplist}のスキップリストにおける2.5と5.5の探索経路を説明せよ。
\end{exc}

\begin{exc}
\figref{skiplist}のスキップリストに対して、値0.5の要素を高さ1で追加し、その後値3.5の要素を高さ2で追加するときの振る舞いを説明せよ。
\end{exc}

\begin{exc}
\figref{skiplist}のスキップリストから1と3を削除するときの振る舞いを説明せよ。
\end{exc}

\begin{exc}
\figref{skiplist-lengths}の#SkiplistList#に#remove(2)#を実行する時の振る舞いを説明せよ。
\end{exc}

\begin{exc}
\figref{skiplist-lengths}の#SkiplistList#に#add(3,x)#を実行する時の振る舞いを説明せよ。
なお、#pickHeight()#は新たなノードの高さとして4を選択すると仮定せよ。
\end{exc}

\begin{exc}\exclabel{skiplist-changes}
#add(x)#または#remove(x)#を実行するとき、#SkiplistSet#のポインタのうち操作されるものの数の期待値は定数であることを示せ。
\end{exc}

\begin{exc}\exclabel{skiplist-opt}
$L_{i-1}$から$L_i$に要素を上げるかどうかを決めるとき、コイントスではなく確率$p (0 < p < 1)$を用いる。
  \begin{enumerate}
   \item このとき探索経路長の期待値は$(1/p)\log_{1/p} #n# + O(1)$以下であることを示せ。
   \item これを最小にする$p$を求めよ。
   \item スキップリストの高さの期待値を求めよ。
   \item スキップリストのノード数の期待値を求めよ。
  \end{enumerate}
\end{exc}


\begin{exc}\exclabel{skiplist-opt-2}
  #SkiplistSet#の#find(x)#は\emph{冗長な比較}を行うことがある。
  これは#x#と同じ値の比較を複数回行うことである。
  $#u.next[r]# = #u.next[r-1]#$を満たすノード#u#が存在すると発生する。
  どのように冗長な比較が発生するかを説明し、#find(x)#においてこれが発生しないようにする方法を示せ。
  そして、このように修正した#find(x)#での比較操作の回数を解析せよ。
\end{exc}

\begin{exc}
#SSet#における要素#x#のランクとは、#SSet#の要素であって、#x#より小さいものの個数である。
#SSet#インターフェースの実装であり、ランクによる要素への高速アクセスが可能なスキップリストを設計・実装せよ。これはランク#i#の要素を返す#get(i)#を持つ。この操作の実行時間は$O(\log #n#)$である。
\end{exc}

\begin{exc}
  \ejindex{finger}{ゆび@指}%
  \ejindex{finger search!in a skiplist}{ふぃんがーさーち@フィンガーサーチ!すきっぷりすと@スキップリスト}%
XXX: 日本語汚い

% 指 or 検索指
スキップリストの\emph{指}とは探索経路において下に向かうノードを保存した配列である。 (\pageref{pg:skiplist-add}のコードで#add(x)#における変数#stack#は\emph{指}である。\figref{skiplist-add}において影になっているノードは指を表している。)
指は$L_0$における経路を示していると解釈することもできる。

\emph{指探索}は指を利用した#find(x)#の実装である。
$#u.x# < #x#$かつ$(#u.next#=#null# or #u.next.x# > #x#)$を満たすノード#u#に到達するまで指の要素を見ていき、そして#u#からふつうの#x#の探索を実行する。
$L_0$において#b#と指が指す値との間にある値の数を$r$とするとき、\emph{指探索}のステップ数の期待値は$O(1+\log r)$である。

#Skiplist#のサブクラス#SkiplistWithFinger#を実装せよ。
これは#find(x)#を指を利用して実装する。
このサブクラスでは指を保持し、指探索によって#find(x)#を実装する。
#find(x)#は#x#の位置を返すと同時に、指が常に前回の#find(x)#の結果を指すように更新する。
\end{exc}

\begin{exc}\exclabel{skiplist-truncate}
#truncate(i)#を実装せよ。
これは#SkiplistList#を#i#番目の位置で切り詰める操作である。
このメソッドを実行するとリストの大きさは#i#になり、リストは添え字$0,\ldots,#i#-1$の要素のみを含むようになる。
返り値は#SkiplistList#であって、添え字$#i#,\ldots,#n#-1$の要素を含むものである。
このメソッドの実行時間は$O(\log #n#)$でなければならない。
\end{exc}

\begin{exc}
#SkiplistList#の操作#absorb(l2)#を実装せよ。
これは#SkiplistList# #l2#を引数に取り、これを空にし、元々入っていた要素をそのままの順番でレシーバーに追加するものだ。
例えば、#l1#の要素が$a,b,c$、#l2#の要素が$d,e,f$であるとき、#l1.absorb(l2)#を呼ぶと、#l1#の要素は$a,b,c,d,e,f$になり#l2#は空になる。
このメソッドの実行時間は$O(\log #n#)$でなければならない。
\end{exc}

\begin{exc}
#SEList#のアイデアを転用し、空間効率の良い#SSet#である#SESSet#を設計・実装せよ。
要素を順に#SEList#に格納し、この#SEList#のブロックを#SSet#に格納すればよい。
もし使った#SSet#の実装が#n#要素を$O(#n#)$のメモリだけを使って保持できるなら、#SESSet#は#n#要素を格納ためのメモリに加えて、消費する無駄なメモリは$O(#n#/#b#+#b#)$である。
\end{exc}

\begin{exc}
#SSet#を使って、(大きな)テキストを読み込み、そのテキストの任意の部分文字列をインタラクティブに検索できるアプリケーションを設計・実装せよ。
このアプリはユーザーがクエリを入力するときテキストのマッチしている部分があればこれを結果として返す。

  \noindent ヒント1:任意の部分文字列はある接尾辞の接頭辞である。よってテキストファイルのすべての接尾辞を保存すれば十分である。

  \noindent ヒント2:任意の接尾辞はテキストの中のどこから接尾辞が始まるのかを表す一つの整数でコンパクトに表現できる。

  \noindent 書いたアプリケーションを長いテキストでテストせよ。
  プロジェクトGutenberg \cite{gutenberg}から本を入手できる。
  正しく動いたら、レスポンスを速くしよう。すなわち、キー入力から結果が得られるまでに要する時間を認識できないくらい小さくしよう。
\end{exc}

\begin{exc}
  \ejindex{skiplist!versus binary search tree}{すきっぷりすと@スキップリスト!にぶんたんさくぎとのひかく@二分探索木との比較}%
  \ejindex{binary search tree!versus skiplist}{にぶんたんさくぎ@二分探索木!すきっぷりすととのひかく@スキップリストとの比較}%
  (この練習問題は\secref{binarysearchtree}で二分探索木について学んでから取り組むべきだ。)
  スキップリストを二分探索木と比較せよ。
  \begin{enumerate}
     \item スキップリストの辺を削除することが、二分木のようにみえること、また二分探索木ににていることを説明せよ。
     \item スキップリストと二分探索木は使うポインタの数は同じである。(ノードあたり2つ)
	 しかしスキップリストの方がこれを上手く使っている。
	 これは何故か、説明せよ。
  \end{enumerate}
\end{exc}
