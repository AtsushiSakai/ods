\chapter{スキップリスト}
\chaplabel{skiplists}

この章ではスキップリストという素晴らしいデータ構造を紹介する。スキップリストはさまざまな形で活用できる。
例えば、#get(i)#、#set(i,x)#、#add(i,x)#、#remove(i)#の実行時間がいずれも$O(\log #n#)$であるような#List#を実装できる。
また、すべての操作の期待実行時間が$O(\log #n#)$であるような#SSet#もスキップリストで実装できる。

スキップリストが効率的なのはランダム性を利用していることによる。
新しい要素を追加するとき、スキップリストでは、ランダムなコイントスの結果に応じて要素の高さを決める。
スキップリストの性能評価には、実行時間の期待値と、要素を見つけるための経路の長さの期待値を使う。
期待値で評価するのは、スキップリストではコイントスにより決まる確率を利用するからだ。
スキップリストで使うコイントスは、実際には擬似乱数生成器を使ってシミュレーションする。

\section{基本的な構造}

\ejindex{skiplist}{すきっぷりすと@スキップリスト}%
スキップリストは、単方向連結リスト$L_0,\ldots,L_h$を並べたものだと考えられる。
#n#個の要素を含むスキップリストを例に考えてみよう。1つめの単方向連結リストである$L_0$には、それら#n#個の要素をすべて含める。
その$L_0$から一部の要素を取り出して$L_1$を作る。さらに$L_1$から$L_2$を作る。これを繰り返す。
$L_{r-1}$の要素を$L_r$に含めるかどうかは、各要素についてコインを投げて決める。
投げたコインが表なら、その要素を$L_r$に含める。
最終的に$L_r$が空になったら繰り返しを終える。
スキップリストの例を\figref{skiplist}に示す。

\begin{figure}
  \begin{center}
    \includegraphics[width=\ScaleIfNeeded]{figs/skiplist}
  \end{center}
  \caption{7つの要素を含むスキップリストの例}
  \figlabel{skiplist}
\end{figure}

スキップリストの各要素#x#について、
#x#を含む単方向連結リスト$L_r$の添字#r#のうち最大のものを、
#x#の\emph{高さ(height)}と定義する。
\ejindex{height!of a skiplist}{たかさ@高さ!すきっぷりすと@スキップリスト}%
例えば、#x#が$L_0$だけに含まれているなら、#x#の高さは$0$である。
少し考えてみればわかるように、#x#は、裏が出るまでにコインを繰り返し投げた回数である。
それまでに表は何回出ただろうか。
この問いの答え、そして#x#の高さの期待値は、当然1である(コイントスの回数の期待値としては2だが、最後のトスは表ではないので、表が出た回数の期待値は1になる)。
スキップリストの各要素の高さのうち、最も高いものを、そのスキップリストの\emph{高さ}と定義する。

すべてのリストの先頭には、\emph{番兵(sentinel)}と呼ばれる特別なノードを置く。
\ejindex{sentinel node}{ばんぺい@番兵}%
これは、当該のリストに対するダミーノードとして機能する。
スキップリストの重要な性質は、
$L_h$の番兵から$L_0$の各ノードまでの短いパスが存在することだ。
このパスを\emph{探索経路(search path)}と呼ぶ。
\ejindex{search path!in a skiplist}{たんさくけいろ@探索経路!すきっぷりすと@スキップリスト}%
あるノード#u#への探索経路は簡単に構築できる。
左上端のノード(つまり$L_h$の番兵)からスタートし、#u#に到達したり#u#を通り越したりしない限り、右へと進み続ける。
#u#に到達、もしくは#u#を通り越してしまうときは、右ではなく下に進む(\figref{skiplist-searchpath}参照)。

もう少し正確に説明する。
$L_h$の番兵#w#から$L_0$のノード#u#への探索経路を見つけるには、
まず#w.next#を見て、これが$L_0$の中で#u#より前にあれば$#w#=#w.next#$とする。
そうでなければ1つ下のリストに降りて、$L_{h-1}$における#w#から処理を続ける。
これを$L_0$における#u#の直前の要素に辿り着くまで繰り返す。
\begin{figure}
  \begin{center}
    \includegraphics[width=\ScaleIfNeeded]{figs/skiplist-searchpath}
  \end{center}
  \caption{あるスキップリストにおける、$4$を含むノードの探索経路}
  \figlabel{skiplist-searchpath}
\end{figure}

このような探索経路は、次の補題が示すように、かなり短い(この補題は\secref{skiplist-analysis}で証明する)。

\begin{lem}\lemlabel{skiplist-searchpath}
$L_0$のノード#u#について、#u#の探索経路の長さの期待値は$2\log #n# + O(1) = O(\log #n#)$である。
\end{lem}

空間効率のよい方法でスキップリストを実装するには、
ノード#u#がデータ#x#とポインタの配列#next#を含むようにし、
#u.next[i]#が$L_{#i#}$における#u#の次のノードを指すようにすればよい。
#x#は複数のリストに現れることがあるが、こうすればノードとしての実体は1つだけあれば済む。

% XXX: height は length of next - 1 ではないか -- 実装を見てもそうなっているっぽい。コード中のコメントは不正確なので一旦消した。Pat に確認する
\javaimport{ods/SkiplistSSet.Node<T>}
\cppimport{ods/SkiplistSSet.Node}

以降の2つの節ではスキップリストの応用を紹介する。
いずれの応用でも、主な構造(リストや整列された集合)は$L_0$に入れる。
2つの応用の違いは、探索経路の辿り方である。
特に、$L_r$から下($L_{r-1}$)に降りるか$L_r$の中でそのまま右に進むかを決める方法が異なる。

\section{#SkiplistSSet#:効率的な#SSet#}
\seclabel{skiplistset}

\index{SkiplistSSet@#SkiplistSSet#}%
#SkiplistSSet#は、スキップリストを使った#SSet#インターフェースの実装である。
#SSet#インターフェースの実装にスキップリストを使う場合、$L_0$には#SSet#の要素を整列して格納する。
探索経路に沿って$#y#\ge#x#$を満たすような最小の#y#を探すメソッド#find(x)#を下記に示す。

\codeimport{ods/SkiplistSSet.find(x).findPredNode(x)}

#y#までの探索経路を辿るのは簡単だ。
$L_{#r#}$の中のノード#u#にいるとしたら、まず右隣#u.next[r].x#を見る。
$#x#>#u.next[r].x#$なら、$L_{#r#}$の中で右に進む。
そうでないなら、$L_{#r#-1}$に下がる。
各ステップ(右または下に進む)は一定の時間で実行できる。
よって、\lemref{skiplist-searchpath}より、#find(x)#の期待実行時間は$O(\log #n#)$である。

#SkipListSSet#に要素を追加する方法を考えるには、新しいノードの高さ#k#を決めるコイントスのシミュレート方法を考える必要がある。
これには、ランダムな整数#z#を生成し、#z#の2進数表現において連続する1の数を数える
\footnote{#k#は#int#のビット数より常に小さくなるから、この方法でコイントスを完全には再現できない。しかし、要素数が$2^{32}=4294967296$を越える場合でもない限り、この影響は無視できるほど小さい。}。

\codeimport{ods/SkiplistSSet.pickHeight()}

#SkiplistSSet#の#add(x)#を実装するには、#x#を入れる場所を見つけ、高さ#k#を#pickHeight()#で決め、#x#を$L_0$,\ldots,$L_{#k#}$に継ぎ足す。
これを最も簡単に実現するため、配列#stack#を用意し、探索経路においてリスト$L_{#r#}$からリスト$L_{#r#-1}$へと下がる箇所に該当するノードを記録しておく。
より正確にいうと、探索経路において$L_{#r#}$から$L_{#r#-1}$へと下がるときの$L_{#r#}$のノードを#stack[r]#とする。
こうすると、#x#の挿入時に修正が必要なノードが、まさに$#stack[0]#,\ldots,#stack[k]#$になる(\figref{skiplist-add})。
このアルゴリズムを利用した#add(x)#の実装を下記に示す。
\label{pg:skiplist-add}

\codeimport{ods/SkiplistSSet.add(x)}

\begin{figure}
  \begin{center}
    \includegraphics[width=\ScaleIfNeeded]{figs/skiplist-add}
  \end{center}
  \caption{値$3.5$を含むノードをskiplistに追加する。#stack#に格納されるノードを強調している}
  \figlabel{skiplist-add}
\end{figure}

要素#x#の削除も同様だ。ただし、削除では#stack#を使って探索経路を記録する必要はない。
削除は探索経路を辿っていく途中でできる。
#x#を探す途中にノード#u#から下に向かうとき、$#u.next.x#=#x#$であるなら#u#を繋ぎ替えればよい(\figref{skiplist-remove})。
\codeimport{ods/SkiplistSSet.remove(x)}

\begin{figure}
  \begin{center}
    \includegraphics[width=\ScaleIfNeeded]{figs/skiplist-remove}
  \end{center}
  \caption{値$3$を含むノードをskiplistから削除する}
  \figlabel{skiplist-remove}
\end{figure}

\subsection{要約}

\thmref{skiplist}は、全順序集合(#SSet#)の実装にスキップリストを使った場合の性能について要約したものだ。

\begin{thm}\thmlabel{skiplist}
#SkiplistSSet#は#SSet#インターフェースの実装である。
#SkiplistSSet#における#add(x)#、#remove(x)#、#find(x)#の実行時間の期待値は、いずれも$O(\log #n#)$である。
\end{thm}

\section{#SkiplistList#:効率的なランダムアクセス#List#}
\seclabel{skiplistlist}

\index{SkiplistList@#SkiplistList#}%
#SkiplistList#は、スキップリストを使った#List#インターフェースの実装である。
#SkiplistList#では、$L_0$に、リスト#List#の要素がリストにおける順番通りに含まれる。
#SkiplistSSet#と同様に、要素の追加、削除、読み書きのいずれの実行時間も$O(\log #n#)$である。

これを可能にするためには、まず$L_0$における#i#番めの要素を見つける方法が必要だ。
これを最も簡単に実現するため、リスト$L_{#r#}$の各辺について\emph{長さ}を定義する。
$L_{0}$については、どの辺の長さも1とする。
$L_{#r#}$($#r#>0$)については、辺#e#の長さを、$L_{#r#-1}$において#e#の下にある辺の長さの和とする。
これは、#e#の下にある$L_0$の辺の数と等しい。
あるスキップリストに辺を併記した例を\figref{skiplist-lengths}に示す。
スキップリストの辺は配列に格納されているので、その長さも同様にして格納すればよい。

\begin{figure}
  \begin{center}
    \includegraphics[width=\ScaleIfNeeded]{figs/skiplist-lengths}
  \end{center}
  \caption{skiplistにおける辺の長さ}
  \figlabel{skiplist-lengths}
\end{figure}

\javaimport{ods/SkiplistList.Node}
% XXX: height は length of next - 1 ではないか -- 実装を見てもそうなっているっぽい。コード中のコメントは不正確なので一旦消した。Pat に確認する
\cppimport{ods/SkiplistList.Node}

この定義には、$L_0$においてj番めのノードから長さ$\ell$の辺を辿ると、$L_0$において$#j#+\ell$のノードに移るという良い性質がある。
これを利用すれば、探索経路を辿りながら$L_0$におけるインデックス#j#を算出できる。
$L_{#r#}$のノード#u#にいるとき、辺#u.next[r]#の長さと#j#の和が#i#より小さいなら右に進む。
そうでないなら下($L_{#r#-1}$)に進む。

\codeimport{ods/SkiplistList.findPred(i)}
\codeimport{ods/SkiplistList.get(i).set(i,x)}

#get(i)#および#set(i,x)#において最も計算時間がかかるのは$L_0$の#i#番めのノードを見つける処理なので、#get(i)#および#set(i,x)#操作の実行時間は$O(\log #n#)$である。

#SkiplistList#の#i#番めの位置に要素を追加するのは簡単だ。
#SkiplistSSet#とは違い、新しいノードが必ず追加されるので、ノードの位置を見つける処理とノードを追加する処理とを同時に実行できる。
まずは新たに挿入するノード#w#の高さ#k#を決め、#i#の探索経路を辿る。
$L_{#r#}$から下に進むのは$#r#\le #k#$のときで、このとき#w#を$L_{#r#}$に継ぎ足す。
その際には辺の長さを適切に更新する必要があることにも注意する
(\figref{skiplist-addix}参照)。

\begin{figure}
  \begin{center}
    \includegraphics[width=\ScaleIfNeeded]{figs/skiplist-addix}
  \end{center}
  \caption{#SkiplistList#への要素の追加}
  \figlabel{skiplist-addix}
\end{figure}

探索経路上でリスト$L_{#r#}$のノード#u#に降りたときは、
#i#番めの位置に要素を追加するため、辺#u.next[r]#の長さを1つ大きくする。
ノード#w#を2つのノード#u#と#z#の間に追加する様子を\figref{skiplist-lengths-splice}に示す。
$L_0$において#u#が何番めなのかは、探索経路を辿りながら数えられる。
そのため、#u#から#w#までの辺の長さは$#i#-#j#$とわかる。
さらに、#u#から#z#への辺の長さ$\ell$から、#w#から#z#への辺の長さを計算できる。
こうして#w#を挿入し、関連する辺の長さの更新を定数時間で終えることができる。

\begin{figure}
  \begin{center}
    \includegraphics[scale=0.90909]{figs/skiplist-lengths-splice}
  \end{center}
  \caption{ノード#w#をskiplistに追加するときの、辺の長さの更新}
  \figlabel{skiplist-lengths-splice}
\end{figure}

複雑そうに聞こえるかもしれないが、実際のコードはとても単純である。

\codeimport{ods/SkiplistList.add(i,x)}
\codeimport{ods/SkiplistList.add(i,w)}

ここまでの話から、#SkiplistList#における#remove(i)#の実装は明らかである。
#i#番めの位置への探索経路を辿る。
高さ#r#のノード#u#から経路が下に向かうとき、その高さにおける#u#から出る辺の長さを1つ小さくする。
また、#u.next[r]#が高さ#i#の要素であるかどうかを確認し、もしそうならリストから取り除く
(\figref{skiplist-removei})。

\begin{figure}
  \begin{center}
    \includegraphics[width=\ScaleIfNeeded]{figs/skiplist-removei}
  \end{center}
  \caption{#SkiplistList#から要素を削除する}
  \figlabel{skiplist-removei}
\end{figure}
\codeimport{ods/SkiplistList.remove(i)}

\subsection{要約}

\thmref{skiplistlist}は、#SkiplistList#の性能を要約したものだ。

\begin{thm}\thmlabel{skiplistlist}
  #SkiplistList#は#List#インターフェースを実装する。
  #SkiplistList#における#get(i)#、#set(i,x)#、#add(i,x)#、#remove(i)#の実行時間の期待値は、いずれも$O(\log #n#)$である。
\end{thm}

\section{スキップリストの解析}
\seclabel{skiplist-analysis}

この節では、スキップリストの高さ、大きさ、探索経路の長さの期待値を解析する。
基本的な確率論の知識を前提とする。
いくつかの証明では、次に述べるコイントスについての考察を利用する。

\begin{lem}\lemlabel{coin-tosses}
  \ejindex{coin toss}{こいんなげ@コイン投げ}%
  $T$を、表裏が等しい確率で出るコインを投げて、表が出るまでに要するコイントスの回数とする(表が出た回数も含まれる点に注意)。このとき、$\E[T]=2$である。
\end{lem}

\begin{proof}
表が出るまでコイントスを繰り返すとき、次のインジケータ確率変数を定義する。
  \[ I_{i} = \left\{\begin{array}{ll}
     0 & \mbox{コイントスが$i$回よりも少ないとき} \\
     1 & \mbox{コイントスが$i$回以上のとき}
     \end{array}\right. % "\right."はドットまで含めてlatexの文法なので消さないで
  \]
  $I_i=1$は、最初の$i-1$回の結果がいずれも裏であることと同値である。
  よって、$\E[I_i]=\Pr\{I_i=1\}=1/2^{i-1}$である。
  コイントスの合計回数$T$は、$T=\sum_{i=1}^{\infty} I_i$と書ける。
  以上より、次のことがわかる。
  \begin{align*}
    \E[T] & =  \E\left[\sum_{i=1}^\infty I_i\right] \\
     & =  \sum_{i=1}^\infty \E\left[I_i\right] \\
     & =  \sum_{i=1}^\infty 1/2^{i-1} \\
     & =  1 + 1/2 + 1/4 + 1/8 + \cdots \\
     & =  2 \enspace \qedhere
  \end{align*}
\end{proof}

次の2つの補題では、スキップリストにおけるノード数が要素数に概ね比例することを示す。

\begin{lem}\lemlabel{skiplist-size1}
  $#n#$要素からなるスキップリストにおける(番兵を除く)ノード数の期待値は、$2#n#$である。
\end{lem}

\begin{proof}
  要素#x#がリスト$L_{#r#}$に含まれる確率は$1/2^{#r#}$である。
  よって、$L_{#r#}$のノード数の期待値は$#n#/2^{#r#}$である
  \footnote{インジケータ確率変数と期待値の線形性からこの結果を得る方法は\secref{randomization}を参照せよ。}。
  以上より、すべてのリストに含まれるノードの総数の期待値が求まる。
  \[ \sum_{#r#=0}^\infty #n#/2^{#r#} = #n#(1+1/2+1/4+1/8+\cdots) = 2#n# \enspace \qedhere \]
\end{proof}

\begin{lem}\lemlabel{skiplist-height}
  #n#要素を含むスキップリストの高さの期待値は$\log #n# + 2$以下である。
\end{lem}

\begin{proof}
  $#r#\in\{1,2,3,\ldots,\infty\}$について次の確率変数を定義する。
  \[ I_{#r#} = \left\{\begin{array}{ll}
     0 & \mbox{$L_{#r#}$が空のとき} \\
     1 & \mbox{$L_{#r#}$が空でないとき}
     \end{array}\right.
  \]
  スキップリストの高さ#h#は次のように計算できる。
  \[
       #h# = \sum_{r=1}^\infty I_{#r#}
  \]
  $I_{#r#}$はリスト$L_{#r#}$の長さ$|L_{#r#}|$を越えないことに注意する。
  \[
     \E[I_{#r#}] \le \E[|L_{#r#}|] = #n#/2^{#r#} \enspace
  \]
  以上より、次のことがわかる。
  \begin{align*}
       \E[#h#] &= \E\left[\sum_{r=1}^\infty I_{#r#}\right] \\
        &= \sum_{#r#=1}^{\infty} E[I_{#r#}] \\
        &= \sum_{#r#=1}^{\lfloor\log #n#\rfloor} E[I_{#r#}]
                 + \sum_{r=\lfloor\log #n#\rfloor+1}^{\infty} E[I_{#r#}]  \\
        &\le \sum_{#r#=1}^{\lfloor\log #n#\rfloor} 1
                 + \sum_{r=\lfloor\log #n#\rfloor+1}^{\infty} #n#/2^{#r#} \\
        &\le \log #n#
                 + \sum_{#r#=0}^\infty 1/2^{#r#} \\
        &= \log #n# + 2 \enspace \qedhere
  \end{align*}
\end{proof}

\begin{lem}\lemlabel{skiplist-size2}
  $#n#$要素からなるスキップリストのノード数の期待値は、番兵を含めて$2#n#+O(\log #n#)$である。
\end{lem}

\begin{proof}
  \lemref{skiplist-size1}より、番兵を除いたノード数の期待値は$2#n#$である。
  番兵の数の期待値はスキップリストの高さ$#h#$に等しく、
  これは\lemref{skiplist-height}より$\log #n#+2 = O(\log #n#)$以下である。
  \end{proof}



\begin{lem}
スキップリストにおける探索経路の長さの期待値は$2\log #n# + O(1)$以下である。
\end{lem}

\begin{proof}
最も簡単な証明方法は、ノード#x#の\emph{逆探索経路}を考えることだ。
この逆探索経路は、$L_0$における#x#の直前のノードから始まる。
パスが上に向かえるときは上に向かう。
そうでないなら左に進む。
少し考えると、#x#の逆探索経路は、方向が逆であることを除いて探索経路と同じになることがわかるだろう。

ある高さで逆探索経路が通過するノードの数#r#は、
コインを投げて表が出れば上に向かってから停止し、
裏が出れば左に向かい試行を続ける、という試行に関連している。
すなわち、表が出るまでにコインを投げる回数は、逆探索経路において、ある高さで左に向かうステップの数に対応している
\footnote{実際には、最初に表が出る、もしくは探索経路で番兵に出くわす場合に試行が終了するはずなので、左に向かう回数を多く数えてしまう可能性がある。しかし、いま考えている補題は上界に関するものなので問題はない。}。
\lemref{coin-tosses}より、初めて表が出るまでのコイントスの回数の期待値は1である。

(順方向の)探索経路において、高さ$#r#$で右に進む回数を$S_{#r#}$で表す。
すると、$\E[S_{#r#}]\le 1$である。
さらに、$L_{#r#}$では$L_{#r#}$の長さよりも多く右に進むことはないので、$S_{#r#}\le |L_{#r#}|$である。
よって次の式が成り立つ。
  \[
    \E[S_{#r#}] \le \E[|L_{#r#}|] = #n#/2^{#r#} \enspace
  \]
あとは\lemref{skiplist-height}と同様に証明を完成すればよい。
$S$をスキップリストにおけるノード#u#の探索経路の長さとする。
また、#h#をそのスキップリストの高さとする。
このとき、次の式が成り立つ。
  \begin{align*}
      \E[S] 
         &= \E\left[ #h# + \sum_{#r#=0}^\infty S_{#r#} \right] \\
         &= \E[#h#] + \sum_{#r#=0}^\infty \E[S_{#r#}]  \\
         &= \E[#h#] + \sum_{#r#=0}^{\lfloor\log #n#\rfloor} \E[S_{#r#}] 
              + \sum_{#r#=\lfloor\log #n#\rfloor+1}^\infty \E[S_{#r#}] \\
         &\le \E[#h#] + \sum_{#r#=0}^{\lfloor\log #n#\rfloor} 1
              + \sum_{r=\lfloor\log #n#\rfloor+1}^\infty #n#/2^{#r#} \\
         &\le \E[#h#] + \sum_{#r#=0}^{\lfloor\log #n#\rfloor} 1
              + \sum_{#r#=0}^{\infty} 1/2^{#r#} \\
         &\le \E[#h#] + \sum_{#r#=0}^{\lfloor\log #n#\rfloor} 1
              + \sum_{#r#=0}^{\infty} 1/2^{#r#} \\
         &\le \E[#h#] + \log #n# + 3 \\
         &\le 2\log #n# + 5  \enspace \qedhere
  \end{align*}
\end{proof}


次の定理にこの節の結果をまとめる。
\begin{thm}
$#n#$要素を含むスキップリストの大きさの期待値は$O(#n#)$である。
ある要素の探索経路の長さの期待値は$2\log #n# + O(1)$以下である。
\end{thm}

\section{ディスカッションと練習問題}

スキップリストを導入したのはPugh \cite{p91}である。Pughはスキップリストの拡張や応用も数多く提案した\cite{p89}。
それ以来、スキップリストについては広く研究されている。
スキップリストの#i#番めの要素を見つける探索経路の長さの期待値や分散については、複数の研究\cite{kp94,kmp95,pmp92}でさらに正確に解析されている。
決定的な(ランダム性を用いない)変種\cite{mps92}、偏りのある変種\cite{bbg02,esss01}、適応的な変種\cite{bdl08}も開発されている。
スキップリストはさまざまな言語やフレームワークで書かれ、またオープンソースのデータベースシステムでも使われている\cite{skipdb,redis}。
オペレーティングシステムHP-UXにおけるカーネルのプロセス制御の構造として、スキップリストの変種が使われている\cite{hpux}。
\javaonly{Skiplistsは、Java 1.6 APIの一部として含まれている\cite{oracle_jdk6}。}

\begin{exc}
\figref{skiplist}のスキップリストにおける2.5と5.5の探索経路を説明せよ。
\end{exc}

\begin{exc}
\figref{skiplist}のスキップリストに対して値0.5の要素を高さ1で追加し、その後、値3.5の要素を高さ2で追加するときの振る舞いを説明せよ。
\end{exc}

\begin{exc}
\figref{skiplist}のスキップリストから1と3を削除するときの振る舞いを説明せよ。
\end{exc}

\begin{exc}
\figref{skiplist-lengths}の#SkiplistList#に#remove(2)#を実行するときの振る舞いを説明せよ。
\end{exc}

\begin{exc}
\figref{skiplist-lengths}の#SkiplistList#に#add(3,x)#を実行するときの振る舞いを説明せよ。
なお、#pickHeight()#は新たなノードの高さとして4を選択すると仮定せよ。
\end{exc}

\begin{exc}\exclabel{skiplist-changes}
#add(x)#または#remove(x)#を実行するとき、#SkiplistSet#のポインタのうち操作されるものの数の期待値は定数であることを示せ。
\end{exc}

\begin{exc}\exclabel{skiplist-opt}
ある要素を$L_{i-1}$から$L_i$へ昇格させるときにコイントスを使わず、確率$p (0 < p < 1)$を利用するとする。
  \begin{enumerate}
   \item このとき、探索経路の長さの期待値は$(1/p)\log_{1/p} #n# + O(1)$以下であることを示せ。
   \item これを最小にする$p$を求めよ。
   \item スキップリストの高さの期待値を求めよ。
   \item スキップリストのノード数の期待値を求めよ。
  \end{enumerate}
\end{exc}


\begin{exc}\exclabel{skiplist-opt-2}
  #SkiplistSet#の#find(x)#では\emph{冗長な比較}を行うことがある。
  具体的には、#x#と同じ値を複数回にわたって比較することがある。
  このような冗長な比較は、$#u.next[r]# = #u.next[r-1]#$を満たすノード#u#が存在すると発生する。
  どのようにして冗長な比較が発生するかを説明し、#find(x)#において冗長な比較が発生しないようにする方法を示せ。
  そして、このように修正した#find(x)#メソッドによる比較の回数を解析せよ。
\end{exc}

\begin{exc}
#SSet#の要素であって#x#より小さいものの個数を#SSet#における要素#x#のランクと呼ぶ。
#SSet#インターフェースを実装するスキップリストを、ランクによる要素への高速アクセスが可能になるように設計、実装せよ。
ランク#i#の要素を返す#get(i)#も実装せよ。#get(i)#操作の実行時間は$O(\log #n#)$とすること。
\end{exc}

\begin{exc}
  \ejindex{finger}{ゆび@指}%
  \ejindex{finger search!in a skiplist}{ふぃんがーさーち@フィンガーサーチ!すきっぷりすと@スキップリスト}%

% 指 or 検索指
探索経路において下に向かうノードを保存した配列のことを、スキップリストの\emph{指}と呼ぶ
(\pageref{pg:skiplist-add}ページのコードで、#add(x)#における変数#stack#は\emph{指}である。\figref{skiplist-add}の網掛のノードが指を表している)。
指は、$L_0$における経路を示していると解釈することもできる。

\emph{指探索}は、指を利用した#find(x)#の実装である。
$#u.x# < #x#$かつ$(#u.next#=#null# or #u.next.x# > #x#)$を満たすノード#u#に到達するまでは指の要素を見ていき、#u#からふつうの#x#の探索を実行する。
$L_0$において#b#から指が指示する値までの間にある値の数を$r$とするとき、\emph{指探索}のステップ数の期待値は$O(1+\log r)$である。

内部の指を利用して#find(x)#を実装した#Skiplist#のサブクラス、#SkiplistWithFinger#を実装せよ。
このサブクラスでは指を保持し、#find(x)#を指探索として実装するためにその指を使う。
#find(x)#では、#x#の位置を返すと同時に、指が常に前回の#find(x)#の結果を指示するように更新する。
\end{exc}

\begin{exc}\exclabel{skiplist-truncate}
#SkiplistList#を#i#番めの位置で切り詰める#truncate(i)#メソッドを実装せよ。
このメソッドを実行すると、リストの大きさは#i#になり、リストは添字$0,\ldots,#i#-1$の要素のみを含むようになる。
このメソッドの返り値は、添字$#i#,\ldots,#n#-1$の要素を含む#SkiplistList#である。
このメソッドの実行時間は$O(\log #n#)$でなければならない。
\end{exc}

\begin{exc}
#SkiplistList#を引数に取り、引数の#SkiplistList#を空にして、その要素をレシーバーにそのままの順番で追加するメソッド#absorb(skiplistlist)#を実装せよ。
例えば、スキップリスト#l1#の要素が$a,b,c$、スキップリスト#l2#の要素が$d,e,f$であるとき、#l1.absorb(l2)#を呼ぶと、#l1#の要素は$a,b,c,d,e,f$になり、#l2#は空になる。
このメソッドの実行時間は$O(\log #n#)$でなければならない。
\end{exc}

\begin{exc}
#SEList#のアイデアを転用し、空間効率の良い#SSet#である#SESSet#を設計、実装せよ。
要素を順に#SEList#に格納し、この#SEList#のブロックを#SSet#に格納すればよい。
オリジナルの#SSet#の実装で#n#要素の保持に$O(#n#)$のメモリを使うとしたら、#SESSet#では#n#要素を格納するためのメモリに加えて$O(#n#/#b#+#b#)$だけの余分な空間を使うことになるだろう。
\end{exc}

\begin{exc}
#SSet#を使って(大きな)テキストを読み込み、そのテキストの任意の部分文字列をインタラクティブに検索できるアプリケーションを設計、実装せよ。
ユーザーがクエリを入力したら、テキストのマッチしている部分があればそれを結果として返すこと。

  \noindent ヒント1:任意の部分文字列は、ある接尾辞の接頭辞である。よって、テキストファイルのすべての接尾辞を保存すれば十分である。

  \noindent ヒント2:任意の接尾辞は、テキストの中のどこから接尾辞が始まるかを表す1つの整数で簡潔に表現できる。

  \noindent 書いたアプリケーションを長いテキストでテストせよ。
  プロジェクトGutenberg \cite{gutenberg}から書籍のテキストを入手できる。
  正しく動いたらレスポンスを速くしよう。すなわち、キー入力から結果が得られるまでに要する時間を認識できないくらい小さくしよう。
\end{exc}

\begin{exc}
  \ejindex{skiplist!versus binary search tree}{すきっぷりすと@スキップリスト!にぶんたんさくぎとのひかく@二分探索木との比較}%
  \ejindex{binary search tree!versus skiplist}{にぶんたんさくぎ@二分探索木!すきっぷりすととのひかく@スキップリストとの比較}%
  (この練習問題は、\secref{binarysearchtree}で二分探索木について学んでから取り組むこと。)

  スキップリストを二分探索木と比較せよ。
  \begin{enumerate}
     \item スキップリストから辺をいくつか削除すると二分木のような構造が得られること、および、二分探索木に似ることを説明せよ。
     \item スキップリストと二分探索木とでは、使うポインタの数はだいたい同じである(ノードあたり2つ)。
	 しかし、スキップリストのほうが使い方がうまい。
	 その理由を説明せよ。
  \end{enumerate}
\end{exc}
