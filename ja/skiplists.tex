\chapter{スキップリスト}
\chaplabel{skiplists}

この章ではスキップリストという面白くて実際の応用も多いデータ構造を紹介する。
スキップリストは#get(i)#・#set(i,x)#・#add(i,x)#・#remove(i)#をいずれも$O(\log n)$の時間で実行できる#List#の実装である。
#SSet#の実装でもあり、すべての操作の期待実行時間は$O(\log #n#)$である。

スキップリストの効率性のキモはランダム性である。
新しい要素を追加するとき、スキップリストではランダムにコインを投げて要素の高さを決める。
スキップリストの性能は期待実行時間とパス長を使って表現できる。
コイントスの結果に応じて決まる確率からこの期待値は計算される。
ランダムなコイントスは擬似乱数(あるいはランダムビット)生成器によるシミュレーションで実装される。
