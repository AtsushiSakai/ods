\chapter{二分木}
\chaplabel{binarytrees}

この章では、コンピュータサイエンスで現れる最も基本的な構造のうちのひとつ、二分木を紹介する。
この構造を\emph{木(tree)}と呼ぶのは、図示したときに(森に生えている)木に似ているためである。
\ejindex{tree}{き@木}%
\ejindex{tree!binary}{き@木!にぶん@二分}%
\ejindex{binary tree}{にぶんぎ@二分木}%
二分木には複数の定義がある。
数学的な\emph{二分木(binary tree)}の定義は、連結(connected)
\footnote{訳注:グラフが連結であるとは、グラフ上の辺を辿ることで任意の2頂点間を行き来できることである。そうして辿った辺の列のことを経路(path)と呼ぶ。}な有限無向グラフであって、閉路(cycle)\footnote{(単純)閉路とは、ある頂点から同じ頂点および同じ辺を通らずにその頂点に戻る経路である。}を持たず、すべての頂点の次数(degree)が3以下のものである
\footnote{訳注:無向グラフの頂点における次数とは、その頂点が持つ辺の数である。例えば\figref{bintree-orientation}において、#u#の次数は3である。}。

コンピュータサイエンスにおける応用では、二分木はふつう\emph{根を持つ(rooted)}。
\ejindex{tree!rooted}{き@木!ねつき@根付き}%
\ejindex{rooted tree}{ねつきぎ@根付き木}%
木の\emph{根(root)}と呼ばれる特殊なノード#r#があり、このノードの次数は2以下である。
%次数2以下のノードのうち、特別なもの#r#を、木の\emph{根}と呼ぶ。
ノード$#u#(\neq #r#)$から#r#に向かう経路における2番めのノードを#u#の\emph{親(parent)}という。
\ejindex{parent}{おや@親}%
#u#に隣接する親以外のノードを#u#の\emph{子(child)}という。
特に\emph{順序付けられている(ordered)}二分木に興味があることが多いので、
\ejindex{ordered tree}{じゅんじょづけられたき@順序付けられた木}%
\ejindex{tree!ordered}{き@木!じゅんじょづけられた@順序付けられた}%
子を\emph{左の子}、\emph{右の子}と呼び分けることにする。
\ejindex{left child}{ひだりのこ@左の子}%
\ejindex{child!left}{こ@子!ひだり@左}%
\ejindex{right child}{みぎのこ@右の子}%
\ejindex{child!right}{こ@子!みぎ@右}%

二分木を図示するとき、ふつうは根を一番上に書く。
また、ノード#u#の左右の子は、#u#の左下と右下にそれぞれ描く
(\figref{bintree-orientation})。
\figref{binary-tree}の(a)に、9個のノードを持つ二分木の例を示す。

\begin{figure}
  \begin{center}
    \includegraphics[scale=0.90909]{figs/bintree-traverse-1}
  \end{center}
  \caption{#BinaryTree#における、ノード#u#の親、左の子、右の子}
  \figlabel{bintree-orientation}
\end{figure}


\begin{figure}
  \begin{center}
    \begin{tabular}{cc}
      \includegraphics[width=\HalfScaleIfNeeded]{figs/bintree-1} &
      \includegraphics[width=\HalfScaleIfNeeded]{figs/bintree-2} \\
      (a) & (b)
    \end{tabular}
  \end{center}
  \caption{(a)~9個の本物のノードを持つ二分木と、(b)~10個の外部ノードを持つ二分木}
  \figlabel{binary-tree}
\end{figure}

% XXX: 別に二分木専用の用語ではないような気が...? % YJ 木に訂正しましょう
木(および二分木)は重要なので、その性質を記述するための専用の語彙が使われている。
木におけるノード#u#の\emph{深さ(depth)}とは、
\ejindex{depth}{ふかさ@深さ}%
#u#から根までの経路の長さである
\footnote{訳注:例えば#u#=#r#であるとき、#u#の深さは0である。}。
ノード#w#が#u#から#r#への経路に含まれるとき、#w#を#u#の\emph{祖先(ancestor)}という。
\ejindex{ancestor}{そせん@祖先}%
一方、このとき#u#を#w#の\emph{子孫(descendant)}という。
\ejindex{descendant}{しそん@子孫}%
木におけるノード#u#の\emph{部分木(subtree)}とは、#u#を根とし、#u#のすべての子孫を含む木である。
ノード#u#の\emph{高さ(height)}とは、#u#から#u#の子孫への経路の長さの最大値である。
\ejindex{height!in a tree}{たかさ@高さ!き@木}
木の\emph{高さ}とは、その根の高さである。
ノード#u#が子を持たない場合、#u#は\emph{葉(leaf)}という。
\ejindex{leaf}{は@葉}%

木を考えるとき、\emph{外部ノード(external node)}で拡張すると便利なことがある。
左の子を持たないノードであれば、左の子として外部ノードを持つ。同様に、右の子を持たないノードであれば、右の子として外部ノードを持つ(\figref{binary-tree}(b)を参照)。
帰納法により、$#n#\ge 1$個の(本物の)ノードを持つ二分木は、$#n#+1$個の外部ノードを持つことが示せる。
% NOTE: n = 0 の場合はどう扱うのが自然なのだろう? % Empty graphは木でなくForestとして扱うべきという話があるらしい https://link.springer.com/chapter/10.1007%2FBFb0066433

\section{#BinaryTree#:基本的な二分木}

\index{BinaryTree@#BinaryTree#}%
二分木におけるノード#u#を簡単に表現するには、#u#に隣接するノードを明示的に保持すればよい。
\javaimport{ods/BinaryTree.BTNode<Node}
\cppimport{ods/BinaryTree.BTNode}
隣接する頂点は最大で3つあるが、そのうち存在しないものを指す変数は#nil#とする。
すると、外部ノードも根の親も#nil#に対応する。

このように二分木を表現すると、二分木自体は根#r#への参照として表現できる
\footnote{訳注:本章の冒頭で、二分木を連結なグラフとして定義したことを思い出そう。連結なグラフなので、木に含まれるいずれかのノードへの参照が1つあれば、そこから他のノードへ辿り着ける。そのような参照として根#r#を選べば、その二分木を示せるというわけである。なお、もし根以外の参照を選ぶと、後の章で出てくるB木(\secref{btree-elem-add}参照)のような木では親への参照がないので、複数のノードが必要になる。}。
%すると、根#r#への参照として二分木自体は表現できる。 % YJ こういう細かい語順を変えるだけでかなり分かりやすくなると思うので、細かいと思いますが直します % +1 さらに少し補足。-kshikano
\codeimport{ods/BinaryTree.r}

ノード#u#の深さは、#u#から根への経路を辿るときのステップ数として計算できる。
% XXX: If node == nil, should depth(u) return -1 as height(u) do?
\codeimport{ods/BinaryTree.depth(u)}


\subsection{再帰的なアルゴリズム}

\ejindex{recursive algorithm}{さいきあるごりずむ@再帰アルゴリズム}%
再帰的なアルゴリズムを使うと二分木に関する計算が簡単になる。
例えば、#u#を根とする二分木のサイズ(ノードの数)は、#u#の子を根とする部分木のサイズを再帰的に計算し、足し合わせ、その結果に1加えると求まる。

\codeimport{ods/BinaryTree.size(u)}

ノード#u#の高さは、#u#の2つの部分木の高さの最大値を計算し、その結果に1加えると求まる。
%\footnote{訳注:完全な余談だが、このように二分木の高さを求めるのは、米国におけるソフトウェアエンジニア面接の頻出問題の一つである。https://leetcode.com/problems/maximum-depth-of-binary-tree/}。% さすがにちょっと余談感がすぎるのでコメントに -kshikano

\codeimport{ods/BinaryTree.height(u)}

\subsection{二分木の走査}
\seclabel{bintree:traversal}

\ejindex{traversal!of a binary tree}{そうさ@走査!にぶんぎ@二分木}%
\ejindex{tree traversal}{きのそうさ@木の走査}%
\ejindex{binary-tree traversal}{にぶんぎのそうさ@二分木の走査}%
前項で説明した2つのアルゴリズムでは、二分木のすべてのノードを訪問するために再帰を使った。
いずれのアルゴリズムも、二分木のノードを次のコードと同じ順番で訪問していた。
% NOTE: (C++では?)部分表現の評価順序は未規定で、訪問順は必ずしもこうなるわけではないのではないか % TODO: YJ C, C++の部分評価順序は規定されていないのでこれは正しくない。著者に言うべきか。
% spinute: ここで重箱の隅をつついても読者の理解の助けにならなそうなので、本文には入れないことにした
\codeimport{ods/BinaryTree.traverse(u)}
% \footnote{\cpponly{訳注:C++ 規格では部分表現の評価順序は未規定なので、同じ順番でノードが訪問されるとは限らないように思う。}}

再帰を使うとこのように簡潔なコードを書けるが、時に困ることもある。
再帰の深さの最大値は、二分木におけるノードの深さの最大値、すなわち木の高さである。
これが非常に大きいと、再帰のためのスタックとして利用できる量以上の領域を要求し、プログラムがクラッシュしてしまうことがある
\footnote{訳注:この問題はスタックオーバーフローと呼ばれる。}。

再帰なしで二分木を走査するためには、どこから来たかによって次の行き先を決めるアルゴリズムを使えばよい
(\figref{bintree-traverse})。
ノード#u#に#u.parent#から来たときは、次は#u.left#に向かう。
#u.left#から来たときは、次は#u.right#に向かう。
#u.right#から来たときは、#u#の部分木を辿り終えたので#u.parent#に戻る。
次のコードはこれを実装したものである。
#u.left#、#u.right#、#u.parent#が#nil#であるケースも適切に処理している。
\codeimport{ods/BinaryTree.traverse2()}

\begin{figure}
  \begin{center}
    \begin{tabular}{cc}
      \includegraphics[scale=0.90909]{figs/bintree-traverse-2}
      \includegraphics[scale=0.90909]{figs/bintree-3}
    \end{tabular}
  \end{center}
  \caption{再帰を使わずに二分木を走査してノード#u#を訪れる3通りの方法と、そのときの木の走査}
  \figlabel{bintree-traverse}
\end{figure}

再帰アルゴリズムで計算できることは、こうして再帰なしでも計算できる。
例えば木のサイズを計算するためには、カウンタ#n#を保持し、新しいノードを訪問するたびにその値を1ずつ増やせばよい。
\codeimport{ods/BinaryTree.size2()}

二分木の実装では#parent#を使わないこともある。
この場合にも再帰を使わない実装は可能だが、いま訪問しているノードから根までの経路を#List#か#Stack#を使って記録しておく必要がある。

ここまでの方法とは別の走査方法として、\emph{幅優先(breadth-first)}な走査がある
\footnote{訳注:\secref{general-bfs}では、木の一般化であるグラフにおける幅優先探索アルゴリズムを紹介する。}。
\ejindex{breadth-first traversal}{はばゆうせんそうさ@幅優先走査}%
\ejindex{traversal!breadth-first}{そうさ@走査!はばゆうせん@幅優先}%
幅優先に走査する場合、根から下に向かって深さごとにすべてのノードを訪問する。同じ深さのノードは左から右の順に訪問する
(\figref{bintree-bfs}を参照せよ)。
これは英語の文章の読み方と似ている。%(訳注:横書きなら、日本語でも同様である。) % この文を読んでいる人なら知っているので
幅優先の走査はキュー#q#を使って実装できる。
初期状態では#q#は根だけを含む。
各ステップでは、#q#から次のノード#u#を取り出し、#u#を処理し、#u.left#と#u.right#を(#nil#でなければ)#q#に追加する。
\codeimport{ods/BinaryTree.bfTraverse()}

\begin{figure}
  \begin{center}
    \includegraphics[scale=0.90909]{figs/bintree-4}
  \end{center}
  \caption{幅優先な走査では、二分木の各ノードを深さごとに訪問する。各深さでは左から右の順で訪問する}
  \figlabel{bintree-bfs}
\end{figure}

\section{#BinarySearchTree#:バランスされていない二分探索木}
\seclabel{binarysearchtree}

\index{BinarySearchTree@#BinarySearchTree#}%
\ejindex{binary search tree}{にぶんたんさくぎ@二分探索木}%
\ejindex{binary tree!search}{にぶんぎ@二分木!たんさく@探索}%
ノード#u#が、ある全順序な集合の要素#x#をデータ#u.x#として持つような特別な二分木、#BinarySearchTree#を考える。
各ノードとそのデータは、次の\emph{二分探索木の性質}を満たすとする。% TODO: YJ これはpropertyなのかdefinitionなのか -- 二分探索木の性質を満たす二分木を二分探索木探索木という(定義)んじゃないかな思うけど、二分探索木のことはこれまであまり調べたことがないので自信はない / 英文的には、「このBinarySearchTreeというのはこういう性質を持っているbinary search treeのひとつのインスタンス」と言いたいだけな感じなので、そのような文章にしました -kshikano
\ejindex{binary search tree property}{にぶんたんさくぎのせいしつ@2分探索木の性質}%
すなわち、ノード#u#について、#u.left#を根とする部分木に含まれるデータはすべて#u.x#より小さく、#u.right#を根とする部分木に含まれるデータはすべて#u.x#より大きい。
このような#BinarySearchTree#の例を\figref{bst}に示す。

\begin{figure}
  \begin{center}
    \includegraphics[scale=0.90909]{figs/bst-example}
    %\includegraphics[scale=0.90909]{figs/binary-tree-4}
  \end{center}
  \caption{二分探索木の例}
  \figlabel{bst}
\end{figure}


\subsection{探索}

\ejindex{search path!in a binary search tree}{たんさくけいろ@探索経路!にぶんたんさくぎ@二分探索木}%
二分探索木の性質はとても有用だ。
この性質を利用して、二分探索木から値#x#を高速に見つけられる。
具体的には、まず根#r#から#x#を探し始める。
ノード#u#を訪問しているとき、次の3つの場合がありうる。
\begin{enumerate}
\item $#x#< #u.x#$なら#u.left#に進む
\item $#x#> #u.x#$なら#u.right#に進む
\item $#x#= #u.x#$なら値が#x#であるノード#u#を見つけた
\end{enumerate}
この探索は3つめのケース、または#u=nil#になると終了する。
前者なら#x#が見つかったことになる。
後者なら#x#がこの木に含まれていないとわかる。
\codeimport{ods/BinarySearchTree.findEQ(x)}

二分探索木における探索の例を\figref{bst-search}に2つ示す。
2つめの例から、#x#が見つからない場合でも、役に立つ情報が得られることがわかる。
探索における最後のノード#u#にて、先の場合分けの1つめのケースであったなら、#u.x#は木に含まれるデータであって、#x#よりも大きい値のうちで最小のものである。
同様に、場合分けの2つめのケースであったなら、#u.x#は#x#より小さい値のうちで最大のものである。
よって、場合分けの1つめのケースが最後に発生したノード#z#を記録しておけば、#x#以上の値のうちで最小のものを返すように#BinarySearchTree#の#find(x)#を実装できる。
\codeimport{ods/BinarySearchTree.find(x)}

\begin{figure}
  \begin{center}
    \begin{tabular}{cc}
    \includegraphics[width=\HalfScaleIfNeeded]{figs/bst-example-2} &
    \includegraphics[width=\HalfScaleIfNeeded]{figs/bst-example-3} \\
    (a) & (b)
    \end{tabular}
  \end{center}
  \caption{二分探索木において、(a)~探索が成功する例($6$が見つかる)と、(b)~探索が失敗する例($10$が見つからない)}
  \figlabel{bst-search}
\end{figure}


\subsection{追加}

#BinarySearchTree#に値#x#を追加するには、まず#x#を検索する。
もし見つかれば挿入の必要がない。
見つからなければ、検索において最後に出会ったノード#p#の子である葉として、#x#を保存する。
このとき、新しいノードが#p#の右の子か左の子かを、#x#と#p.x#の比較結果によって決める。
\codeimport{ods/BinarySearchTree.add(x)}
\codeimport{ods/BinarySearchTree.findLast(x)}
\codeimport{ods/BinarySearchTree.addChild(p,u)}
\figref{bst-insert}に例を示す。
% XXX: 原著が間違ってる? According to the definition at the head of this chapter, it seems to be appropriate to use not "height" but "depth" here. -- YJ 深さかと
% > An example is shown in \figref{bst-insert}. The most time-consuming
% part of this process is the initial search for #x#, which takes an
% amount of time proportional to the height of the newly added node #u#.
最も時間がかかるのは#x#を検索する処理で、この時間は新たに追加するノード#u#の深さに比例する。
これは、最悪の場合、#BinarySearchTree#の高さである。

\begin{figure}
  \begin{center}
    \begin{tabular}{cc}
    \includegraphics[width=\HalfScaleIfNeeded]{figs/bst-example-4} &
    \includegraphics[width=\HalfScaleIfNeeded]{figs/bst-example-5}
    \end{tabular}
  \end{center}
  \caption{二分探索木に$8.5$を追加}
  \figlabel{bst-insert}
\end{figure}


\subsection{削除}

ある値を格納するノード#u#を#BinarySearchTree#から削除する処理はもう少し複雑だ。
#u#が葉なら、#u#を単に親から切り離せばよい。
#u#が子を1つだけ持っているなら、#u#で両端を継ぎ合わせる、すなわち、#u.parent#と#u#の子とを親子関係にすればよい
(\figref{bst-splice}を参照)。
\codeimport{ods/BinarySearchTree.splice(u)}

\begin{figure}
  \begin{center}
    \includegraphics[scale=0.90909]{figs/bst-splice}
  \end{center}
  \caption{葉($6$)、または子を1つだけ持つノード($9$)の削除は簡単}
  \figlabel{bst-splice}
\end{figure}

#u#が子を2つ持っている場合は、もっと手の込んだ操作が必要になる。
この場合、子の数が1以下のノード#w#で、#w.x#と#u.x#とを入れ替えられるようなものを見つけるのが最も単純だ。
二分探索木の性質を保つためには、#w.x#の値と#u.x#の値が近ければよい。
例えば、#w.x#が、#u.x#より大きい中で最小の値であればよい。
このような#w#は簡単に見つけられる。
これは#u.right#を根とする部分木の中で最小の値である。
このノードは左の子を持たないので、取り除くのも簡単である
(\figref{bst-remove}を参照)。
\javaimport{ods/BinarySearchTree.remove(u)}
\cppimport{ods/BinarySearchTree.remove(u)}
\pcodeimport{ods/BinarySearchTree.remove_node(u)}

\begin{figure}
  \begin{center}
    \begin{tabular}{cc}
    \includegraphics[width=\HalfScaleIfNeeded]{figs/bst-delete-1}
    \includegraphics[width=\HalfScaleIfNeeded]{figs/bst-delete-2}
    \end{tabular}
  \end{center}
  \caption{2つの子を持つノード#u#から値$11$を削除するために、#u#の値と、#u#の右の部分木における最小の値とを入れ替える}
  \figlabel{bst-remove}
\end{figure}

\subsection{要約}

#BinarySearchTree#における#find(x)#、#add(x)#、#remove(x)#の処理は、いずれも根から特定のノードへの経路を辿る処理を伴う。
木の形状について何らかの仮定をしない限り、この経路の長さについて、「木の中のノード数を超えない」というより強い主張をするのは難しい。
次の定理は、あまり面白いものではないが、#BinarySearchTree#の性能をまとめたものだ。

\begin{thm}\thmlabel{bst}
  #BinarySearchTree#は#SSet#インターフェースの実装であって、#add(x)#、#remove(x)#、#find(x)#の実行時間は$O(#n#)$である。
\end{thm}

\thmref{skiplist}と比べると、\thmref{bst}の性能は良くない。
#SkiplistSSet#では、各操作の期待実行時間が$O(\log #n#)$であるように#SSet#インターフェースを実装できた。
#BinarySearchTree#には、木の形状が\emph{アンバランス(unbalanced)}かもしれないという問題がある。
\figref{bst}のような木の形ではなく、ほとんどのノードが子を1つだけ持ち、#n#個のノードからなる長い鎖のような見た目かもしれないのである
\footnote{訳注:このような長い鎖のような見た目をしたデータ構造には見覚えがあるかもしれない。\chapref{linkedlists}で扱った連結リストを思い出そう(細かな相違はある)。}。

二分探索木がアンバランスになることを回避する方法はたくさんある。そのような方法を採用すれば、$O(\log #n#)$の時間で各操作を行えるようになる。
\chapref{rbs}では、ランダム性を利用することで、$O(\log #n#)$の\emph{期待実行時間}を達成する方法を説明する。
\chapref{scapegoat}では、部分的な再構築を利用することで、$O(\log #n#)$の\emph{償却実行時間}を達成する方法を説明する。
\chapref{redblack}では、子を4つまで持ちうる木をシミュレートすることで、$O(\log #n#)$の\emph{最悪実行時間}を達成する方法を説明する。

\section{ディスカッションと練習問題}

二分木は血縁関係のモデルとして数千年にわたって使われてきた。
二分木を使うことで、家系図を自然にモデル化できる。
\ejindex{family tree}{かけいず@家系図}%
\index{pedigree family tree}%
ある家系図の書き方では、ある人物を根に配し、その人物の両親を左右の子ノードとする。
生物学における系統樹でも、数世紀にわたって二分木が使われてきた。
\ejindex{species tree}{けいとうじゅ@系統樹}
生物学では、現存の種を二分木における葉で表し、\emph{分化}の発生を内部ノードで表す。
分化とは、1つの種から2つの別々の種が派生することである。

二分探索木は、1950年代に複数のグループが独立に発見したようである
\cite[Section~6.2.2]{k97v3}。
個々の二分探索木に関する詳細な文献は後の各章で紹介する。

二分木をゼロから実装するときには、設計上の考慮が必要になる点がいくつかある。
その1つは、各ノードが親へのポインタを持つかどうかである。
根から葉への経路を辿るだけの操作が多いなら、親へのポインタは不要である。親へのポインタを組み込めばメモリが無駄になり、バグの原因ともなりうる。
一方、親へのポインタがないと、走査のために再帰を使うことになる(もしくは明示的にスタックを利用することになる)。
また、ある種の二分探索木における挿入や削除など、実装が複雑になってしまう操作もある。

もう1つの設計上のポイントは、親と左右の子へのポインタをどう持つかである。
本章の実装では、それぞれを別々の変数に保持していた。
そうではなく、長さ3の配列#p#を使って、
#u.p[0]#、#u.p[1]#、#u.p[2]#がそれぞれ#u#の左右の子と親へのポインタを保持するようにしてもよい。
配列を使うことで、プログラム内の#if#文の連続を代数的な表現でより単純に書けるようになる。

例えば、木を辿る処理をより単純に書ける。
#u.p[i]#から#u#に来たとき、次に向かうのは$#u.p#[(#i#+1)\bmod 3]$である。
左右の対称性があるときにも似たようなことができる。
すなわち、#u.p[i]#の兄弟が$#u.p#[(#i#+1)\bmod 2]$になる。
これは、#u.p[i]#が左の子($#i#=0$)であっても右の子($#i#=1$)であっても有効だ。
この表現を使うことで、左右に分けてそれぞれ書いていた複雑なコードを1つにまとめられる場合がある。
\pageref{page:rotations}ページの#rotateLeft(u)#と#rotateRight(u)#がその好例である。

\begin{exc}
  $#n#\ge 1$個のノードからなる二分木は$#n#-1$本の辺を持つことを示せ。
\end{exc}

\begin{exc}
  $#n#\ge 1$個の(本物の)ノードからなる二分木は$#n#+1$個の外部ノードを持つことを示せ。
\end{exc}

\begin{exc}
  二分木$T$が葉を1つ以上持つとき、$T$における根の子の数が1以下であるか、$T$が葉を2つ以上持つかのいずれかであることを示せ。
\end{exc}

\begin{exc}
ノード#u#を根とする部分木の大きさを計算する再帰的でないメソッド#size2(u)#を実装せよ。
\end{exc}

\begin{exc}
ノード#u#の高さを計算する再帰的でないメソッド#height2(u)#を実装せよ。
\end{exc}

\begin{exc}
二分木が\emph{サイズでバランスされている(size-balanced)}とは、
  \ejindex{size-balanced}{さいずでばらんすされた@サイズでバランスされた}%
  \ejindex{binary search tree!size-balanced}{にぶんたんさくぎ@二分探索木!さいずでばらんすされた@サイズでバランスされた}%
任意のノード#u#について、#u.left#を根とする部分木のサイズと、#u.right#を根とする部分木のサイズとの差が1以下であることをいう。
二分木がこの意味でバランスされているかを判定する再帰的なメソッド#isBalanced()#を書け。
なお、このメソッドの実行時間は$O(#n#)$でなければならない
(さまざまな形状の大きい木でテストしてみること。$O(#n#)$よりも時間を使う実装は簡単である)。
\end{exc}

\ejindex{traversal!pre-order}{そうさ@走査!いきがけじゅん@行きがけ順}%
\ejindex{traversal!in-order}{そうさ@走査!とおりがけじゅん@通りがけ順}%
\ejindex{traversal!post-order}{そうさ@走査!かえりがけじゅん@帰りがけ順}%
\ejindex{pre-order traversal}{いきがけじゅんでのそうさ@行きがけ順での走査}%
\ejindex{in-order traversal}{とおりがけじゅんでのそうさ@通りがけ順での走査}%
\ejindex{post-order traversal}{かえりじゅんでのそうさ@帰りがけ順での走査}%
\emph{行きがけ順(pre-order)}とは、二分木の訪問順であって、ノード#u#をそのいずれの子よりも先に訪問するものである。
\emph{通りがけ順(in-order)}とは、二分木の訪問順であって、ノード#u#を左の部分木に含まれる子よりも後かつ右の部分木に含まれる子よりも先に訪問するものである。
\emph{帰りがけ順(post-order)}とは、二分木の訪問順であって、ノード#u#を#u#を根とする部分木に含まれるいずれの子よりも後に訪問するものである。
行きがけ番号、通りがけ番号、帰りがけ番号とは、それぞれの対応する順序に従って頂点を訪問したときにノードに付される訪問順の番号である。
\figref{binarytree-numbering}に例を示す。

\begin{figure}
  \begin{center}
    \includegraphics[scale=0.90909]{figs/binarytree-numbering-1}
    \includegraphics[scale=0.90909]{figs/binarytree-numbering-2} \\[2ex]
    \includegraphics[scale=0.90909]{figs/binarytree-numbering-3}
  \end{center}
  \caption{二分木における行きがけ順、通りがけ順、帰りがけ順}
  \figlabel{binarytree-numbering}
\end{figure}

\begin{exc}
  \ejindex{number!pre-order}{ばんごう@番号!いきがけじゅん@行きがけ順}%
  \ejindex{number!post-order}{ばんごう@番号!とおりがけじゅん@通りがけ順}%
  \ejindex{number!in-order}{ばんごう@番号!かえりがけじゅん@帰りがけ順}%
  \ejindex{pre-order number}{いきがけじゅんばんごう@行きがけ順番号}%
  \ejindex{post-order number}{とおりじゅんばんごう@通りがけ順番号}%
  \ejindex{in-order number}{かえりじゅんばんごう@帰りがけ順番号}%
#BinarySearchTree#のサブクラスとして、ノードのフィールドに行きがけ番号、通りがけ番号、帰りがけ番号を持つものを作れ。% TODO YJ フィールドはどこかで定義されていたか?
これらの値を適切に割り当てる再帰的な関数#preOrderNumber()#、#inOrderNumber()#、#postOrderNumber()#を書け。
なお、いずれの実行時間も$O(#n#)$でなければならない。
\end{exc}

\begin{exc}\exclabel{tree-traversal}
再帰的でない関数#nextPreOrder(u)#、#nextInOrder(u)#、#nextPostOrder(u)#を実装せよ。
これらは各順序におけるノード#u#の次のノードを返す関数である。
いずれの償却実行時間も高々定数でなければならない。
なお、ノード#u#から始めて、この関数を繰り返し呼んでノードを辿り、$#u#=#null#$になるまでこれを続けるとき、すべての呼び出しの合計コストは$O(#n#)$でなければならない。
\end{exc}

\begin{exc}
ノードに行きがけ番号、通りがけ番号、帰りがけ番号が付された二分木があるとする。
この番号を使って次の質問に定数時間で答える方法を考えよ。
  \begin{enumerate}
    \item ノード#u#が与えられたとき、#u#を根とする部分木の大きさを求めよ。% YJ 部分木の大きさの定義はされているか?
    \item ノード#u#が与えられたとき、#u#の深さを求めよ。
    \item ノード#u#と#w#が与えられたとき、#u#が#w#の祖先であるかを判定せよ。
  \end{enumerate}
\end{exc}

\begin{exc}
ノードに対する行きがけ番号、通りがけ番号の組からなるリストが与えられたとする。
このような行きがけ番号、通りがけ番号が付される木は一意に定まることを示せ。
また、具体的にこの木を構成する方法を与えよ。
\end{exc}

\begin{exc}
#n#個のノードからなる二分木は$2(#n#-1)$ビット以下で表現できることを示せ。

ヒント:木を走査する際に起きることを記録し、これを再生して木を再構築することを考えるとよい。
\end{exc}

\begin{exc}
\figref{bst}の二分木に$3.5$を追加し、続けて$4.5$を追加するときの様子を図示せよ。
\end{exc}

\begin{exc}
\figref{bst}の二分木に$3$を削除し、続けて$5$を削除するときの様子を図示せよ。
\end{exc}

\begin{exc}
#BinarySearchTree#のメソッド#getLE(x)#を実装せよ。
これは、木に含まれる要素のうち、#x#以下のものを集めたリストを返すものである。
このメソッドの実行時間は$O(#n#'+#h#)$でなければならない。
ここで、$#n#'$は木に含まれる#x#以下の要素の数、#h#は木の高さである。
\end{exc}

\begin{exc}
空の#BinarySearchTree#に$\{1,\ldots,#n#\}$をすべて追加し、結果として得られる木の高さが$#n#-1$になるためにはどうすればよいか。
また、そのやり方は何通りあるか。
\end{exc}

\begin{exc}
ある#BinarySearchTree#に#add(x)#を実行し、同じ#x#について#remove(x)#を実行すると、木は必ず元の状態に戻るか。
\end{exc}

\begin{exc}
#BinarySearchTree#において#remove(x)#を実行するとき、あるノードの高さが大きくなることはあるか。
もしそうなら、どのくらい大きくなりうるか。
\end{exc}

\begin{exc}
#BinarySearchTree#において#add(x)#を実行するとき、あるノードの高さが大きくなることはあるか。
また、そのとき木の高さが大きくなることはあるか。
もしそうなら、どれくらい大きくなりうるか。
\end{exc}

\begin{exc}
各ノード#u#が#u.size#(#u#を根とする部分木の大きさ)、#u.depth#(#u#の深さ)、#u.height#(#u#を根とする部分木の高さ)を保持するような#BinarySearchTree#の変種を設計、実装せよ。

なお、#add(x)#と#remove(x)#を呼んでもこれらの値は適切に保たれる必要がある。#add(x)#や#remove(x)#のコストが定数時間より大きくはならないように注意すること。
\end{exc}
