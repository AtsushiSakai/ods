\chapter{二分木}
\chaplabel{binarytrees}

この章では、コンピュータサイエンスで現れる最も基本的な構造のうちのひとつ、二分木を紹介する。
この構造を\emph{木}と呼ぶのは、図示したときに(森に生えている)木に似ているためである。
\ejindex{tree}{き@木}%
\ejindex{tree!binary}{き@木!にぶん@二分}%
\ejindex{binary tree}{にぶんぎ@二分木}%
二分木には複数の定義がある。
数学的には、\emph{二分木}とは連結
\footnote{訳注:グラフが連結(connected)であるとは、グラフ上の辺を辿ることで任意の2頂点間を行き来できることである。そうして辿った辺の列のことを経路(path)と呼ぶ。}な有限無向グラフであって、サイクルを持たず、すべての頂点の次数が3以下のものである
\footnote{訳注:無向グラフの頂点における次数(degree)とは、その頂点が持つ辺の数である。たとえば\figref{bintree-orientation}において、uの次数は3である。}。

コンピュータサイエンスにおける応用では、二分木はふつう\emph{根を持つ}。
\ejindex{tree!rooted}{き@木!ねつき@根付き}%
\ejindex{rooted tree}{ねつきぎ@根付き木}%
木の\emph{根}と呼ばれる特殊なノード#r#があり、このノードの次数は2以下である。
%次数2以下のノードのうち、特別なもの#r#を、木の\emph{根}と呼ぶ。
ノード$#u#(\neq #r#)$から#r#に向かう経路における二番目のノードを#u#の\emph{親}という。
\ejindex{parent}{おや@親}%
#u#に隣接する親以外のノードを#u#の\emph{子}という。
特に\emph{順序付けられている}二分木に興味があることが多いので、
\ejindex{ordered tree}{じゅんじょづけられたき@順序付けられた木}%
\ejindex{tree!ordered}{き@木!じゅんじょづけられた@順序付けられた}%
子を\emph{左の子}・\emph{右の子}と呼び分けることにする。
\ejindex{left child}{ひだりのこ@左の子}%
\ejindex{child!left}{こ@子!ひだり@左}%
\ejindex{right child}{みぎのこ@右の子}%
\ejindex{child!right}{こ@子!みぎ@右}%

二分木を図示するとき、ふつう根から下に向かって描く。
根が一番上に描かれる。また、ノード#u#の左右の子は、#u#の左下・右下にそれぞれ描かれる。
(\figref{bintree-orientation})
\figref{binary-tree}.a は9個のノードを持つ二分木の例である。

\begin{figure}
  \begin{center}
    \includegraphics[scale=0.90909]{figs/bintree-traverse-1}
  \end{center}
  \caption{#BinaryTree#における、ノード#u#の親・左の子・右の子}
  \figlabel{bintree-orientation}
\end{figure}


\begin{figure}
  \begin{center}
    \begin{tabular}{cc}
      \includegraphics[width=\HalfScaleIfNeeded]{figs/bintree-1} &
      \includegraphics[width=\HalfScaleIfNeeded]{figs/bintree-2} \\
      (a) & (b)
    \end{tabular}
  \end{center}
  \caption{(a)~9個の本物のノードを持つ二分木と、(b)~10個の外部ノードを持つ二分木}
  \figlabel{binary-tree}
\end{figure}

% XXX: 別に二分木専用の用語ではないような気が...? % YJ 木に訂正しましょう
木(および二分木)は重要なので、その性質を記述するための専用の語彙が使われている。
木におけるノード#u#の\emph{深さ}とは、
\ejindex{depth}{ふかさ@深さ}%
#u#から根までの経路の長さである
\footnote{訳注:たとえば#u#=#r#であるとき、#u#の深さは0である。}。
ノード#w#が#u#から#r#への経路に含まれるとき、#w#を#u#の\emph{祖先}という。
\ejindex{ancestor}{そせん@祖先}%
一方、このとき#u#を#w#の\emph{子孫}という。
\ejindex{descendant}{しそん@子孫}%
木におけるノード#u#の\emph{部分木}とは、#u#を根とし、#u#のすべての子孫を含む木である。
ノード#u#の\emph{高さ}とは、#u#から#u#の子孫への経路の長さの最大値である。
\ejindex{height!in a tree}{たかさ@高さ!き@木}
木の\emph{高さ}とはその根の高さである。
ノード#u#が子を持たないなら、#u#は\emph{葉}である。
\ejindex{leaf}{は@葉}%

\emph{外部ノード}という概念を考えると便利なことがある。
左の子を持たないノードは左の子として外部ノードを持ち、右の子を持たないノードは右の子として外部ノードを持つとする。(\figref{binary-tree}.bを参照せよ。)
帰納法により、$#n#\ge 1$個の(本物の)ノードを持つ二分木は、$#n#+1$個の外部ノードを持つことを示せる。
% NOTE: n = 0 の場合はどう扱うのが自然なのだろう? % Empty graphは木でなくForestとして扱うべきという話があるらしい https://link.springer.com/chapter/10.1007%2FBFb0066433

\section{#BinaryTree#:基本的な二分木}

\index{BinaryTree@#BinaryTree#}%
二分木におけるノード#u#を表現するには、#u#に隣接する(3つ以下の)ノードを明示的に保持するのが簡単だ。
\javaimport{ods/BinaryTree.BTNode<Node}
\cppimport{ods/BinaryTree.BTNode}
隣接する頂点が3つ揃っていないときには、そこは#nil#とする。
このとき、外部ノードと根の親とが#nil#に対応する。

すると、二分木自体は根#r#への参照として表現できる
\footnote{訳注:二分木を連結なグラフとして定義したことを思い出そう。木に含まれる何らかのノードへの参照が1つあれば、他のノードへたどり着ける。そのような参照として根を選ぶ理由は、後の章で出てくるような木(たとえばB木。\secref{btree-elem-add}を参照せよ。)では、親への参照が存在せず、根以外の参照を選んだ場合に複数持つ必要が生じるためだ。}。
%すると、根#r#への参照として二分木自体は表現できる。 % YJ こういう細かい語順を変えるだけでかなり分かりやすくなると思うので、細かいと思いますが直します
\codeimport{ods/BinaryTree.r}

ノード#u#の深さを、#u#から根への経路を辿るときのステップ数として計算できる。
\codeimport{ods/BinaryTree.depth(u)}


\subsection{再帰的なアルゴリズム}

\ejindex{recursive algorithm}{さいきあるごりずむ@再帰アルゴリズム}%
再帰アルゴリズムを使うと二分木に関する計算が簡単になる。
例えば#u#を根とする二分木のサイズ(ノードの数)は、#u#の子を根とする部分木のサイズを再帰的に計算し、足し合わせ、その結果に1加えると求まる。

\codeimport{ods/BinaryTree.size(u)}

ノード#u#の高さは、#u#のふたつの部分木の高さの最大値を計算し、その結果に1加えると求まる
\footnote{訳注:完全な余談だが、このように二分木の高さを求めるのは、米国におけるソフトウェアエンジニア面接の頻出問題の一つである。https://leetcode.com/problems/maximum-depth-of-binary-tree/ }。

\codeimport{ods/BinaryTree.height(u)}

\subsection{二分木の走査}
\seclabel{bintree:traversal}

\ejindex{traversal!of a binary tree}{そうさ@走査!にぶんぎ@二分木}%
\ejindex{tree traversal}{きのそうさ@木の走査}%
\ejindex{binary-tree traversal}{にぶんぎのそうさ@二分木の走査}%
先の小節で説明したふたつのアルゴリズムは二分木のすべてのノードを訪問するために再帰を使った。
いずれのアルゴリズムも二分木のノードを次のコードと同じ順番で訪問していた。
% NOTE: (C++では?)部分表現の評価順序は未規定で、訪問順は必ずしもこうなるわけではないのではないか % TODO: YJ C, C++の部分評価順序は規定されていないのでこれは正しくない。著者に言うべきか。
\codeimport{ods/BinaryTree.traverse(u)}
%\cpponly{C++の部分表現の評価順序は規定されていないので同じ順番とは限らない。}

再帰を使うとこのように簡潔なコードを書けるが、時に困ることもある。
再帰の深さの最大値は、二分木におけるノードの深さの最大値、すなわち木の高さである。
これが非常に大きいと、再帰のためのスタックとして、利用できる量以上の領域を要求し、プログラムがクラッシュしてしまうことがある
\footnote{訳注:この問題はスタックオーバーフローと呼ばれる。}。

再帰なしで二分木を走査するためには、どこから来たかによって次の行き先を決めるアルゴリズムを使えばよい。
\figref{bintree-traverse}を参照せよ。
ノード#u#に#u.parent#から来たときは、次は#u.left#に向かう。
#u.left#から来たときは、次は#u.right#に向かう。
#u.right#から来たときは、#u#の部分木を辿り終えたので#u.parent#に戻る。
次のコードはこれを実装したものである。
ただし、#u.left#・#u.right#・#u.parent#が#nil#であるケースも適切に処理している。
\codeimport{ods/BinaryTree.traverse2()}

\begin{figure}
  \begin{center}
    \begin{tabular}{cc}
      \includegraphics[scale=0.90909]{figs/bintree-traverse-2}
      \includegraphics[scale=0.90909]{figs/bintree-3}
    \end{tabular}
  \end{center}
  \caption{二分木を再帰を使わずに走査するときの、三通りのノード#u#の訪れ方と、その結果として生じる木の走査}
  \figlabel{bintree-traverse}
\end{figure}

再帰アルゴリズムで計算できることは、こうして再帰なしでも計算できる。
例えば木のサイズを計算するためには、カウンタ#n#を保持し、新しいノードを訪問するたびにその値をひとつずつ増やせばよい。
\codeimport{ods/BinaryTree.size2()}

二分木の実装には、#parent#を使わないこともある。
この場合にも再帰を使わない実装は可能だが、#List#か#Stack#を使って、今訪問しているノードから根までの経路を記録しておく必要がある。

ここまで説明したものとは別の走査方法として、\emph{幅優先}な走査がある
\footnote{訳注:\secref{general-bfs}では、一般的な場合の幅優先探索のアルゴリズムを紹介する。}。
\ejindex{breadth-first traversal}{はばゆうせんそうさ@幅優先走査}%
\ejindex{traversal!breadth-first}{そうさ@走査!はばゆうせん@幅優先}%
幅優先に走査する場合、根から下に向かって深さごとに、同じ深さのノードは左から右の順に、すべてのノードを訪問する。
(\figref{bintree-bfs}を参照せよ。)
これは英語の文章の読み方と似ている。(訳注:横書きなら、日本語でも同様である。)
幅優先の走査はキュー#q#を使って実装できる。
初期状態では#q#は根だけを含む。
各ステップでは、#q#から次のノード#u#を取り出し、#u#を処理し、#u.left#と#u.right#を(#nil#でなければ)#q#に追加する。
\codeimport{ods/BinaryTree.bfTraverse()}

\begin{figure}
  \begin{center}
    \includegraphics[scale=0.90909]{figs/bintree-4}
  \end{center}
  \caption{幅優先な走査では、二分木の各ノードは深さ毎に、また各深さでは左から右の順で訪問される。}
  \figlabel{bintree-bfs}
\end{figure}

\section{#BinarySearchTree#:バランスされていない二分探索木}
\seclabel{binarysearchtree}

\index{BinarySearchTree@#BinarySearchTree#}%
\ejindex{binary search tree}{にぶんたんさくぎ@二分探索木}%
\ejindex{binary tree!search}{にぶんぎ@二分木!たんさく@探索}%
#BinarySearchTree#(二分探索木)はノード#u#がある全順序な集合の要素であるデータ#u.x#を持つ二分木である。
二分探索木の各ノードとそのデータは次の\emph{二分探索木の性質}を満たす。 % TODO: YJ これはpropertyなのかdefinitionなのか
\ejindex{binary search tree property}{にぶんたんさくぎのせいしつ@2分探索木の性質}%
ノード#u#について、#u.left#を根とする部分木に含まれるデータはすべて#u.x#より小さく、#u.right#を根とする部分木に含まれるデータはすべて#u.x#より大きい。
#BinarySearchTree#の例を\figref{bst}に示す。

\begin{figure}
  \begin{center}
    \includegraphics[scale=0.90909]{figs/bst-example}
    %\includegraphics[scale=0.90909]{figs/binary-tree-4}
  \end{center}
  \caption{二分探索木の例}
  \figlabel{bst}
\end{figure}


\subsection{探索}

\ejindex{search path!in a binary search tree}{たんさくけいろ@探索経路!にぶんたんさくぎ@二分探索木}%
二分探索木の性質は有用だ。
この性質を利用して、二分探索木から値#x#を高速に見つけられる。
具体的には、まず根#r#から#x#を探し始める。
ノード#u#に訪問しているとき、次の3つの場合がありうる。
\begin{enumerate}
\item $#x#< #u.x#$なら#u.left#に進む。
\item $#x#> #u.x#$なら#u.right#に進む。
\item $#x#= #u.x#$なら値が#x#であるノード#u#を見つけた。
\end{enumerate}
この探索は三つ目のケース、または#u=nil#になると終了する。
前者なら#x#が見つかったことになる。
後者なら#x#がこの木に含まれていないとわかる。
\codeimport{ods/BinarySearchTree.findEQ(x)}

二分探索木における探索の例を\figref{bst-search}にふたつ示す。
二つ目の例から、#x#が見つからない場合でも、役に立つ情報が得られることがわかる。
探索における最後のノード#u#にて、先の場合分けの一つ目のケースであったなら、#u.x#は木に含まれるデータであって#x#よりも大きい値のうち、最小のものである。
同様に場合分けの二つ目のケースであったなら、#u.x#は#x#より小さい値のうち、最大のものである。
よって、場合分けの一つ目のケースが最後に発生したノード#z#を記録しておけば、#BinarySearchTree#の#find(x)#を、#x#以上の値のうち最小のものを返すよう実装することもできる。
\codeimport{ods/BinarySearchTree.find(x)}

\begin{figure}
  \begin{center}
    \begin{tabular}{cc}
    \includegraphics[width=\HalfScaleIfNeeded]{figs/bst-example-2} &
    \includegraphics[width=\HalfScaleIfNeeded]{figs/bst-example-3} \\
    (a) & (b)
    \end{tabular}
  \end{center}
  \caption{二分探索木において、(a)~探索が成功する例($6$が見つかる)と、(b)~探索が失敗する($10$が見つからない)例}
  \figlabel{bst-search}
\end{figure}


\subsection{追加}

#BinarySearchTree#に値#x#を追加するために、まずは#x#を検索する。
もし見つかれば挿入の必要がない。
見つからなければ、検索において最後に出会ったノード#p#の子である葉として、#x#を保存する。
このとき、新しいノードが#p#の右の子か左の子かを、#x#と#p.x#の比較結果によって決める。
\codeimport{ods/BinarySearchTree.add(x)}
\codeimport{ods/BinarySearchTree.findLast(x)}
\codeimport{ods/BinarySearchTree.addChild(p,u)}
\figref{bst-insert}に例を示した。
% XXX: 原著が間違ってる? According to the definition at the head of this chapter, it seems to be appropriate to use not "height" but "depth" here.
% > An example is shown in \figref{bst-insert}. The most time-consuming
% part of this process is the initial search for #x#, which takes an
% amount of time proportional to the height of the newly added node #u#.
最も時間がかかるのは#x#を検索する処理で、この時間は新たに追加するノード#u#の深さに比例する。 % YJ 深さかと
最悪の場合にはこれは#BinarySearchTree#の高さである。

\begin{figure}
  \begin{center}
    \begin{tabular}{cc}
    \includegraphics[width=\HalfScaleIfNeeded]{figs/bst-example-4} &
    \includegraphics[width=\HalfScaleIfNeeded]{figs/bst-example-5}
    \end{tabular}
  \end{center}
  \caption{二分探索木に$8.5$を追加する様子}
  \figlabel{bst-insert}
\end{figure}


\subsection{削除}

#BinarySearchTree#から、ある値を格納するノード#u#を削除する処理はもう少し複雑だ。
#u#が葉なら、#u#を単に親から切り離せばよい。
#u#がひとつだけの子を持つなら、#u#の点を継ぎ合わせる、すなわち#u.parent#と#u#の子とを新たに親子関係とすればよい。
(\figref{bst-splice}を参照せよ。)
\codeimport{ods/BinarySearchTree.splice(u)}

\begin{figure}
  \begin{center}
    \includegraphics[scale=0.90909]{figs/bst-splice}
  \end{center}
  \caption{葉($6$)、または一つだけの子を持つノード($9$)を削除するのは簡単である。}
  \figlabel{bst-splice}
\end{figure}

#u#がふたつの子を持つ場合はもっと手の込んだことをする必要がある。
この場合、子の数が1以下のノード#w#であって、#w.x#と#u.x#とを入れ替えられるものを見つけるのが最も単純なやり方だ。
二分探索木の性質を保つためには、#w.x#の値と#u.x#の値が近ければよい。
例えば、#w.x#が#u.x#より大きい中で最小の値であればよい。
このような#w#は簡単に見つけられる。
これは#u.right#を根とする部分木の中で最小の値である。
このノードは左の子を持たないため、取り除くのも簡単である。
(\figref{bst-remove}を参照せよ。)
\javaimport{ods/BinarySearchTree.remove(u)}
\cppimport{ods/BinarySearchTree.remove(u)}
\pcodeimport{ods/BinarySearchTree.remove_node(u)}

\begin{figure}
  \begin{center}
    \begin{tabular}{cc}
    \includegraphics[width=\HalfScaleIfNeeded]{figs/bst-delete-1}
    \includegraphics[width=\HalfScaleIfNeeded]{figs/bst-delete-2}
    \end{tabular}
  \end{center}
  \caption{ふたつの子を持つノード#u#から値$11$を削除するために、#u#の値と、#u#の右の部分木における最小の値とを入れかえる。}
  \figlabel{bst-remove}
\end{figure}

\subsection{要約}

#BinarySearchTree#における#find(x)#・#add(x)#・#remove(x)#の処理は、いずれも根から特定のノードへの経路を辿る処理を伴う。
木の形状についてなにか仮定しない限り、この経路の長さについて「木の中のノード数は超えない」より強い主張をするのは難しい。
次の(あまりパッとしない)定理は#BinarySearchTree#の性能をまとめたものだ。

\begin{thm}\thmlabel{bst}
  #BinarySearchTree#は#SSet#インターフェースの実装であって、#add(x)#・#remove(x)#・#find(x)#の実行時間は$O(#n#)$である。
\end{thm}

\thmref{skiplist}と比べると、\thmref{bst}の性能は良くない。
#SkiplistSSet#は#SSet#インターフェースの操作を期待実行時間$O(\log #n#)$で実装していた。
#BinarySearchTree#の問題は、木の形状が\emph{アンバランス}かもしれないことだ。
\figref{bst}のような木の形ではなく、ほとんどのノードがひとつだけの子を持ち、#n#個のノードからなる長い鎖のような見た目かもしれないのである
\footnote{訳注:このような長い鎖のような見た目をしたデータ構造は見覚えがあるかもしれない。\chapref{linkedlists}で扱った連結リストを思い出そう(細かな違いはあるが)。}。

アンバランスな二分探索木を避ける方法はたくさんあり、そうすると$O(\log #n#)$の時間で各操作を行えるようになる。
\chapref{rbs}で\emph{期待実行時間}$O(\log #n#)$を、ランダム性を利用して達成する方法を説明する。
\chapref{scapegoat}では\emph{償却実行時間}$O(\log #n#)$を、部分的な再構築を利用して達成する方法を説明する。
\chapref{redblack}では\emph{最悪実行時間}$O(\log #n#)$を、4つまで子を持ちうる木をシミュレートすることで達成する方法を説明する。

\section{ディスカッションと練習問題}

二分木は血縁関係のモデルとして数千年に渡って使われている。
二分木は家系図を自然にモデルすることができる。
\ejindex{family tree}{かけいず@家系図}%
\index{pedigree family tree}%
ある家系図の書き方では、根をある人、左右の子ノードをその人の両親とする。
数世紀前からは生物学における系統樹にも二分木は使われている。
\ejindex{species tree}{けいとうじゅ@系統樹}
ここでは葉は現存の種を表し、内部ノードは\emph{分化}が起きたことを表す。
分化とは一つの種から、二つの別々の種が派生することである。

二分探索木は、1950年代に複数のグループが独立に発見したようである。
\cite[Section~6.2.2]{k97v3}
個々の二分探索木のより詳細な文献は後の章で紹介する。

ゼロから二分木を実装するとき、いくつか設計上考えることがある。
ひとつは各ノードが親へのポインタを持つかどうかである。
多くの操作が根から葉への経路を辿るだけのものなら、親へのポインタは不要で、メモリを無駄にし、バグを入れ込む原因となりうる。
一方親へのポインタがないと、走査のために再帰を(または明示的にスタックを)使うことになる。
また、実装が複雑になってしまう操作もある。(ある種の二分探索木における挿入や削除など)

もうひとつの設計上のポイントは、親と左右の子へのポインタをどう持つかである。
この章の実装では、それぞれを別々の変数に保持していた。
一方、長さ3の配列#p#を使うこともできる。
この場合、#u.p[0]#・#u.p[1]#・#u.p[2]#がそれぞれ、#u#の左右の子と親へのポインタを保持する。
配列を使うと、プログラム内の#if#文の連続を、代数的な表現でより単純に書ける。

例えば、この単純化は木を辿るときに役立つ。
#u.p[i]#から#u#に来たとき、次に向かうのは$#u.p#[(#i#+1)\bmod 3]$である。
左右の対称性があるときにも似たようなことができる。
すなわち、#u.p[i]#の兄弟は$#u.p#[(#i#+1)\bmod 2]$である。
これは#u.p[i]#が左の子($#i#=0$)であっても右の子($#i#=1$)であっても有効だ。
この表現を使うと、左右の場合ためにそれぞれ書いていた複雑なコードを、ひとつにまとめられることがある。
例として\pageref{page:rotations}の#rotateLeft(u)#・#rotateRight(u)#を参照せよ。

\begin{exc}
  $#n#\ge 1$個のノードからなる二分木は$#n#-1$本の辺を持つことを示せ。
\end{exc}

\begin{exc}
  $#n#\ge 1$個の(本物の)ノードからなる二分木は$#n#+1$個の外部ノードを持つことを示せ。
\end{exc}

\begin{exc}
  二分木$T$が一つ以上葉を持つとき、$T$における根の子の数が1以下であるか、$T$は二つ以上の葉を持つかのいずれかであることを示せ。
\end{exc}

\begin{exc}
ノード#u#を根とする部分木の大きさを計算する再帰的でないメソッド#size2(u)#を実装せよ。
\end{exc}

\begin{exc}
ノード#u#の高さを計算する再帰的でないメソッド#height2(u)#を実装せよ。
\end{exc}

\begin{exc}
二分木が\emph{サイズでバランスされている}とは、
  \ejindex{size-balanced}{さいずでばらんすされた@サイズでバランスされた}%
  \ejindex{binary search tree!size-balanced}{にぶんたんさくぎ@二分探索木!さいずでばらんすされた@サイズでバランスされた}%
任意のノード#u#について、#u.left#を根とする部分木のサイズと、#u.right#を根とする部分木のサイズとの差が1以下であることをいう。
二分木がこの意味でバランスされているか判定する再帰的なメソッド#isBalanced()#を書け。
なお、このメソッドの実行時間は$O(#n#)$でなければならない。
(色々な形状の大きい木でテストしてみること。$O(#n#)$より多く時間を使う実装は簡単である。)
\end{exc}

\ejindex{traversal!pre-order}{そうさ@走査!いきがけじゅん@行きがけ順}%
\ejindex{traversal!post-order}{そうさ@走査!とおりがけじゅん@通りがけ順}%
\ejindex{traversal!in-order}{そうさ@走査!かえりがけじゅん@帰りがけ順}%
\ejindex{pre-order traversal}{いきがけじゅんでのそうさ@行きがけ順での走査}%
\ejindex{post-order traversal}{とおりがけじゅんでのそうさ@通りがけ順での走査}%
\ejindex{in-order traversal}{かえりじゅんでのそうさ@帰りがけ順での走査}%
\emph{行きがけ順}とは、二分木の訪問順であって、ノード#u#をそのいずれの子よりも先に訪問するものである。
\emph{通りがけ順}とは、二分木の訪問順であって、ノード#u#を左の部分木に含まれる子よりも後かつ右の部分木に含まれる子よりも先に訪問するものである。
\emph{帰りがけ順}とは、二分木の訪問順であって、ノード#u#を#u#を根とする部分木に含まれるいずれの子よりも後に訪問するものである。
行きがけ番号・通りがけ番号・帰りがけ番号とは、各対応する順序に従って頂点を訪問した時のノードに付された訪問順の番号である。
例として\figref{binarytree-numbering}を見よ。

\begin{figure}
  \begin{center}
    \includegraphics[scale=0.90909]{figs/binarytree-numbering-1}
    \includegraphics[scale=0.90909]{figs/binarytree-numbering-2} \\[2ex]
    \includegraphics[scale=0.90909]{figs/binarytree-numbering-3}
  \end{center}
  \caption{二分木における行きがけ順・通りがけ順・帰りがけ順}
  \figlabel{binarytree-numbering}
\end{figure}

\begin{exc}
  \ejindex{number!pre-order}{ばんごう@番号!いきがけじゅん@行きがけ順}%
  \ejindex{number!post-order}{ばんごう@番号!とおりがけじゅん@通りがけ順}%
  \ejindex{number!in-order}{ばんごう@番号!かえりがけじゅん@帰りがけ順}%
  \ejindex{pre-order number}{いきがけじゅんばんごう@行きがけ順番号}%
  \ejindex{post-order number}{とおりじゅんばんごう@通りがけ順番号}%
  \ejindex{in-order number}{かえりじゅんばんごう@帰りがけ順番号}%
#BinarySearchTree#のサブクラスとしてノードのフィールドに行きがけ番号・通りがけ番号・帰りがけ番号を持つものを作れ。 % TODO YJ フィールドはどこかで定義されていたか?
これらの値を適切に割り当てる再帰的な関数#preOrderNumber()#・#inOrderNumber()#・#postOrderNumber()#を書け。
なお、いずれの実行時間も$O(#n#)$でなければならない。
\end{exc}

\begin{exc}\exclabel{tree-traversal}
再帰的でない関数#nextPreOrder(u)#・#nextInOrder(u)#・#nextPostOrder(u)#を実装せよ。
これらは各順序におけるノード#u#の次のノードを返す関数である。
いずれの償却実行時間も高々定数でなければならない。
また、ノード#u#からはじめて、この関数を繰り返し呼んでノードを辿り、$#u#=#null#$になるまでこれを続けるとき、すべての呼び出しの合計コストは$O(#n#)$でなければならない。
\end{exc}

\begin{exc}
ノードに行きがけ番号・通りがけ番号・帰りがけ番号が付された二分木があるとする。
この番号を使って次の質問に定数時間で答える方法を考えよ。
  \begin{enumerate}
    \item ノード#u#が与えられたとき、#u#を根とする部分木の大きさを求めよ。 % YJ 部分木の大きさの定義はされているか?
    \item ノード#u#が与えられたとき、#u#の深さを求めよ。
    \item ノード#u#と#w#が与えられたとき、#u#が#w#の祖先であるかを判定せよ。
  \end{enumerate}
\end{exc}

\begin{exc}
ノードに対する行きがけ番号・通りがけ番号の組みのリストが与えられたとする。
このような行きがけ番号・通りがけ番号が付される木は一意に定まることを示せ。
また具体的にこの木を構成方法を与えよ。
\end{exc}

\begin{exc}
#n#個のノードからなる二分木は$2(#n#-1)$ビット以下で表現できることを示せ。
(ヒント:木を走査する際に起きることを記録し、これを再生して木を再構築することを考るとよい。)
\end{exc}

\begin{exc}
\figref{bst}の二分木に$3.5$を追加し、続けて$4.5$を追加するときの様子を図示せよ。
\end{exc}

\begin{exc}
\figref{bst}の二分木に$3$を削除し、続けて$5$を削除するときの様子を図示せよ。
\end{exc}

\begin{exc}
#BinarySearchTree#のメソッド#getLE(x)#を実装せよ。
これは木に含まれる要素のうち、#x#以下のものを集めたリストを返すものである。
このメソッドの実行時間は$O(#n#'+#h#)$でなければならない。
ここで$#n#'$は木に含まれる#x#以下の要素の数、#h#は木の高さである。
\end{exc}

\begin{exc}
空の#BinarySearchTree#に$\{1,\ldots,#n#\}$をすべて追加し、結果として得られる木の高さが$#n#-1$になるためにはどうすればよいか。
また、このやり方は何通りあるか。
\end{exc}

\begin{exc}
ある#BinarySearchTree#に#add(x)#を実行し、(同じ#x#について)#remove(x)#を実行すると、必ず木は元の状態に戻るか?
\end{exc}

\begin{exc}
#BinarySearchTree#において#remove(x)#を実行するとき、あるノードの高さが大きくなることがあるか?
もしそうなら、どのくらい大きくなりうるか?
\end{exc}

\begin{exc}
#BinarySearchTree#において#add(x)#を実行するとき、あるノードの高さが大きくなることがあるか?
また、そのとき木の高さが大きくなることがあるか?
もしそうなら、どのくらい大きくなりうるか?
\end{exc}

\begin{exc}
#BinarySearchTree#の一種であり、各ノード#u#が#u.size#(#u#を根とする部分木の大きさ)、#u.depth#(#u#の深さ)、#u.height#(#u#を根とする部分木の高さ)を保持するものを設計・実装せよ。

なお、#add(x)#・#remove(x)#を呼んでもこれらの値は適切に保たれる必要があり、一方で、これらの操作のコストが定数時間より大きくはならないように注意すること。
\end{exc}
